\begin{resumo}
% O resumo deve ressaltar o objetivo, o método, os resultados e as conclusões
% do documento. A ordem e a extensão
% destes itens dependem do tipo de resumo (informativo ou indicativo) e do
% tratamento que cada item recebe no documento original. O resumo deve ser
% precedido da referência do documento, com exceção do resumo inserido no
% próprio documento. (\ldots) As palavras-chave devem figurar logo abaixo do
% resumo, antecedidas da expressão Palavras-chave:, separadas entre si por
% ponto e finalizadas também por ponto. O texto pode conter no mínimo 150 e
% no máximo 500 palavras, é aconselhável que sejam utilizadas 200 palavras.
% E não se separa o texto do resumo em parágrafos.

Com um histórico oscilante, o cenário de Startups de Brasília tem suas peculiaridades. A cidade que em 2015 foi classificada com a pior Cultura Empreendedora do Brasil pela Endeavor\footciteref{indiceglobaldoempreendedorismo} e que outrora foi referência nacional com a maior delegação de Startups do Brasil no Tech Crunch Disrupt em 2012\footciteref{menezes2012} graças a uma forte comunidade de empreendedores, passou alguns anos em decadência e hoje, em 2016, volta a crescer.

Aproveitando o momento em que muitas iniciativas dão força para as Startups brasilienses, este trabalho tem como objetivo realizar uma avaliação do atual estado do Ecossistema de Startups do Distrito Federal utilizando uma metodologia criada pelo grupo de pesquisa InovaSampa da Universidade de São Paulo por meio de um Estudo Qualitativo que tem como base entrevistas individuais com diversos Empreendedores locais.

A expecativa de resultado final é que seja obtida uma visão geral de como os diversos fatores do compõem esse Ecossistema interagem entre si, quais as suas características, bem como pontos fortes e fracos, o seu histórico, nível de maturidade, prospecções dos empreendedores para o futuro e como ele pode ser comparado com outros ecossistemas.

 \vspace{\onelineskip}

 \noindent
 \textbf{Palavras-chaves}: Startups, Ecossistema de Startups, Empreendedorismo
\end{resumo}
