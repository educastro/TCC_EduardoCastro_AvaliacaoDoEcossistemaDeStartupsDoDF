\begin{resumo}
% O resumo deve ressaltar o objetivo, o método, os resultados e as conclusões
% do documento. A ordem e a extensão
% destes itens dependem do tipo de resumo (informativo ou indicativo) e do
% tratamento que cada item recebe no documento original. O resumo deve ser
% precedido da referência do documento, com exceção do resumo inserido no
% próprio documento. (\ldots) As palavras-chave devem figurar logo abaixo do
% resumo, antecedidas da expressão Palavras-chave:, separadas entre si por
% ponto e finalizadas também por ponto. O texto pode conter no mínimo 150 e
% no máximo 500 palavras, é aconselhável que sejam utilizadas 200 palavras.
% E não se separa o texto do resumo em parágrafos.

Com um histórico oscilante, o cenário de Startups do Distrito Federal tem suas peculiaridades. A cidade que em 2015 e 2016 foi classificada com a pior Cultura Empreendedora do Brasil pela Endeavor outrora foi citada como referência nacional por, em 2012, ter tido a maior delegação de startups do Brasil no Tech Crunch Disrupt. Atualmente, em 2017, começa a se formar o que pode ser entendido como um ecossistema de startups. Aproveitando o momento em que muitas iniciativas colocam as startups na agenda política e midiática da cidade este trabalho tem como objetivo realizar uma avaliação do atual estado deste ecossistema utilizando uma metodologia para avaliação de maturidade de ecossistemas de startups criada pelo grupo de pesquisa InovaSampa da Universidade de São Paulo.

 \vspace{\onelineskip}

 \noindent
 \textbf{Palavras-chaves}: Startups, Ecossistema de Startups, Empreendedorismo
\end{resumo}
