\begin{resumo}
% O resumo deve ressaltar o objetivo, o método, os resultados e as conclusões
% do documento. A ordem e a extensão
% destes itens dependem do tipo de resumo (informativo ou indicativo) e do
% tratamento que cada item recebe no documento original. O resumo deve ser
% precedido da referência do documento, com exceção do resumo inserido no
% próprio documento. (\ldots) As palavras-chave devem figurar logo abaixo do
% resumo, antecedidas da expressão Palavras-chave:, separadas entre si por
% ponto e finalizadas também por ponto. O texto pode conter no mínimo 150 e
% no máximo 500 palavras, é aconselhável que sejam utilizadas 200 palavras.
% E não se separa o texto do resumo em parágrafos.

Com um histórico oscilante, o cenário de Startups do Distrito Federal tem suas peculiaridades. A cidade que em 2015 e 2016 foi classificada com a pior Cultura Empreendedora do Brasil pela \citeonline{indiceglobaldoempreendedorismo} e outrora fora referência nacional com a maior delegação de Startups do Brasil no Tech Crunch Disrupt em 2012\footciteref{menezes2012} passou alguns anos sem maiores movimentos e atualmente, em 2017, volta a ganhar destaque. Aproveitando o momento em que muitas iniciativas colocam na agenda política e midiática da cidade as startups da cidade este trabalho tem como objetivo realizar uma avaliação do atual estado deste ecossistema utilizando uma metodologia criada pelo grupo de pesquisa InovaSampa da Universidade de São Paulo por meio de estudos qualitativos, com foco em entrevistas individuais com diversos empreendedores locais, e quantitativos.

 \vspace{\onelineskip}

 \noindent
 \textbf{Palavras-chaves}: Startups, Ecossistema de Startups, Empreendedorismo
\end{resumo}
