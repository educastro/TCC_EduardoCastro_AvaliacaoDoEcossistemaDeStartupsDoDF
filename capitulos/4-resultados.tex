\chapter[Resultados]{Resultados}
\label{cap-resultados}

A exploração do Ecossistema de Startups de Brasília tem sido deveras interessante, primeiro pela extensa rede de contatos que venho contruindo graças à essa pesquisa mas também por em tão pouco tempo já permitir uma visualização superficial e uma compreensão de diversos elementos desse Ecossistema e a razão de certas características locais.

Traçando uma espécie de linha temporal do Ecossistema de Startups do Distrito Federal, claramente a figura de uma forte liderança e de um Ecossistema unido e integrado era mais clara entre meados de 2010 e 2012, diversos Empreendedores citaram um grupo chamado ``Startup Brasília'' composto por Empreendedores de empresas como a Intacto, Qual Canal, SEA Tecnologia, Rota dos Concursos, IPê Tecnologia, Trip2gether, etc. 

Nessa época os encontros - também conhecidos como meetups - também eram mais frequentes e, segundo alguns Empreendedores, de maior qualidade. Foi relatado que os representantes do Startup Brasília muitas vezes atraiam palestrantes que eram referência em todo o Brasil e pagavam do próprio bolso para que víessem para atrair os Empreendedores, eram líderes natos do Ecossistema.

Mesmo sem um grande canal para investimentos na cidade um dos Empreendedores mencionou que foi uma época em que Brasília entrou no radar como um dos melhores Ecossistemas de Startups do Brasil. A presença do Governo também era forte, muitos Empreendedores citaram o constante apoio do SEBRAE DF como essencial para o crescimento do Ecossistema na época. A união entre o SEBRAE e o grupo Startup Brasília foi o principal ponto para que em 2012 Brasília tivesse a maior delegação Brasileira no Tech Crunch Disrupt, um grande marco do nosso Ecossistema. Nessa mesma época também houve uma edição do Startup Farm em Brasília, atraida por esses atores.

Infelizmente, segundo alguns Empreendedores, esse foi o auge da cidade. Chegou a ser discutida a possibilidade de ser criada uma Associação que melhor representasse o Ecossistema, principalmente perante ao Estado, mas o grupo optou por não seguir esse caminho e coincidentemente após esse momento alguns dos líderes do Ecossistema deixaram o país, alguns para serem acelerados no Vale do Silício, e a pessoa chave no Sebrae responsável pelas Startups também fora transferida para outra área. 

Esse é o ponto em que um dos Empreendedores classifica como o momento em que o Ecossistema do Distrito Federal começou a perder sua força, entre meados de 2012 e 2014. Também foi quando nasceu a Associação de Startups e Empreendedores Digitais, criada por outro grupo de Empreendedores, e naturalmente tomou a posição de liderança e referência do Ecossistema mas sem o apoio e a confiança desse antigo grupo.

Um dos Empreendedores menciona que entre 2015 e 2016, o Ecossistema de Startups de Brasília voltou a reagir. Ele não sabe dizer o que encadeou o movimento, mas diz que o cenário definitivamente não é mais o mesmo e voltou a crescer. Com o reencontro da maior parte dos membros do ``Startup Brasília'', a presença de dois grandes atores de investimento trazendo cerca de R\$ 100 milhões para Startups e três programas de aceleração se estabelecendo esse Empreendedor acredita que o Ecossistema de Brasília logo voltará a ser o que era se os atores se unirem e formarem um grupo forte novamente.

\section{Perguntas de Pesquisa}
\label{section:perguntas_de_pesquisa}

\subsection{Pergunta de Pesquisa 1: Quais são as características socioculturais de Brasília que promovem ou inibem o espirito empreendedor?}
\label{subsection:pergunta_de_pesquisa_1}

A Endeavor cita a Cultura Empreendedora de Brasília como a pior do Brasil\citerefonline{indiceglobaldoempreendedorismo}, mas esse cenário claramente está mudando e iniciativas estão nascendo em Brasília. 

Alguns empreendedores relatam que em Brasília, atualmente, acontece no mínimo um evento relacionado a Empreendedorismo e Startups por semana como um fator muito positivo, mas muitos afirmam que a aversão ao risco do brasiliense e o desejo pelo funcionalismo público ainda são fatores preocupantes que inibem o Empreendedorismo.

\subsection{Pergunta de Pesquisa 2: Quais são os mecânismos institucionais de Brasília que promovem ou dificultam o Empreendedorismo?}
\label{subsection:pergunta_de_pesquisa_2}

  
\subsection{Pergunta de Pesquisa 3: Quais são os mecânismos educacionais de Brasília que promovem o Empreendedorismo?}
\label{subsection:pergunta_de_pesquisa_3}

A Endeavor cita a Cultura Empreendedora de Brasília como a pior do Brasil\citerefonline{indiceglobaldoempreendedorismo}, mas esse cenário claramente está mudando. Já é comum encontrar nichos de estudantes universitários frequentando os eventos relacionados à Startups da cidade e pequenos núcleos se formando, como a Liga Universitária Marco Zero.


\subsection{Pergunta de Pesquisa 4: Como os fatores tecnológicos influenciam o sucesso ou fracasso das Startups de Brasília? Qual o papel executado pela comunidade e pelo Software Livre?
\label{subsection:pergunta_de_pesquisa_4}


\subsection{Pergunta de Pesquisa 5: Qual a relação do empreendedor de Brasília com as opções de investimento disponíveis e como elas influenciam o Ecossistema?
\label{subsection:pergunta_de_pesquisa_5}


\subsection{Pergunta de Pesquisa 6: Quais ações devem ser tomadas no Ecossistema de Brasília para que ele cresça? 
\label{subsection:pergunta_de_pesquisa_6}


\subsection{Pergunta de Pesquisa 7: Qual a prospecção para o futuro dos Empreendedores de Brasília?}
\label{subsection:pergunta_de_pesquisa_7}

\section{Mapa do Ecossistema}
\label{section:mapa_do_ecossistema}

Será desenvolvido na segunda parte do TCC.