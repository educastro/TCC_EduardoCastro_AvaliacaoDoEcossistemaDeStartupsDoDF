\chapter[Resultados]{Resultados}
\label{cap-resultados}

Este trabalho começou a ser elaborado em meados de Janeiro e Fevereiro de 2016, tendo seu início marcado por extensos estudos por livros e publicações com relação aos temas de empreendedorismo ou ecossistemas/hubs de inovação, também foram realizadas 16 entrevistas com atores do ecossistema de startups do Distrito Federal com o objetivo de explorar o cenário local e responder as questões propostas.

\section{Respostas para as Perguntas}
\label{section:perguntas_de_pesquisa}

\subsection*{Pergunta 01: Quais são as características socioculturais do DF que promovem ou inibem o espirito empreendedor?}
\label{subsection:pergunta_de_pesquisa_1}

A \citeonline{indiceglobaldoempreendedorismo} cita a Cultura Empreendedora de Brasília como a pior do Brasil, mas há indícios de que esse cenário possa estar mudando quando observamos as iniciativas que estão nascendo em Brasília como, por exemplo, a participação de 5056 pessoas em grupos relacionados a encontros de tecnologia, empreendedorismo, inovação e/ou startups na rede social Meetup.com.

Diversos empreendedores relataram a forte cultura em torno da estabilidade, do salário disponível na conta ao quinto dia útil do mês e do emprego seguro no Distrito Federal, muitos acreditam ser por influência do funcionalismo público, dos benefícios garantidos pela leis trabalhistas que os regem e do que a presença da estrutura governamental representa. Um dos empreendedores relatou que não encontra com facilidade jovens no Distrito Federal com ambição para criar e crescer, com vontade de empreender e dedicar "horas e mais horas programando madrugada adentro", o mesmo relatou que mesmo hoje, após participar da criação de uma empresa de tecnologia que fatura aproximadamente R\$ 160 milhões por ano, o pai o enxerga como um fracassado e tem o irmão, servidor público, como símbolo de sucesso na família, percepções similares foram relatadas por vários outros entrevistados. Outro empreendedor descreveu essa percepção como "muitos preferem ser passageiros do que os criadores de um veículo" e que falta um maior senso de curiosidade, outro como "falta ritmo de trabalho" no brasiliense, esse fator foi um dos motivos pela qual um dos empreendedores definiu a experiência de contratar para startup no Distrito Federal como: "terror, pânico e aflição". 

A cultura em torno do funcionalismo público e do governo se mostra tão presente que um dos empreendedores relatou notar que nos grupos de discussão ou mesas de bar com pessoas envolvidas com o ecossistema de startups de Brasília é comum a agenda política entrar em discussão, algo que o mesmo relata não encontrar em seus círculos de Belo Horizonte, São Paulo ou Florianopólis.

Também foi falado sobre a falta de disposição do brasiliense em aceitar a possibilidade de dedicar anos para uma startup sem qualquer retorno financeiro, mesmo entre jovens universitários que poderiam usufruir da segurança financeira oferecida pelos pais, o baixo custo de vida e a ausência de maiores compromissos familiares e profissionais para empreender. Um empreendedor mencionou que o baixo número de pessoas dispostas a executar e a correr riscos com o empreendedorismo é um problema presente em todo o país, mas no Distrito Federal relatou ter a impressão de ser mais forte. Um dos empreendedores relatou que o "embate" das startups para obter os melhores profissionais em Florianopólis é com outras empresas, em São Paulo com multinacionais e em Brasília com o Estado, que é para onde o mesmo relata que vão a maior parte dos melhores profissionais.

Foi identificada também que essa visão de "funcionário" está presente até mesmo na forma como as pessoas se relacionam na cidade, segundo o mesmo de uma forma "hierarquizada" por não haver uma mentalidade de abundância ou colaboração entre os empreendedores da cidade, o que os afastam como grupo social e criam barreiras invisíveis e fortalecem os grupos fechados. Esse afastamento também reflete de outras formas na dinâmica do ecossistema do Distrito Federal, por haver baixa interação entre os atores é comum que as diversas ações de fomento ao ecossistema não conversem entre si, muitas vezes gerando até mesmo eventos co-relacionados competindo pelo mesmo público no mesmo dia/horário, ou que haja afastamento com as grandes empresas de tecnologia da região.

Essa proximidade e baixa barreira para interação entre atores foi relatada por um dos empreendedores como um dos pontos fortes de ecossistemas como o de Belo Horizonte, conhecido como "San Pedro Valley", ou em Florianópolis. Foi relatado que nesses ecossistemas é comum que os empreendedores se encontrem com frequência, organizem "happy hours" e façam reuniões entre si com o fim de gerar colaboração como ecossistema. Um dos empreendedores do Distrito Federal queixou-se da dificuldade de se reunir com outros e o quão desmotivador isso pode ser durante os dias difíceis da vida de empresário. Um empreendedor descreveu que um dos motivos para esse distanciamento é a própria arquitetura e dinâmica de mobilidade urbana da cidade, o mesmo relatou que enquanto em Belo Horizonte é comum encontrar vários empreendedores almoçando na mesma região ou saindo para tomar um café na mesma padaria do bairro em Brasília as pessoas tendem a se distanciar.

Um empreendedor relatou que as startups locais ainda são muito imaturas e ser comum encontrar profissionais com altos salários, sendo muitos servidores públicos, empreendendo durante o seu tempo livre, mas o mesmo acredita que dedicação integral é um fator crucial para o sucesso de qualquer negócio. O mesmo empreendedor relatou a impressão de que falta coragem no brasiliense para empreender ``de verdade''. O mesmo entrevistado relatou ser clara a diferença entre uma startup do Distrito Federal com 6 meses de vida e uma startup de São Paulo, segundo ele as nossas mal sabem validar o produto, não aplicam as metodologias corretamente e, talvez pela natureza da cidade, estão acostumadas a sobreviver por meio de editais públicos de investimento.

Uma empreendedora acredita que esses fatores negativos são naturais e parte do desenvolvimento que Brasília ainda vai enfrentar. Segundo ela o próprio termo empreender é recente na cidade e até pouco tempo atrás essa não era uma realidade, aos poucos as mudanças culturais de qual necessitamos acontecem e as peças do ecossistema vão se complementando.

Quanto as grandes empresas de tecnologia, embora tenham interesse em se aproximar das startups, não o fazem por não haver um ecossistema estruturado ou canais para que a interação aconteça, foi dito também que as mesmas entendem a importância do cenário de startups a níveis globais mas não compreendem como aproximar essas inovações do seu modelo de negócios. 

Encabeçando a mudança de cultura em Brasília alguns movimentos como o Startupeiro, o Cerrado Valley, o Garagem Dextra, universidades, aceleradoras e a Associação de Startups e Empreendedores Digitais (ASTEPS) tentam promover encontros com o objetivo de conectar o ecossistema mas a questão dos grupos que não conversam entre si ainda se mostram uma grande barreira para um ambiente de colaboração sádio no Distrito Federal. Outro fator que também contribui para a geração de conexões entre atores é o crescimento dos espaços de coworking da capital, foram mapeados sete que atuam de forma proativa junto ao ecossistema local e há mais quatro que devem ser lançados até o fim de 2017, atualmente há eventos que conversam com a pauta do empreendedorismo e startups quase que semanalmente e isso é visto como um fator muito positivo. 

Outro fator chave relatado por vários dos empreendedores como um momento essencial para a sua formação empreendedora fora a participação no Movimento Empresa Junior enquanto universitários não apenas pela experiência profissional como também pelas redes de contato formadas no período e toda a "energia empreendedora" presente neste circulo estudantil. 

Do ponto de vista de mercado, um dos empreendedores relatou que Brasília, por ser uma cidade muito conectada, com boa infraestrutura de tecnologia e população com alto poder aquisitivo, se mostra excelente para estratégias com foco na aquisição de usuários finais ("Go-to-Market"). Segundo ele, além de muitos usuários de planos pós-pago com boas conexões de internet móvel há muitas pessoas dispostas a serem "early adopters", como são chamados os primeiros usuários a adotarem uma nova tecnologia.

\subsection*{Pergunta 02: Quais são os mecânismos institucionais de Brasília que promovem ou dificultam o Empreendedorismo?}
\label{subsection:pergunta_de_pesquisa_2}

Fazendo um contraponto aos impactos apontados como negativos quanto a forte influência do governo no Distrito Federal o mesmo também é o maior comprador de tecnologia do país, apenas o governo federal, em 2015, gastou cerca de R\$ 5 bilhões em equipamentos e serviços de TI. A presença de tamanho orçamento também faz com que muitas das grandes empresas de tecnologia do país tenham escritórios no Distrito Federal, além de grandes brancos, o próprio Banco Central, etc. Essa proximidade natural com um grande comprador foi apontada como um potencial atrativo e uma vantagem para as startups do DF, podendo, inclusive, o modelo de negócios "business-to-government" vir a se tornar uma das vocações da região.

Do ponto de vista regulatório a proximidade também é positiva, tanto pela possibilidade de construção de políticas públicas como também pela facilidade de simplificação, visto que as empresas do Distrito Federal respondem a apenas duas esferas governamentais que fisicamente estão muito próximas: o  distrital e o federal, essa proximidade fez com que o Distrito Federal fosse o primeiro "Estado" brasileiro a testar um novo processo para abertura de empresas de baixo risco 100\% digital, dispensando até mesmo que o empresário compareça a junta comercial para assinar documentos. Esse novo processo fez com que fosse possível abrir uma empresa em menos de 7 dias, um recorde no país.

Em uma postagem institucional da Dínamo, um movimento formado por empreendedores do setor brasileiro de startups com apoio de grandes organizações como o Google, a aceleradora Startup Farm e as entidades de apoio Anjos do Brasil, Associação Brasileira de Privaty Equity \& Venture Capital e AB Startups, \citeonline{Pacheco2017} descreve o papel de "preparar o terreno" de um determinado ecossistema para que os demais atores que o compõem (investidores, aceleradoras, incubadoras, corporações, universidades e empreendedores) possam atuar como uma das responsabilidades do governo, o que incluem ações de fomento para atração de investidores, redução da burocracia e destinação de espaços públicos ociosos para ações de fomento e apoio ao empreendedorismo.

Nessa linha o governo local tem como principal política pública para apoio ao ecossistema de startups do Distrito Federal o programa o Startups Brasília, um edital de subvenção econômica para investimento de até R\$ 150.000,00 para startups em estágio nascente. O programa já investiu um total de R\$ 13 milhões em cerca de 80 startups e, embora algumas demonstrem grande potencial de crescimento alguns empreendedores definiram o referido edital como "mal estruturado" e "perigoso" para os empreendedores por conta da alta burocracia e complexidade ligada ao dinheiro. O Governo de Brasília também foi responsável pela execução de uma edição do evento de tecnologia e empreendedorismo Campus Party e reuniu cerca de 70 mil pessoas para palestras, workshops, feira de negócios, rodadas de mentorias, exposição de startups locais, etc. Ambos os projetos também tiveram apoio e participação direta da Associação de Startups e Empreendedores Digitais (ASTEPS), embora a associação tenha sido alvo de críticas entre os empreendedores entrevistados.

Um ponto interessante que fora observado no que tange governo refere-se ao fato de que os grandes programas de apoio e aceleração à startups do governo federal, como o Startup Brasil e o Inovativa Brasil, serem executados por equipes alocadas Distrito Federal, ainda assim nota-se que a presença de projetos oriundos do DF não são significativas, das 549 startups aceleradas pelo Inovativa, por exemplo, apenas 20 eram da capital, pouco mais de 3\%. Ao questionar o dado com um dos responsáveis pelo programa um ponto chave relatado na divulgação do programa foi a atuação do SEBRAE local de cada região, um dos empreendedores também reforçou o potencial do SEBRAE como um dos atores mais importantes e atuantes no ecossistema de startups de Florianopólis.

No Distrito Federal o SEBRAE foi apontado por alguns entrevistados como um dos principais apoiadores durante a boa fase de crescimento do ecossistema de startups de Brasília em meados de 2010 a 2012 mas também foi relatado que atualmente não possui atuação tão significativa, embora esteja presente e ofereça, além de suporte para os empreendedores, uma série de atividades para o público interessado em startups em um espaço chamado SEBRAE Lab, uma espécie de "coworking público" que é utilizado para programas de aceleração, eventos, palestras, treinamentos, etc. 

Outro fator mencionado como positivo no DF foi a presença de grandes organismos internacionais e fundações com grande poder de compra e/ou investimento em startups que apresentem soluções de seu interesse, um dos motivos pelos quais um dos empreendedores acredita que as melhores oportunidades para os empreendedores do Distrito Federal estão concentradas nos negócios e soluções de impacto social. Ele também acredita que o Brasil é um país com muitos desafios sociais latentes ao mesmo tempo em que possui uma população bem conectada, o que cria um cenário favorável para startups que queiram explorar essas oportunidades.

Uma crítica apontada por vários empreendedores refere-se a falta de lideranças atuantes no ecossistema. No passado a atuação e união dos que eram reconhecidos como lideranças da época era bastante presente mas, por diversas situações que não convém serem exploradas neste trabalho, houveram rupturas entre os empreendedores do Distrito Federal entre 2012 e 2013, até o momento deste trabalho, em 2017, esses conflitos e separações no ecossistema de startups do Distrito Federal ainda são claros. 

Segundo relatos dos empreendedores o ecossistema de startups do Distrito Federal voltou a ganhar destaque e a se recuperar da ruptura em meados de 2015 e 2016, mesmo período em que novos atores, como aceleradoras, investidores e espaços de coworking, começaram a se estabelecer na cidade. Um dos empreendedores relatou que o afastamento se deu por notarem que a melhor forma de construírem um bom ecossistema era com empresas de sucesso e não eventos, outros empreendedores mencionaram os conflitos como um dos pivôs para o seu afastamento. De qualquer forma também é notável que aqueles que no passado deram início ao ecossistema de startups do Distrito Federal se encontram em outros momentos pessoais, familiares e profissionais que não mais o anterior, motivos pelos quais é compreensível os afastamentos, de qualquer forma também é notável que foram postos que não foram tomados por uma nova geração de empreendedores.

Ao todo, no DF, estão presentes quatro aceleradoras, cerca de vinte espaços de coworkings, duas pré-aceleradoras, dois fundos de capital semente ligados ao fundo Criatec, um fundo de venture capital, uma representação de investidores anjos e diversas representações de empresários, sendo uma delas focada em startups e outras duas de jovens empresários que também conversam com o mercado de startups. O Distrito Federal possui, também, quatro incubadoras, dois parques tecnológicos, mesmo que ainda bem tímidos, e a presença, recém-chegada, da Endeavor que trouxe uma profissional com o objetivo de identificar e disseminar boas práticas para a construção de um melhor ambiente regulatório no DF e no Brasil, um trabalho similar ao da organização Dínamo, mencionada anteriormente, que também frequenta a capital mas com o objetivo de colaborar no âmbito federal.

Para fins de comparação, um estudo sobre aceleradoras de startups no Brasil realizado por \citeonline{abreu2016panorama} de outubro de 2015 a janeiro de 2016 localizou 45 aceleradoras no Brasil mas nenhuma no centro-oeste, durante esta pesquisa foram identificadas cinco no Distrito Federal e uma em Goiânia, sendo esta uma das principais do país, a paulista ACE.

Diversos atores também criticaram as incubadoras do Distrito Federal pelas mesmas não oferecerem condições interessantes e para o processo de incubação e pequena oportunidade de crescimento, um dos entrevistados relatou insatisfação com o baixo número de conexão com investidores e representantes de grandes empresas que o Centro de Desenvolvimento Tecnológico, responsável pelo programa de incubação da Universidade de Brasília, proporcionava.

Atualmente três canais de comunicação (dois jornais e um impresso) do Distrito Federal criaram colunas com o fim especifico de promover o empreendedorismo no Distrito Federal, espaços onde as startups locais figuram como protagonistas com frequência.

\subsection*{Pergunta 03: Quais são os mecânismos educacionais de Brasília que promovem o Empreendedorismo?}
\label{subsection:pergunta_de_pesquisa_3}

Universidades como a Universidade de Brasília (UnB), o UniCEUB, o IESB e a UDF foram mencionadas como as  interessadas em criar conexões com as startups, o que são fatores positivos. Em especial nas particulares já existem uma série de ações pontuais como programas de pré-aceleração, aceleração, "maker spaces" e "coworkings" para os estudantes empreendedores, bem como reformulações das disciplinas de empreendedorismo e a inserção da pauta "startups" nas salas de aulas. Na UnB, como instituição, o movimento ainda se mostra incipiente mas foram mencionadas diversas iniciativas executadas por estudantes, como o Movimento Empresa Júnior, a Liga Universitária Marco Zero, o Núcleo da Fundação Estudar e o Garagem Dextra responsáveis por organizar atividades, palestras e capacitações com a pauta empreendedorismo. 

Um dos entrevistados descreveu sua experiência na empresa junior como uma "escola de empreendedorismo", reforçou sua importância para o seu desenvolvimento como empreendedor e críticou a falta de disciplinas que desenvolvam competências empreendedoras e incentivem os alunos a criar produtos reais. O mesmo relata que uma política de fomento ao empreendedorismo contínua e frequente na universidade pode ser importante para que ela se aproxime do mercado. 

Vários outros empreendedores reforçaram a baixa contribuição das universidades em suas formações como empreendedores, alguns afirmaram que seu papel de formação na área fora zero e que não visualizam nenhuma universidade preparada para formar empreendedores. No estudo sobre São Paulo, \citeonline{MonnaSantos2015} identificou que um direcionamento das universidades para o mercado de trabalho, no Distrito Federal alguns empreendedores destacaram uma atenção à academia e ao universo científico, ao mesmo tempo fora críticada o baixo número de Institutos de Pesquisa de CT\&I no DF. Um professor de uma das universidades privadas relatou que a atualização de currículos para uma maior aproximação ao empreendedorismo ou as demandas do mercado de trabalho é uma dificuldade que todas as instituições possuem devido as exigências do MEC e a pressão das instituições em preparar os alunos para que alcancem bons indicadores em provas como o Exame Nacional de Desempenho de Estudantes (ENADE). 

A principal fonte de aprendizado relatada por estes empreendedores envolveram o auto-didatismo por meio de livros e conteúdos online (podcasts, sites, cursos online, etc.) e alguns criticaram a dificuldade de se encontrar conteúdo de qualidade em português, visto que a maior parte das boas referências eram publicadas na língua inglesa. Assim como em São Paulo esses aprendizados foram levantados como ponto que merecem cuidado quando aplicados por empreendedores do Distrito Federal, visto que ecossistemas possuem suas particulares e diferentes níveis de maturidade. Muitos relataram que o maior valor da universidade deve-se as oportunidades de conexões daquele ambiente, principalmente entre professores e alunos.

Também fora identificado um receio por parte de alguns empreendedores em relação aos conteúdos ou atores que tem recebido destaque no campo de ensino ao empreendedorismo no Brasil e no Distrito Federal por não serem vistos como boas referências ou sequer como empreendedores, algo que os mesmos identificaram como "empreendedores de palco".

Um dos empreendedores acredita que uma boa universidade é capaz de se tornar uma geradora de grandes startups visto que é durante a graduação que os estudantes possuem um baixo custo de oportunidade e são capazes de lidar o risco e a incerteza que empreender proporcionam sem maiores danos em suas vidas pessoais e que a mesma deve ter uma conexão forte com empreendedores para que os mesmos atuem como mentores e como referências e vistos como exemplos de que é possível seguir um caminho profissional com o empreendedorismo.

\subsection*{Pergunta 04: Como os fatores tecnológicos influenciam o sucesso ou fracasso das Startups de Brasília? Qual o papel executado pela comunidade e pelo Software Livre?}
\label{subsection:pergunta_de_pesquisa_4}

Um dos empreendedores relatou que considera um grande erro utilizar tecnologias modernas e amplamente utilizadas por outros ecossistemas de startups como Ruby on Rails, Python, Swift, etc pela falta de profissionais capacitados e acessíveis. Ele relata que por diversas vezes manteve uma vaga aberta por meses por não encontrar um profissional adequado, o que para uma startup é inviável e possivelmente o motivo de sua falência. A startup precisa crescer rápido, e o mercado precisa ser capaz de suprir suas necessidades de escala ainda mais rápido. Em sua Startup atual ele optou por utilizar as linguagens Java e PHP, e por ter como uma grande base do seu negócio o aprendizado de máquina e diversas técnicas de inteligência de dados seu principal foco tem sido atrair profissionais insatisfeitos de bancos, que segundo o empreendedor estão em abundância no mercado e são mais baratos do que bons desenvolvedores de linguagens mais modernas. 

O mesmo empreendedor fez o seguinte comentário: "Se a sua startup conseguir tracionar e captar investimento e surgir uma demanda de 50 novos funcionários para trabalharem com Node.js em Brasília o que você faz? Não devem ter 50 programadores ativos nas comunidades de Node do DF, dirá disponíveis no mercado. Ou você vai morrer ou vai precisar gastar uma fortuna trazendo gente de fora. Com Java e PHP eu tenho a segurança de um mercado com profissionais em abundância disponíveis".

Outro empreendedor relata que um dos pontos decisivos antes de trocarem Brasília por São Paulo após captarem investimento, em meados de 2014, fora a dificuldade de contratar desenvolvedores aptos a trabalhar com Ruby on Rails na capital. Como mencionado na QP \ref{subsection:pergunta_de_pesquisa_1} a experiência de contratar em Brasília para um dos empreendedores foi descrita como "terror, pânico e aflição", também por essa dificuldade.

Todos os entrevistados relataram ou demonstraram utilizar metodologias de gestão ágeis em seus projetos ou ao menos algumas das boas práticas, um dos empreendedores diz ser uma prática essencial pelo tamanho enxuto da equipe e pela alta pressão por entregar um produto de alta qualidade, validado e testado em tempo hábil para que a empresa possa faturar, investir e expandir. O uso de softwares livres e soluções disponíveis na internet, em especial linguagens de programação e frameworks, também se mostraram um padrão entre os entrevistados, mas preocupações com testes automatizados, programação em pares e demais práticas que visam aumentar a qualidade do código desenvolvido foram identificadas apenas em entrevistados com perfis técnicos.

\subsection*{Pergunta 05: Qual a relação do empreendedor de Brasília com as opções de investimento disponíveis e como elas influenciam o Ecossistema?}
\label{subsection:pergunta_de_pesquisa_5}

Como relatado por um dos empreendedores em meados de 2012 não havia capital disponível de forma ampla para startups no Distrito Federal além do proveniente de investidores anjos, diferente dos tempos atuais que existem aproximadamente R\$ 100 milhões destinados para investimento no setor na mão de atores privados, como o fundo de venture capital Cedro Capital, os gestores regionais do fundo Criatec Garan Ventures e Blackhold, alguns investidores anjos e os fundos governamentais. 

A presença de aceleradoras também foi bastante mencionada, principalmente a Acceleratus, Lemonade e Cotidiano, mas também fora críticado o fato de que a maior parte do investimento anunciado se dava na forma de serviços. Um dos empreendedores também mencionou que em sua maior parte elas estão em busca de startups que busquem inovar nos mercados tradicionais, em fazer transferência de tecnologia e levar inovação para grandes empresas. Alguns relataram alta dificuldade de obter investimento e outros que a dificuldade no Distrito Federal é a mesma de outros ecossistema: requer perseverança, muitos emails e muitos pitches até encontrar o ideal.

Ao entrevistar dois dos atores vistos como referência no cenário de investimentos do Distrito Federal ambos relataram a alta dificuldade de se encontrar startups capazes de receber aporte devido a baixa maturidade e baixo preparo do empreendedor, os mesmos relatam que o número de projetos é vasto mas o número de projetos de qualidade muito baixo. Um deles relatou que o corte no DF é tão baixo que uma startup que fatura R\$ 100 mil por ano chama atenção de seu fundo, mesmo sendo um valor considerado muito baixo para outros ecossistemas. 

Um dos empreendedores mencionou que existe um representante da Anjos do Brasil no Ecossistema, mas que o mesmo passou anos sem demonstrar interesse em investir na capital até que recentemente começou a participar de alguns dos eventos organizados por aceleradoras. Pelo mesmo motivo da falta de maturidade e dedicação integral dos Empreendedores mencionado na subseção \ref{subsection:pergunta_de_pesquisa_1} um empreendedor menciona que existe capital em abundância no Distrito Federal, mas que muitos investidores preferem procurar startups em cidades como Florianopólis, Belo Horizonte, Recife ou São Paulo, onde os empreendedores lidam melhor com o risco e com a falta de segurança e estabilidade.Também foi identificada uma falta de preparo para apresentações de pitches alinhados ao que investidores buscam (indicadores de crescimento, faturamento, etc.), reforçando a visão transmitida pelos investidores. Essa baixa maturidade, segundo alguns empreendedores, já foi motivo para o não estabelecimento de grandes fundos e aceleradoras na capital.

Uma ação importante que as aceleradoras tem feito com os investidores envolve um papel de "capacitação" de grandes empreendedores da área de tecnologia do DF sobre o que é uma startup, qual a dinâmica deste mercado, e como e porque investir como anjos ou sócios de programas de aceleração. O fato de que os fundadores das aceleradoras do Distrito Federal são grandes empresários do setor contribue para a transmissão de confiança para esses possíveis novos atores. 

Fora mencionado diversas vezes a alta densidade de pessoas no Distrito Federal com alto poder aquisitivo e capacidade de atuar com investidores anjos de startups, ainda há pouco conhecimento acerca de como funciona, do risco envolvido e do potencial ganho. Com o advento da Lei Complementar número 15, de 27 de outubro de 2017, e regulamentação do investimento anjo no Brasil, juntamente com a atuação da mídia e conexão entre atores acredita-se que esse cenário possa crescer como uma opção de investimento mais atraente do que as opções conservadoras.

Em meados de junho deste ano fora anunciado um aporte "Serie A" na startup "Configr", o valor oficial não foi divulgado mas este tipo de investimento costuma ser acima de US\$ 2 milhões. Alguns empreendedores mencionaram ser o maior investimento no DF desde a ZeroPaper, vendida para a empresa da California Intuit em 2015. Um dos empreendedores descreve o atual momento do Distrito Federal como crítico pelo tamanho do investimento realizado, o mesmo acredita que a depender dos resultados obtidos pela "Configr" poderemos afastar ou atrair novos fundos.

\subsection*{Pergunta 06: Quais ações devem ser tomadas para que o ecossistema cresça?}
\label{subsection:pergunta_de_pesquisa_6}

A maior dificuldade e ponto de melhora mencionada envolve a falta de uma liderança que una os atores e cesse o constante ``conflito de egos'' no ecossistema, muitos relataram que esse é um dos fatores de maior limitação do DF. Um dos empreendedores do que pode ser visto como a "primeira geração de startup" relacionou a queda do ecossistema entre 2013 e 2015 com a separação do antigo grupo do ``Startup Brasília'' mas acredita que o crescimento será impulsionado nos próximos anos com a volta desse grupo e com o apoio de novos atores, como investidores. Ele observa que os antigos atores estão quase todos envolvidos em pelo menos uma das iniciativas de aceleração disponíveis na cidade, e que dessa forma eles voltarão a causar impacto constante e a explorar sua rede de contatos para trazer iniciativas de fomento para Brasília.  

Muito foi falado sobre a baixa cultura de colaboração e troca de experiências no Distrito Federal, o que também fora apontado como outro ponto de melhora. Um dos empreendedores mencionou que não temos uma comunidade, provavelmente fruto desses conflitos. De acordo com os relatos a figura de uma forte liderança e de um ecossistema unido e integrado era bem clara e presente entre meados de 2010 e 2012, diversos empreendedores citaram um grupo chamado ``Startup Brasília'' composto por empreendedores de empresas como a Intacto, Qual Canal, SEA Tecnologia, Rota dos Concursos, IPê Tecnologia, Trip2gether, etc. Muitos dos participantes desse grupo e do ecossistema em 2012 estão representados na Figura \ref{figure:startups_board_2012}, criada pelo empreendedor Marcos Oliveira. Esse grupo foi responsável por organizar encontros, startup weekends, um ciclo de capacitação chamado startup dojo, etc.

\begin{figure}[!htb]
	\centering
	\includegraphics[width=15cm,angle=0]{figuras/startups_board_2012}
	\caption{Representantes do Ecossistema de Startups de Brasília em 2012}
	\label{figure:startups_board_2012}
\end{figure} 

A união era tamanha que mesmo sem um grande canal para investimentos na cidade um dos empreendedores mencionou que foi uma época em que Brasília entrou no radar como um dos melhores ecossistemas de startups do Brasil. Muitos empreendedores citaram o constante apoio do SEBRAE DF como essencial para o crescimento na época, esse apoio foi essencial para que, em 2012, Brasília tivesse a maior delegação brasileira no Tech Crunch Disrupt\footciteref{Correio2012}, momento que ainda é mencionado por muitos como um grande marco do ecossistema. Nessa mesma época, por incentivo de alguns do grupo, houve uma rodada de aceleração da Startup Farm\footciteref{StartupFarm} em Brasília, uma das maiores aceleradoras de startups da América Latina. Sem dúvidas são pontos que demonstram o poder que uma liderança unida e coesa pode ter em um ecossistema de startups. 

Infelizmente, segundo alguns empreendedores, esse foi o auge. Quando foi levantada a possibilidade de ser criada uma associação que melhor representasse o ecossistema, principalmente perante ao Estado, o grupo aparentava já estar "quebrado" e coincidentemente após esse momento alguns dos líderes do ecossistema deixaram o país para se dedicarem as suas startups no Vale do Silício, também foi quando a pessoa chave no Sebrae responsável pela área de startups também logo fora transferida para outro setor da instituição.

Um empreendedor mencionou que faltam incentivos fiscais e que o tema precisa entrar com mais força na agenda pública. O mesmo acredita que o DF é muito carente de bons eventos sobre empreendedorismo, que há uma forte demanda pelo tema e que há baixa interação entre os atores que organizam tais eventos por ser comum de termos vários acontecendo no mesmo horário e dia. 

Outro empreendedor mencionou a necessidade de se criar um conselho para discussões mensais sobre startups criado pelo Governo para que ações com o objetivo de flexibilidar e apoiar iniciativas empreendedoras sejam tomadas, bem como desenvolver políticas públicas que apoiem o setor. Um grupo formado pela Associação de Startups e Empreendedores Digitais (ASTEPS) e pelo SEBRAE DF tenta unir governo local, universidades, empreendedores, aceleradoras e donos de coworkings com o objetivo de fortalecer as ações de fomento ao empreendedorismo, talvez esse grupo possa contribuir para a melhora desses  dois últimos pontos.

Alguns empreendedores acreditam que Brasília está no seu melhor momento mas falta "orquestração", também foi mencionado a importância de se criar espaços que possam agir como "hubs" de conexão entre empreendedores, startups, grandes empresas, investidores, aceleradoras, governo, universidades e estudantes. Um dos empreendedores entrevistados acredita que apenas depois de todos os atores estarem alinhados e colaborando será possível ter um ecossistema sólido, frutífero e que faça sentido, ele acredita que ainda falta um entendimento do que é um ecossistema, de quais os papéis de cada ator e que esta "orquestração" deve ser orgânica e horizontal, e não apenas um empreendedor que decide tomar o controle da situação. 

Um dos entrevistados, de uma aceleradora, visualiza um ecossistema de startups como uma esteira em que as universidades e institutos de pesquisa capacitam os estudantes para criarem ou trabalharem em empresas e criam inovações, governo cria um ambiente adequado para o seu crescimento, investidores aportam capital, aceleradoras impulsionam as startups, grandes empresas incorporam essas inovações, etc. Para as aceleradoras do Distrito Federal esse funcionamento se mostra bem claro mas para as universidades não. Um exemplo claro mencionado pelo empreendedor é que as universidades poderiam preparar mais os estudantes para lidar com clientes e obtê-los do quepara criar soluções e produtos.

A cultura empreendedora, apontada como a pior do Brasil pela \citeonline{indicedecidadesempreendedores2016} também é um grande ponto de melhora. A diminuição dos concursos públicos foi apontada por alguns empreendedores como um dos fatores que devem impulsionar a busca pelo empreendedorismo mas acredita-se que mais deve ser feito nas universidades do DF como criação de disciplinas que incentivem os alunos a criar startups, maior contato com empreendedores em sala de aula e/ou como mentores, etc.

\begin{table}
\centering
\begin{tabular}{ | c | c | c | c | c | c |}
\hline
\thead{Universidades} & \thead{Aceleradoras} & \thead{Espaços\\de\\Coworking} & \thead{Organizações\\de\\Apoio} & \thead{Mídia} & \thead{Investidores} \\
\hline
\makecell{UnB} & \makecell{Accelerattus} & \makecell{55lab} & \makecell{ASTEPS} & \makecell{Metropoles} & \makecell{Garan\\Ventures} \\
\hline
\makecell{UniCEUB} & \makecell{Cotidiano} & \makecell{Multi-\\plicidade} & \makecell{Garagem\\DEXTRA} & \makecell{Jornal\\de\\Brasília} & \makecell{Cedro Capital} \\
\hline
\makecell{IESB} & \makecell{Techmall} & \makecell{Co-\\piloto} &  \makecell{Startupeiro} & \makecell{Bizmeet} & \makecell{Polaris\\Investimentos} \\
\hline
\makecell{UDF} & \makecell{UniCEUB\\(Impulso\\e\\InovaSabin)} & \makecell{Manifesto} & \makecell{Cerrado\\Valley} & \makecell{Correio\\Braziliense} & \makecell{Blockhold} \\
\hline
\makecell{UCB} & \makecell{Runpal} & \makecell{The\\Brain} & \makecell{Movimento\\Empresa\\Junior} & \makecell{} & \makecell{FAP\\DF} \\
\hline
\makecell{CTJ} & \makecell{} & \makecell{W3\\Work} & \makecell{Impact\\Hub} & \makecell{} & \makecell{} \\
\hline
\makecell{} & \makecell{} & \makecell{Nós} & \makecell{Acelere\\me} & \makecell{} & \makecell{} \\
\hline
\makecell{} & \makecell{} & \makecell{} & \makecell{Fundação\\Estudar} & \makecell{} & \makecell{} \\
\hline
\makecell{} & \makecell{} & \makecell{} & \makecell{Endeavor} & \makecell{} & \makecell{} \\
\hline
\end{tabular}

\caption{Organizações que trabalham a cultura empreendedora no DF}
\label{table:metricas_de_classificacao_dos_fatores}
\end{table}

\section{Nível de maturidade e mapa do ecossistema de startups do DF}
\label{mapa_do_ecossistema_do_distrito_federal}

Como explicitado na subseção \cite{subsection:adaptacoes_para_o_trabalho} para essa pesquisa será utilizada uma adaptação da versão enxuta da metodologia de avaliação de ecossistemas de startups proposta por \citeonline{Cukier2016}, ao todo os nove indicadores seguintes foram analisados com base em dados disponíveis sobre o Distrito Federal e nos aprendizados com as entrevistas:

\begin{table}[H]
\centering
\begin{tabular}{ | c | c | c |}
\hline
\thead{Fator} & \thead{Valor} & \thead{Classificação}\\
\hline
\makecell{Estratégias\\de saída}&\makecell{poucas}&\makecell{Crescente}\\
\hline
\makecell{Investimento Anjo}&\makecell{irrelevante}&\makecell{Crescente}\\
\hline
\makecell{Gerações do\\ecossistema}&\makecell{2}&\makecell{Maduro}\\
\hline
\makecell{Eventos}&\makecell{semanais}&\makecell{Crescente}\\
\hline
\makecell{Atores da mídia\\com foco no\\empreendedorismo}&\makecell{3}&\makecell{Crescente}\\
\hline
\makecell{Dados do ecossistema\\e pesquisas}&\makecell{parciais}&\makecell{Maduro}\\
\hline \hline
\makecell{Incubadoras \\e\\Parques Tecnológicos}&\makecell{5}&\makecell{Maduro}\\
\hline
\makecell{Ambiente regulatório}&\makecell{7.4}&\makecell{Maduro}\\
\hline
\makecell{Cultura Empreendedora}&\makecell{3.6}&\makecell{Nascente}\\
\hline
\end{tabular}

\caption{Resultado da análise de indicadores do Distrito Federal}
\label{table:resultado_da_analise_de_indicadores_do_DF}
\end{table}

Em sua breve história o ecossistema de startups do Distrito Federal demonstrou poucas estratégias de saída, tendo essas se concentrado em algumas poucas aquisições e fusões de maior destaque, como o caso das startups ZeroPaper, Apetitar e Urbanizo. Não foram encontrados relatos de startups do ecossistema estudado que tenham aberto oferta em bolsas de valores (IPO). Por esse motivo, no indicador Estratégias de saída, fora definida a classificação crescente. 

Há de ressaltar, também, a entrada de aceleradoras e investidores privados no mercado local, ambos em busca de negócios promissores e capazes de prover lucros superiores ao de investimentos convencionais. Embora a entrada desses atores responsáveis por investimentos mereça destaque foi identificado, tanto pela visão dos investidores quanto pelos empreendedores, um cenário pouco maduro para esse tipo de investimento no Distrito Federal. Conforme essa pauta é desenvolvida dentro do ecossistema, curso com investidores de outros ecossistemas realizados e os conhecidos investidores do Distrito Federal vão a público relatar o que esperam encontrar em uma startup há a tendência de melhora nesse cenário, portanto este indicador fora definido, também, como crescente.

Parte dos atores que influenciam o cenário de startups do Distrito Federal também podem ser divididos entre gerações. Um dos empreendedores relatou a empresa Mirante Tecnologia como um dos representantes do que pode ser visto como a primeira geração de startups do DF (lançados em 1998), outros apontam que a primeira geração de startups foi formada por empreendedores em meados de 2008 a 2010. De qualquer forma, o cenário começou a se organizar como um "ecossistema de startups" com este segundo grupo de empreendedores, utilizando pela alcunha "Startup Brasília", e por isso fora considerado que o Distrito Federal possui duas gerações de empreendedores em torno do ecossistema.

Esses que foram considerados como os representantes do que pode ser dito como membros da primeira geração do ecossistema também foram responsáveis por diversos eventos, meetups, palestras, startup weekends, etc. Atualmente muitos desses eventos são organizados por novos atores, em especial pelas aceleradoras. Por meio de um levantamento no site meetup.com é possível identificar que há uma frequência semanal de eventos em torno dos temas empreendedorismo, tecnologia e inovação, o que pode vir a ser bastante frutífero para o ecossistema. Portanto, sua classificação foi crescente.

Impulsionados por um ecossistema que está começando a demonstrar destaque a mídia local vem dando destaque ao empreendedorismo, influentes atores locais como o site Metropóles e o jornal de Brasília possuem editoriais e jornalistas especializados em cobrir o tema. Alguns blogs, como o Bizmeet, também fazem a cobertura na capital. De forma similar, dados e pesquisas sobre o ecossistema local também começam a surgir. Por hora não há nenhum trabalho sistemático ou preciso capaz de dizer qual o faturamento ou o tamanho do setor no Distrito Federal, por exemplo, mas instituições como a Associação de Startups e Empreendedores Digitais (ASTEPS) e a Endeavor fazem o acompanhamento do cenário local. Recentemente a Associação Brasileira de Startups (AB Startups) também realizou uma pesquisa a nível nacional e apontou dados bem interessantes sobre o Distrito Federal como, por exemplo, possuir a segunda maior densidade de startups do Brasil. Embora tenhamos dados parciais sobre o ecossistema como um todo ainda faltam pesquisas que explorem os detalhes do ecossistema a um maior nível de detalhe como, por exemplo, quem são os empreendedores locais, quais são as startups atuantes no ecossistema, qual seu faturamento, mercado, número de funcionários, etc.

Em relação as Incubadoras e Parques Tecnológicos o Distrito Federal possui cinco: quatro incubadoras de faculdades e universidades locais (Uniceub, Universidade Católica de Brasília, IESB e UnB, tendo esta última, também, um pequeno parque tecnológico) e o Parque Tecnológico Capital Digital ainda em vias de lançamento pelo governo local. Assim como no cenário de investimento as aceleradoras locais também ocupam um forte papel próximo as incubadoras, sendo estas responsáveis por abrigar startups em períodos iniciais e prepara-las para posteriormente aplicarem para uma residência de médio a longo prazo com outros atores. Há previsão para que em 2018 o Governo de Brasília lance dois editais que beneficiam diretamente o ecossistema local: um para seleção de startups para que ocupem um prédio dedicado para incubação e aceleração no Parque Tecnológico Capital Digital e outro para aceleradoras (chamadas "Agentes de Inovação") que serão responsávéis pela gestão do espaço e suporte aos empreendedores.

O ambiente regulatório local talvez seja dos fatores que merecem maior destaque, atualmente o Distrito Federal é uma das únicas unidades da federação onde é possível abrir uma empresa de baixo risco (classificação que a maior parte das empresas de tecnologia da informação, como as startups estudadas por este trabalho, recebem) pela internet em até 5 dias. Foi feito um levantamento com empreendedores locais para verificar se o novo processo de abertura criado pelo governo realmente funciona e a resposta foi positiva, atualmente o novo processo está sendo implementado em outros estados do país. O fato do Distrito Federal compartilhar das competências de Estados e Munícipios também é visto como um fator positivo, o que reduz a burocracia para o empreendedor local e pode facilitar a criação de políticas públicas. Atualmente dois grupos que buscam melhorar o ambiente regulatório brasileiro para startups, a Dínamo e a Endeavor, interagem com frequência com o Governo Federal e por isso possuem representantes no DF. Vale ressaltar que esse é um dos indicadores propostos por esse trabalho e que tomam como referência indicadores criados pela Endeavor no Índice de Cidades Empreendedoras, trabalho citado como uma das referências bibliográficas deste.

A Cultura Empreendedora, quiça dos indicadores criticos do Distrito Federal, não apenas recebe a menor classificação como também foi apontada como a pior do país e duramente criticada por empreendedores locais. Conforme o ecossistema amadureça, a mídia dê um maior enfoque ao empreendedorismo, mais eventos aconteçam no estado e as gerações de empreendedores consigam perpetuar seu legado esse indicador pode ser melhorado.

De acordo com o proposto por \citeonline{Cukier2016} o nível de maturidade de um determinado ecossistema será classificado de acordo com o resultados dos indicadores analisados, para que o ecossistema receba a classificação de um determinado nível ele deve apresentar o mesmo nível ou superior em, no mínimo, 7 dos indicadores estudados. Utilizando o proposto por Cukier, o Distrito Federal apresenta seis dos sete indicadores em nível crescente ou maduro, tendo apenas sua cultura empreendedora como nascente. Como explicitado, o indicador 
empreendedorismo nas universidades não foi considerado por falta de dados. 

Como neste pesquisa foram avaliados nove indicadores ao todo temos oito, dos nove indicadores, classificados como Crescente ou Maduro e um classificado como Nascente, o de cultura empreendedora. Portanto, o nível de maturidade do ecossistema do Distrito Federal, de acordo com a metodologia proposta por este estudo, é Crescente. Para que haja uma evolução para um maior nível de maturidade o DF precisa ampliar suas opções de eventos sobre empreendedorismo, estratégias de saída, investimento anjo, atores da mídia cobrindo o cenário empreendedor local e, principalmente, amadurecendo sua cultura empreendedora.

\begin{figure}[!htb]
	\centering
	\includegraphics[width=17cm,angle=0]{figuras/ecossistema}
	\caption{Mapa do Ecossistema de Startups do Distrito Federal}
	\label{figure:ecossistema}
\end{figure} 
