\chapter[Resultados Parciais]{Resultados}
\label{cap-resultados}

A exploração do Ecossistema de Startups de Brasília tem sido deveras interessante, primeiro pela extensa rede de contatos que venho contruindo graças à essa pesquisa mas também por em tão pouco tempo já permitir uma visualização superficial e uma compreensão de diversos elementos desse Ecossistema e a razão de certas características locais.

Traçando uma espécie de linha temporal do Ecossistema de Startups do Distrito Federal, claramente a figura de uma forte liderança e de um Ecossistema unido e integrado era mais clara entre meados de 2010 e 2012, diversos Empreendedores citaram um grupo chamado ``Startup Brasília'' composto por Empreendedores de empresas como a Intacto, Qual Canal, SEA Tecnologia, Rota dos Concursos, IPê Tecnologia, Trip2gether, etc. 

Nessa época os encontros - também conhecidos como meetups - também eram mais frequentes e, segundo alguns Empreendedores, de maior qualidade. Foi relatado que os representantes do Startup Brasília muitas vezes atraiam palestrantes que eram referência em todo o Brasil e pagavam do próprio bolso para que víessem para atrair os Empreendedores, eram líderes natos do Ecossistema.

Mesmo sem um grande canal para investimentos na cidade um dos Empreendedores mencionou que foi uma época em que Brasília entrou no radar como um dos melhores Ecossistemas de Startups do Brasil. A presença do Governo também era forte, muitos Empreendedores citaram o constante apoio do SEBRAE DF como essencial para o crescimento do Ecossistema na época. A união entre o SEBRAE e o grupo Startup Brasília foi o principal ponto para que em 2012 Brasília tivesse a maior delegação Brasileira no Tech Crunch Disrupt, um grande marco do nosso Ecossistema. Nessa mesma época também houve uma edição do Startup Farm em Brasília, atraida por esses atores.

Infelizmente, segundo alguns Empreendedores, esse foi o auge da cidade. Chegou a ser discutida a possibilidade de ser criada uma Associação que melhor representasse o Ecossistema, principalmente perante ao Estado, mas o grupo optou por não seguir esse caminho e coincidentemente após esse momento alguns dos líderes do Ecossistema deixaram o país, alguns para serem acelerados no Vale do Silício, e a pessoa chave no Sebrae responsável pelas Startups também fora transferida para outra área. 

Esse é o ponto em que um dos Empreendedores classifica como o momento em que o Ecossistema do Distrito Federal começou a perder sua força, entre meados de 2012 e 2014. Também foi quando nasceu a Associação de Startups e Empreendedores Digitais, criada por outro grupo de Empreendedores, e naturalmente tomou a posição de liderança e referência do Ecossistema mas sem o apoio e a confiança desse antigo grupo.

Um dos Empreendedores menciona que entre 2015 e 2016, o Ecossistema de Startups de Brasília voltou a reagir. Ele não sabe dizer o que encadeou o movimento, mas diz que o cenário definitivamente não é mais o mesmo e voltou a crescer. Com a presença de dois grandes atores de investimento trazendo cerca de R\$ 100 milhões para Startups e três programas de aceleração se estabelecendo esse Empreendedor acredita que o Ecossistema de Brasília logo voltará a ser o que era se os atores se unirem e formarem um grupo forte novamente.

\section{Perguntas de Pesquisa}
\label{section:perguntas_de_pesquisa}

\subsection{Pergunta de Pesquisa 1: Quais são as características socioculturais de Brasília que promovem ou inibem o espirito empreendedor?}
\label{subsection:pergunta_de_pesquisa_1}

A Endeavor cita a Cultura Empreendedora de Brasília como a pior do Brasil\footciteref{indiceglobaldoempreendedorismo}, mas esse cenário claramente está mudando e iniciativas estão nascendo em Brasília. 

Alguns empreendedores relatam que em Brasília, atualmente, acontece no mínimo um evento relacionado a Empreendedorismo e Startups por semana como um fator muito positivo, mas muitos afirmam que a aversão ao risco do brasiliense e o desejo pelo funcionalismo público ainda são fatores preocupantes que inibem o Empreendedorismo.

Um empreendedor relatou que os Empreendedores de Brasília ainda são muito imaturos e, segundo ele, é comum encontrar profissionais com altos salários, sendo muitos servidores públicos, dispostos a Empreender em seu tempo livre, mas o mesmo acredita que dedicação integral é necessária para o sucesso de qualquer negócio, se você o Empreendedor não investe o que ele tem de mais precioso, eu tempo, então é porque não acredita o bastante no negócio. Ele diz que falta coragem no brasiliense para empreender ``de verdade''.

\subsection{Pergunta de Pesquisa 2: Quais são os mecânismos institucionais de Brasília que promovem ou dificultam o Empreendedorismo?}
\label{subsection:pergunta_de_pesquisa_2}

Um representante de aceleradora mencionou que Brasília possui um Ambiente Regulatório favorável, visto que as empresas do Distrito Federal não precisam responder à legislações municipais, apenas estaduais e federais, o que facilita ações do governo com o objetivo de apoiar os empreendedores com a criação de um ambiente regulatório mais amigável, mas o mesmo relata pouco interesse por parte dos líderes políticos em apoiar Startups e empresas de tecnologia.
  
\subsection{Pergunta de Pesquisa 3: Quais são os mecânismos educacionais de Brasília que promovem o Empreendedorismo?}
\label{subsection:pergunta_de_pesquisa_3}

A Endeavor cita a Cultura Empreendedora de Brasília como a pior do Brasil\footciteref{indiceglobaldoempreendedorismo}, mas esse cenário claramente está mudando. Já é comum encontrar nichos de estudantes universitários frequentando os eventos relacionados à Startups da cidade e pequenos núcleos se formando, como a Liga Universitária Marco Zero. 

Um dos membros da Associação de Startups e Empreendedores Digitais mencionou um projeto educacional com duração de seis meses que acontecerá em Agosto em quatro universidades de forma simultânea, onde os alunos terão a oportunidade de desenvolver muitas das competências de um Empreendedor recebendo mentoria e desenvolvendo uma Startup, e ao fim do programa farão um concurso entre os melhores de cada universidade em um Demoday.

Um dos entrevistados menciona que o Movimento Empresa Junior é muito forte na Universidade de Brasília e relata que a importância das Empresas Juniores para o desenvolvimento de diversas competências empreendedoras necessárias, mas também crítica a falta de disciplinas que desenvolvam essas competências, incentivem os alunos e melhor se adequem ao contexto das Startups. Observando as universidades privadas, é notável que elas estão bem mais interessadas em incentivar seus alunos a Empreenderem e estão melhor preparadas nesse quesito.

Atualmente Brasília possui duas Incubadoras de Empresas em contextos universitários e um Parque Tecnológico. Uma terceira universidade atualmente está criando sua própria incubadora que deve ficar pronta ainda em 2016.

\subsection{Pergunta de Pesquisa 4: Como os fatores tecnológicos influenciam o sucesso ou fracasso das Startups de Brasília? Qual o papel executado pela comunidade e pelo Software Livre?}
\label{subsection:pergunta_de_pesquisa_4}

Um dos Empreendedores relatou que considera um grande erro utilizar tecnologias modernas e amplamente utilizadas por outros Ecossistemas de Startups como Ruby on Rails, Python, Swift, etc pela falta de profissionais capacitados e acessíveis. Ele relata que por diversas vezes manteve uma vaga aberta por meses por não encontrar o profissional ideal, e diz que para uma Startup é inviável. 

Você precisa crescer rápido, e o mercado precisa suprir suas necessidades de escala ainda mais rápido. Em sua Startup atual ele optou por utilizar Java e PHP, e por ter como uma grande base do seu negócio em Aprendizado de Máquina e Inteligência de Negócio seu principal foco é atrair profissionais de banco, que segundo estão em abundância no mercado e são mais baratos do que um bom desenvolvedor de alguma dessas novas linguagens. O mesmo fez o seguinte comentário: ``Se a sua Startup consegue tracionar, escalar muito rápido e surgir a demanda de 150 programadores Node em Brasília o que você faz? Não devem ter 150 programadores cadastrados nas comunidades de Node, dirá disponíveis no mercado. Ou você vai morrer ou vai precisar gastar uma fortuna trazendo gente de fora. Com Java e PHP eu tenho a segurança de um mercado com profissionais em abundância disponíveis''.

Outro Empreendedor menciona a crise e o corte nos gastos públicos como um ótimo fator para Startups de Tecnologia, ele relata haver centenas de ex-terceirizados dos órgãos públicos que são ótimos programadores disponíveis no mercado de Brasília.

\subsection{Pergunta de Pesquisa 5: Qual a relação do empreendedor de Brasília com as opções de investimento disponíveis e como elas influenciam o Ecossistema?}
\label{subsection:pergunta_de_pesquisa_5}

Como relatado por um dos membros do ``Startup Brasilia'' em meados de 2012 eles não tinham capital para Startups no Distrito Federal, diferente de hoje que existem cerca de R\$ 100 milhões disponíveis no mercado na mão de atores privados, por meio da Cedro Capital e da Garan Ventures, diversos investidores anjos, muitos deles empreendedores bem sucedidos dessa primeira leva de Startups, e fundos de subvenção pública mais acessíveis. A presença de Aceleradoras também foi bastante mencionada, principalmente a Acceleratus, Impulso e Cotidiano. 

Um dos Empreendedores mencionou que existe um representante da Anjos do Brasil no Ecossistema, mas que o mesmo não demonstra interesse em investir na capital.

Pelo mesmo motivo da falta de maturidade e dedicação integral dos Empreendedores mencionado na subseção \ref{subsection:pergunta_de_pesquisa_1} um Empreendedor menciona que existe capital em abundância no Distrito Federal, mas que muitos investidores preferem procurar Startups em cidades como Florianopólis, Belo Horizonte, Recife ou São Paulo, onde os Empreendedores lidam melhor com o risco e com a falta de segurança e estabilidade.

\subsection{Pergunta de Pesquisa 6: Quais ações devem ser tomadas no Ecossistema de Brasília para que ele cresça?}
\label{subsection:pergunta_de_pesquisa_6}

Alguns empreendedores mencionam a falta de uma liderança que una os atores e a constante ``briga'' de egos entre os atores como um dos fatores limitadores.

Um Empreendedor mencionou um grupo composto por Governo, Universidades, Empreendedores, Aceleradoras e Associações que tem se encontrado mensalmente com o objetivo de discutir ações de fomento ao Ecossistema, mas que mesmo com a baixa colaboração entre os atores o grupo pode ser importante para o Ecossistema. 

Outro empreendedor mencionou a necessidade de se criar um Conselho para discussões mensais sobre Startups criado pelo Governo para que ações com o objetivo de flexibilidar e apoiar iniciativas empreendedoras sejam tomadas.

Um dos Empreendedores mencionou a grande queda que o Ecossistema de Brasília sofreu entre 2013 e 2015, com a separação do antigo grupo do ``Startup Brasília'', mas diz que o crescimento será natural nos próximos anos com a volta desse grupo. Ele nota que os antigos atores estão quase todos envolvidos em pelo menos uma das três iniciativas de aceleração disponíveis na cidade, e que dessa forma eles voltarão a causar impacto constante e a explorar sua rede de contatos para trazer iniciativas de fomento para Brasília.