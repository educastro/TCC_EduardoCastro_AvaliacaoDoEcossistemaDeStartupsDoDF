\chapter[Resultados]{Resultados}
\label{cap-resultados}

Este trabalho começou a ser elaborado em meados de Janeiro e Fevereiro de 2016, tendo seu início marcado por extensos estudos por livros e publicações com relação aos temas de empreendedorismo ou ecossistemas/hubs de inovação, também foram realizadas 16 entrevistas com atores do ecossistema de startups do Distrito Federal com o objetivo de explorar o cenário local e responder as questões propostas.

\section{Respostas para as Questões de Pesquisa}
\label{section:perguntas_de_pesquisa}

\subsection{QP1: Quais são as características socioculturais do DF que promovem ou inibem o espirito empreendedor?}
\label{subsection:pergunta_de_pesquisa_1}

A \citeonline{indiceglobaldoempreendedorismo} cita a Cultura Empreendedora de Brasília como a pior do Brasil, mas há indícios de que esse cenário possa estar mudando quando observamos as iniciativas que estão nascendo em Brasília como, por exemplo, a participação de 5056 pessoas em grupos relacionados a encontros de tecnologia, empreendedorismo, inovação e/ou startups na rede social Meetup.com.

Diversos empreendedores relataram a forte cultura em torno da estabilidade, do salário disponível na conta ao quinto dia útil do mês e do emprego seguro no Distrito Federal, muitos acreditam ser por influência do funcionalismo público, dos benefícios garantidos pela leis trabalhistas que os regem e do que a presença da estrutura governamental representa. Um dos empreendedores relatou que não encontra com facilidade jovens no Distrito Federal com ambição para criar e crescer, com vontade de empreender e dedicar \"horas e mais horas programando madrugada adentro\", o mesmo relatou que mesmo hoje, após participar da criação de uma empresa de tecnologia que fatura aproximadamente R\$ 160 milhões por ano, o pai o enxerga como um fracassado e tem o irmão, servidor público, como símbolo de sucesso na família, percepções similares foram relatadas por vários outros entrevistados. Outro empreendedor descreveu essa percepção como \"muitos preferem ser passageiros do que os criadores de um veículo\" e que falta um maior senso de curiosidade, outro como \"falta ritmo de trabalho\" no brasiliense, esse fator foi um dos motivos pela qual um dos empreendedores definiu a experiência de contratar para startup no Distrito Federal como: \"terror, pânico e aflição\".

Também foi falado sobre a falta de disposição do brasiliense em aceitar a possibilidade de dedicar anos para uma startup sem qualquer retorno financeiro, mesmo entre jovens universitários que poderiam usufruir da segurança financeira oferecida pelos pais, o baixo custo de vida e a ausência de maiores compromissos familiares e profissionais para empreender. Um empreendedor mencionou que o baixo número de pessoas dispostas a executar e a correr riscos com o empreendedorismo é um problema presente em todo o país, mas no Distrito Federal relatou ter a impressão de ser mais forte. Um dos empreendedores relatou que o \"embate\" das startups para obter os melhores profissionais em Florianopólis é com outras empresas, em São Paulo com multinacionais e em Brasília com o Estado, que é para onde o mesmo relata que vão a maior parte dos melhores profissionais.

Foi identificada também que essa visão de \"funcionário\" está presente até mesmo na forma como as pessoas se relacionam na cidade, segundo o mesmo de uma forma \"hierarquizada\" por não haver uma mentalidade de abundância ou colaboração entre os empreendedores da cidade, o que os afastam como grupo social e criam barreiras invisíveis e fortalecem os grupos fechados. Esse afastamento também reflete de outras formas na dinâmica do ecossistema do Distrito Federal, por haver baixa interação entre os atores é comum que as diversas ações de fomento ao ecossistema não conversem entre si, muitas vezes gerando até mesmo eventos co-relacionados competindo pelo mesmo público no mesmo dia/horário, ou que haja afastamento com as grandes empresas de tecnologia da região.

Essa proximidade e baixa barreira para interação entre atores foi relatada por um dos empreendedores como um dos pontos fortes de ecossistemas como o de Belo Horizonte, conhecido como \"San Pedro Valley\", ou em Florianópolis. Foi relatado que nesses ecossistemas é comum que os empreendedores se encontrem com frequência, organizem \"happy hours\" e façam reuniões entre si com o fim de gerar colaboração como ecossistema. Um dos empreendedores do Distrito Federal queixou-se da dificuldade de se reunir com outros e o quão desmotivador isso pode ser durante os dias difíceis da vida de empresário. Um empreendedor descreveu que um dos motivos para esse distanciamento é a própria arquitetura e dinâmica de mobilidade urbana da cidade, o mesmo relatou que enquanto em Belo Horizonte é comum encontrar vários empreendedores almoçando na mesma região ou saindo para tomar um café na mesma padaria do bairro em Brasília as pessoas tendem a se distanciar.

Um empreendedor relatou que as startups locais ainda são muito imaturas e ser comum encontrar profissionais com altos salários, sendo muitos servidores públicos, empreendendo durante o seu tempo livre, mas o mesmo acredita que dedicação integral é um fator crucial para o sucesso de qualquer negócio. O mesmo empreendedor relatou a impressão de que falta coragem no brasiliense para empreender ``de verdade''. O mesmo entrevistado relatou ser clara a diferença entre uma startup do Distrito Federal com 6 meses de vida e uma startup de São Paulo, segundo ele as nossas mal sabem validar o produto, não aplicam as metodologias corretamente e, talvez pela natureza da cidade, estão acostumadas a sobreviver por meio de editais públicos de investimento.

Uma empreendedora acredita que esses fatores negativos são naturais e parte do desenvolvimento que Brasília ainda vai enfrentar. Segundo ela o próprio termo empreender é recente na cidade e até pouco tempo atrás essa não era uma realidade, aos poucos as mudanças culturais de qual necessitamos acontecem e as peças do ecossistema vão se complementando.

Quanto as grandes empresas de tecnologia, embora tenham interesse em se aproximar das startups, não o fazem por não haver um ecossistema estruturado ou canais para que a interação aconteça, foi dito também que as mesmas entendem a importância do cenário de startups a níveis globais mas não compreendem como aproximar essas inovações do seu modelo de negócios.

Encabeçando a mudança de cultura em Brasília alguns movimentos como o Startupeiro, o Cerrado Valley, o Garagem Dextra, universidades, aceleradoras e a Associação de Startups e Empreendedores Digitais (ASTEPS) tentam promover encontros com o objetivo de conectar o ecossistema mas a questão dos grupos que não conversam entre si ainda se mostram uma grande barreira para um ambiente de colaboração sádio no Distrito Federal. Outro fator que também contribui para a geração de conexões entre atores é o crescimento dos espaços de coworking da capital, foram mapeados sete que atuam de forma proativa junto ao ecossistema local e há mais quatro que devem ser lançados até o fim de 2017, atualmente há eventos que conversam com a pauta do empreendedorismo e startups quase que semanalmente e isso é visto como um fator muito positivo. 

Outro fator chave relatado por vários dos empreendedores como um momento essencial para a sua formação empreendedora fora a participação no Movimento Empresa Junior enquanto universitários não apenas pela experiência profissional como também pelas redes de contato formadas no período e toda a \"energia empreendedora\" muito presente neste circulo estudantil. 

Do ponto de vista de mercado, um dos empreendedores relatou que Brasília, por ser uma cidade muito conectada, com boa infraestrutura de tecnologia e população com alto poder aquisitivo, se mostra excelente para estratégias com foco na aquisição de usuários finais (\"Go-to-Market\"). Segundo ele, além de muitos usuários de planos pós-pago com boas conexões de internet móvel há muitas pessoas dispostas a serem \"early adopters\", como são chamados os primeiros usuários a adotarem uma nova tecnologia.

\begin{table}
\centering
\begin{tabular}{ | c | c | c | c | c |}
\hline
\thead{Universidades} & \thead{Aceleradoras} & \thead{Espaços\\de\\Coworking} & \thead{Organizações\\de\\Apoio} & \thead{Mídia} \\
\hline
\makecell{UnB} & \makecell{Accelerattus} & \makecell{55lab} & \makecell{ASTEPS} & \makecell{Metropoles} \\
\hline
\makecell{UniCEUB} & \makecell{Cotidiano} & \makecell{Multiplicidade} & \makecell{Garagem\\DEXTRA} & \makecell{Jornal\\de\\Brasília} \\
\hline
\makecell{IESB} & \makecell{Acelere.me} & \makecell{Copiloto} &  \makecell{Startupeiro} & \makecell{Bizmeet} \\
\hline
\makecell{UDF} & \makecell{UniCEUB\\(Impulso\\e\\InovaSabin)} & \makecell{Manifesto} & \makecell{Cerrado\\Valley} & \makecell{Correio\\Braziliense}\\
\hline
\makecell{UCB} & \makecell{Lemonade} & \makecell{The\\Brain} & \makecell{Movimento\\Empresa\\Junior} & \makecell{} \\
\hline
\makecell{CTJ} & \makecell{} & \makecell{W3\\Work} & \makecell{Impact\\Hub} & \makecell{} \\
\hline
\makecell{} & \makecell{} & \makecell{Nós} & \makecell{Marco\\Zero} & \makecell{} \\
\hline
\makecell{} & \makecell{} & \makecell{} & \makecell{Fundação\\Estudar} & \makecell{} \\
\hline
\makecell{} & \makecell{} & \makecell{} & \makecell{Endeavor} & \makecell{} \\
\hline
\end{tabular}

\caption{Organizações que trabalham a cultura empreendedora no DF}
\label{table:metricas_de_classificacao_dos_fatores}
\end{table}

\subsection{QP2: Quais são os mecânismos institucionais de Brasília que promovem ou dificultam o Empreendedorismo?}
\label{subsection:pergunta_de_pesquisa_2}

Fazendo um contraponto aos impactos apontados como negativos quanto a forte influência do governo no Distrito Federal o mesmo também é o maior comprador de tecnologia do país, apenas o governo federal, em 2015, gastou cerca de R\$ 5 bilhões em equipamentos e serviços de TI. A presença de tamanho orçamento também faz com que muitas das grandes empresas de tecnologia do país tenham escritórios no Distrito Federal, além de grandes brancos, o próprio Banco Central, etc. Essa proximidade natural com um grande comprador foi apontada como um potencial atrativo e uma vantagem para as startups do DF, podendo, inclusive, vir a se tornar uma das vocações da região.

Do ponto de vista regulatório a proximidade também é positiva, tanto pela possibilidade de construção de políticas públicas como também pela facilidade de simplificação, visto que as empresas do Distrito Federal respondem a apenas duas esferas governamentais que fisicamente estão muito próximas: o  distrital e o federal, essa proximidade fez com que o Distrito Federal fosse o primeiro \"Estado\" brasileiro a testar um novo processo para abertura de empresas de baixo risco 100\% digital, dispensando até mesmo que o empresário compareça a junta comercial para assinar documentos. Esse novo processo fez com que fosse possível abrir uma empresa em menos de 7 dias, um recorde no país.

Um ponto interessante que fora observado refere-se ao fato de que os grandes programas de apoio e aceleração à startups do governo federal, como o Startup Brasil e o Inovativa Brasil, serem executados por equipes em Brasília, ainda assim nota-se que a presença de projetos oriundos do DF não são muito significativas, das 549 startups aceleradas pelo Inovativa, por exemplo, apenas 20 da capital, pouco mais de 3\%. Ao conversar com um dos responsáveis pelo programa um ponto chave na divulgação do programa foi a atuação do SEBRAE local de cada região, um dos empreendedores também descreveu a atuação do SEBRAE como um dos pontos chaves para o crescimento de ecossistema de Florianopólis. Durante este trabalho os empreendedores entrevistados apresentaram relatos contraditórios quanto a atuação da entidade no Distrito Federal.

Outro fator positivo mencionado como um fator positivo foi a presença de grandes organismos internacionais e fundações com grande poder de compra e/ou investimento em startups que apresentem soluções de seu interesse, um dos motivos pelos quais um dos empreendedores acredita que as melhores oportunidades para os empreendedores do Distrito Federal estão concentradas nos negócios e soluções de impacto social. Ele também acredita que o Brasil é um país com muitos desafios sociais latentes ao mesmo tempo em que possui uma população bem conectada, o que cria um cenário favorável para startups que queiram explorar essas oportunidades.

O SEBRAE, apontado por alguns entrevistados como um dos principais apoiadores durante a boa fase de crescimento do ecossistema de startups de Brasília em meados de 2010 a 2012 atualmente não possui atuação tão significativa, embora esteja presente e ofereça, além de suporte para os empreendedores, uma série de atividades para o público interessado em startups em um espaço chamado SEBRAE Lab, uma espécie de \"coworking público\".

Ao todo estão presentes quatro aceleradoras, cerca de vinte espaços de coworkings, duas pré-aceleradoras, dois fundos de capital semente ligados ao fundo Criatec, um fundo de venture capital, uma representação de investidores anjos e diversas representações de empresários, sendo uma delas focada em startups e outras duas de jovens empresários que também conversam com o mercado de startups. O Distrito Federal possui, também, quatro incubadoras, dois parques tecnológicos e a presença, recém-chegada, da Endeavor que veio para Brasília com o objetivo de trabalhar melhores práticas para a construção de um melhor ambiente regulatório no DF e no Brasil. O governo local também busca formas de investir em startups, atualmente a principal política pública para esse meio é o Startups Brasília, um edital de subvenção econômica para investimento de até R\$ 150.000,00 para startups em estágio nascente. Alguns empreendedores definiram o referido edital como \"mal estruturado\" e \"perigoso\" para os empreendedores por conta da alta burocracia e complexidade ligada ao dinheiro.

Uma observação interessante é que boa parte dessas \"peças\" que compõem o ecossistema de startups do Distrito Federal surgiram nos últimos 3 anos

startup farm 2012
startup weekend 2012
rota dos concursos 2010 - 2011 500s
quando o canalcanal foi em 2012

Diversos atores também criticaram as incubadoras do Distrito Federal pelas mesmas não oferecerem condições interessantes e para o processo de incubação e pequena oportunidade de crescimento, um dos entrevistados relatou insatisfação com o baixo número de conexão com investidores e representantes de grandes empresas que o Centro de Desenvolvimento Tecnológico, responsável pelo programa de incubação da Universidade de Brasília, proporcionava.

Uma crítica apontada por vários empreendedores refere-se a falta de uma lideranca no ecossistema, por diversas situações passadas, que não convém serem explorados neste trabalho, os empreendedores de startups do Distrito Federal se dividiram em torno da Associação de Startups e Empreendedores Digitais. Como explorado por \citeonline{Feld2012} Rainda =resta uma pessoa, ou um grupo de pessoas, que domine a dinâmica do ecossistema e seja responsável por \"quebrar as barreiras invisíveis".

Atualmente três canais de comunicação (dois jornais e um impresso) do Distrito Federal criaram colunas com o fim especifico de promover o empreendedorismo no Distrito Federal, espaços onde as startups locais figuram como protagonistas com frequência.


\subsection{QP3: Quais são os mecânismos educacionais de Brasília que promovem o Empreendedorismo?}
\label{subsection:pergunta_de_pesquisa_3}

Falar das incubadoras seguindo o relato do Marcos

Universidades como a Universidade de Brasília, o UniCEUB, o IESB e a UDF foram mencionadas como as mais interessadas em criar envolvimentos com as startups, o que é um fator positivo. Em especial nas particulares já existem uma série de ações pontuais como programas de aceleração, criação de \"maker spaces\" e \"coworkings\" e reformulação das disciplinas de empreendedorismo.

A \citeonline{Endeavor} cita a cultura empreendedora de Brasília como a pior do Brasil mas esse cenário claramente está mudando. Já é comum encontrar nichos de estudantes universitários frequentando diversos eventos relacionados à startups e pequenos núcleos se formando, como a Liga Universitária Marco Zero. 

Um dos membros da Associação de Startups e Empreendedores Digitais mencionou um projeto educacional com duração de seis meses que acontecerá em agosto em quatro universidades de forma simultânea, onde os alunos terão a oportunidade de desenvolver muitas das competências de um Empreendedor recebendo mentoria e desenvolvendo uma startup, e ao fim do programa farão um concurso entre os melhores de cada universidade em um evento de demonstração de projetos.

Um dos entrevistados menciona que o Movimento Empresa Junior é muito forte na Universidade de Brasília e relata que a importância das empresas juniores para o desenvolvimento de diversas competências empreendedoras necessárias, mas também crítica a falta de disciplinas que desenvolvam essas competências, incentivem os alunos e melhor se adequem ao contexto das startups. Observando as universidades privadas, é notável que elas estão bem mais interessadas em incentivar seus alunos a empreenderem e estão com disciplinas e programas educacionais melhor preparados.

Atualmente Brasília possui duas incubadoras de empresas em contextos universitários e um parque tecnológico ativos. Uma terceira universidade está criando sua própria incubadora e o governo local possui um projeto de criação do Parque Tecnológico Capital Digital em uma área com quase 1 kilômetro quadrado destinado para empresas da área de tecnologia.

\subsection{QP4: Como os fatores tecnológicos influenciam o sucesso ou fracasso das Startups de Brasília? Qual o papel executado pela comunidade e pelo Software Livre?}
\label{subsection:pergunta_de_pesquisa_4}

Um dos empreendedores relatou que considera um grande erro utilizar tecnologias modernas e amplamente utilizadas por outros Ecossistemas de Startups como Ruby on Rails, Python, Swift, etc pela falta de profissionais capacitados e acessíveis. Ele relata que por diversas vezes manteve uma vaga aberta por meses por não encontrar o profissional ideal, e diz que para uma startup é inviável. 

Você precisa crescer rápido, e o mercado precisa suprir suas necessidades de escala ainda mais rápido. Em sua Startup atual ele optou por utilizar Java e PHP, e por ter como uma grande base do seu negócio o aprendizado de máquina e diversas técnicas de inteligência de negócio seu principal foco é atrair profissionais de banco, que segundo o empreendedor estão em abundância no mercado e são mais baratos do que um bom desenvolvedor de linguagens mais modernas. 

O mesmo empreendedor fez o seguinte comentário: ``Se a sua Startup consegue tracionar, escalar muito rápido e surgir a demanda de 150 programadores Node em Brasília o que você faz? Não devem ter 150 programadores cadastrados nas comunidades de Node, dirá disponíveis no mercado. Ou você vai morrer ou vai precisar gastar uma fortuna trazendo gente de fora. Com Java e PHP eu tenho a segurança de um mercado com profissionais em abundância disponíveis''.

Outro Empreendedor menciona a crise e o corte nos gastos públicos como um ótimo fator para startups de tecnologia, ele relata haver centenas de ex-terceirizados dos órgãos públicos que são ótimos programadores disponíveis no mercado de Brasília.

\subsection{QP5: Qual a relação do empreendedor de Brasília com as opções de investimento disponíveis e como elas influenciam o Ecossistema?}
\label{subsection:pergunta_de_pesquisa_5}

Como relatado por um dos membros do ``Startup Brasilia'' em meados de 2012 eles não tinham capital para Startups no Distrito Federal, diferente de hoje que existem cerca de R\$ 100 milhões disponíveis no mercado na mão de atores privados, por meio da Cedro Capital e da Garan Ventures, diversos investidores anjos, muitos deles empreendedores bem sucedidos dessa primeira leva de Startups, e fundos de subvenção pública mais acessíveis. A presença de aceleradoras também foi bastante mencionada, principalmente a Acceleratus, Impulso e Cotidiano. 

Um dos empreendedores mencionou que existe um representante da Anjos do Brasil no Ecossistema, mas que o mesmo não demonstra interesse em investir na capital.

Pelo mesmo motivo da falta de maturidade e dedicação integral dos Empreendedores mencionado na subseção \ref{subsection:pergunta_de_pesquisa_1} um empreendedor menciona que existe capital em abundância no Distrito Federal, mas que muitos investidores preferem procurar Startups em cidades como Florianopólis, Belo Horizonte, Recife ou São Paulo, onde os empreendedores lidam melhor com o risco e com a falta de segurança e estabilidade.

\subsection{QP6: Quais ações devem ser tomadas no Ecossistema de Brasília para que ele cresça?}
\label{subsection:pergunta_de_pesquisa_6}

Alguns empreendedores mencionam a falta de uma liderança que una os atores e a constante ``briga'' de egos entre os atores como um dos fatores limitadores.

Um empreendedor mencionou um grupo composto por governo, universidades, empreendedores, aceleradoras e associações que tem se encontrado mensalmente com o objetivo de discutir ações de fomento ao ecossistema, mas que mesmo com a baixa colaboração entre os atores o grupo pode ser importante para o ecossistema. 

Outro empreendedor mencionou a necessidade de se criar um conselho para discussões mensais sobre startups criado pelo Governo para que ações com o objetivo de flexibilidar e apoiar iniciativas empreendedoras sejam tomadas.

Um dos empreendedores mencionou a grande queda que o ecossistema de Brasília sofreu entre 2013 e 2015, com a separação do antigo grupo do ``Startup Brasília'', mas diz que o crescimento será natural nos próximos anos com a volta desse grupo para a cidade. Ele nota que os antigos atores estão quase todos envolvidos em pelo menos uma das três iniciativas de aceleração disponíveis na cidade, e que dessa forma eles voltarão a causar impacto constante e a explorar sua rede de contatos para trazer iniciativas de fomento para Brasília.

\section{Considerações pré-eliminares}
\label{consideracoes_pre_eliminares}

Traçando uma espécie de linha temporal do ecossistema de startups do Distrito Federal, claramente a figura de uma forte liderança e de um ecossistema unido e integrado era mais clara entre meados de 2010 e 2012, diversos empreendedores citaram um grupo chamado ``Startup Brasília'' composto por Empreendedores de empresas como a Intacto, Qual Canal, SEA Tecnologia, Rota dos Concursos, IPê Tecnologia, Trip2gether, etc. 

Nessa época os encontros - também conhecidos como meetups - eram mais frequentes e, segundo alguns empreendedores, de maior qualidade. Foi relatado que os representantes do Startup Brasília muitas vezes atraiam empreendedores que eram referência em todo o Brasil e em alguns momentos pagaram do próprio bolso para que víessem ministrar palestras em Brasília.

Mesmo sem um grande canal para investimentos na cidade um dos empreendedores mencionou que foi uma época em que Brasília entrou no radar como um dos melhores ecossistemas de startups do Brasil. A presença do governo também era forte, muitos Empreendedores citaram o constante apoio do SEBRAE DF como essencial para o crescimento do Ecossistema na época. A união entre o SEBRAE e o grupo Startup Brasília foi o principal ponto para que, em 2012, Brasília tivesse a maior delegação brasileira no Tech Crunch Disrupt, momento que ainda é mencionado por muitos como um grande marco do ecossistema. Nessa mesma época, por incentivo de alguns desses atores, também houve um programa de aceleração da Startup Farm\footciteref{StartupFarm} em Brasília, uma das maiores aceleradoras de startups da América Latina.

Muitos dos participantes desse grupo e do ecossistema em 2012 estão representados na Figura \ref{figure:startups_board_2012}, criada pelo empreendedor Marcos Oliveira.

\begin{figure}[!htb]
	\centering
	\includegraphics[width=15cm,angle=0]{figuras/startups_board_2012}
	\caption{Representantes do Ecossistema de Startups de Brasília em 2012}
	\label{figure:startups_board_2012}
\end{figure}

Infelizmente, segundo alguns empreendedores, esse foi o auge da cidade. Chegou a ser discutida a possibilidade de ser criada uma Associação que melhor representasse o Ecossistema, principalmente perante ao Estado, mas o grupo optou por não seguir esse caminho e coincidentemente após esse momento alguns dos líderes do Ecossistema deixaram o país, alguns para serem acelerados no Vale do Silício, e a pessoa chave no Sebrae responsável pelas startups também fora transferida para outra área. 

Esse é o ponto em que um dos empreendedores classifica como o momento em que o ecossistema do Distrito Federal começou a perder sua força, entre meados de 2012 e 2014. Também foi quando nasceu a Associação de Startups e Empreendedores Digitais (ASTEPS), criada por outro grupo de empreendedores, e naturalmente tenta tomar a posição de liderança e referência do ecossistema mas sem o apoio, envolvimento e confiança de alguns atores desse antigo grupo.

Um dos empreendedores mencionou que entre 2015 e 2016, o ecossistema de startups de Brasília voltou a reagir. Ele não sabe dizer o que encadeou o movimento, mas diz que o cenário definitivamente não é mais o mesmo e voltou a crescer. Com a presença de dois grandes atores de investimento trazendo cerca de R\$ 100 milhões para startups e três programas de aceleração se estabelecendo esse empreendedor acredita que o ecossistema de Brasília logo voltará a ser o que era se os atores se unirem e formarem um grupo forte novamente.