\begin{apendicesenv}

\partapendices

\chapter{Fatores de um Ecossistema}
\label{apendices:fatores_de_um_ecossistema}

\begin{description}

  \item [Estratégias de Saída:] Quando falamos de Estratégias de Saída falamos de formas de transformar uma empresa em capital, em converter ações em dinheiro real. Investidores não estão em busca de empresas com modelos de negócios conservadores e com taxas de crescimento controladas e tímidas, eles estão em busca de empresas que vão obter uma taxa de crescimento muito alta e proporcionar possibilidades de saída rápido, geralmente com a venda da empresa ou por meio da abertura de capital na bolsa de valores, para que eles possam concretizar o investimento e lucrar. Para um investidor nada é pior do que ter o seu dinheiro investido em uma empresa sem prospecções de saída, mesmo que a empresa demonstre crescimento constante. Se não há como converter o investimento em dinheiro no bolso ele terá sido em vão. Um Ecossistema com diversas opções e uma quantidade alta de saídas bem sucedidas certamente atrairá muitos investidores e contribuirá para o seu crescimento. Elemento(s) relacionado(s) no arcabouço: Startup, Investimentos, Empresas Estabelecidas.

  \item [Mercado Global:] Porcentagem de Startups no Ecossistema com abrangência de mercado global. Elemento(s) relacionado(s) no arcabouço: Mercado.

  \item [Empreendedorismo nas Universidades:] Porcentagem de ex-alunos que fundaram uma empresa em até 5 anos após a graduação. Elemento(s) relacionado(s) no arcabouço: Universidades, Centros de Pesquisa e Educação.

  \item [Número de Startups:] Número de Startups em atividade por ano de acordo com fontes de dados confiáveis em um dado ano. Elemento(s) relacionado(s) no arcabouço: Mercado.

  \item [Acesso ao investimento em US\$ por ano:] Quantidade de dinheiro investido em Startups locais, em dólares americanos, de acordo com fontes confiáveis em um dado ano. Elemento(s) relacionado(s) no arcabouço: Opções de Investimento. 

  \item [Investidores Anjo: ]

  \item [Acesso ao investimento em quantidade de negócios realizados:] Contagem simples de quantos investimentos foram realizados em Startups locais, independente do valor, de acordo com fontes confiáveis em um dado ano. Elemento(s) relacionado(s) no arcabouço: Opções de Investimento.

  \item [Qualidade dos Mentores:] Um mentor de qualidade é um empreendedor experiente, alguém que já viveu os problemas que o novo empreendedor está passando e entende perfeitamente a sua situação, ninguém melhor para orienta-lo do que alguém que já passou por problemas similares ou iguais. Elemento(s) relacionado(s) no arcabouço: Empreendedor.

  \item [Burocracia:] Em sua maioria envolve o ambiente regulatório do Ecossistema Local e representa o quanto a burocrácia impacta as Startups como, por exemplo, envolvendo o tempo, custo médio e a complexidade tributária para se abrir e manter uma empresa. Elemento(s) relacionado(s) no arcabouço: Ambiente Regulatório.

  \item [Gastos com Impostos:] Baseado no ranking de impostos entre países criado por \citeonline{schwab2015}. Elemento(s) relacionado(s) no arcabouço: Ambiente Regulatório, Mercado. 

  \item [Incubadoras e Parques Tecnológicos:] Representação da quantidade de incubadoras e parques tecnológicos presentes no Ecossistema. Elemento(s) relacionado(s) no arcabouço: Incubadoras.

  \item [Qualidade das Aceleradoras:] Porcentagem das Startups que passaram por algum programa de Aceleração ou Incubação e se estabeleceram bem no mercado ou avançaram com sucesso para a fase de captação de investimento de terceiros. Elemento(s) relacionado(s) no arcabouço: Aceleradoras, Incubadoras e Parques Tecnológicos.

  \item [Presença de Empresas de Alta Tecnologia:] Quantidade de empresas de alta tecnologia presentes no Ecossistema. Elemento(s) relacionado(s) no arcabouço: Empresas Estabelecidas. (mas qual o parâmetro para considerar uma empresa de alta tecnologia? talvez uma multinacional ou empresa com valor acima de X milhões)

  \item [Influência de Empresas já estabelecidas:] A quantidade de empresas estabelecidas e engajadas em movimentar o Ecossistema por meio de eventos, líderança, mentoria e apoio, investimentos ou programas de aceleração para Startups locais. Elemento(s) relacionado(s) no arcabouço: Eventos, Empresas Estabelecidas, Aceleradoras, Empreendedores.

  \item [Qualidade do Capital Humano:] Fator baseado no índice de talentos definido por \citeonline{Hermann2015}. Elemento(s) relacionado(s) no arcabouço: Empreendedor, Educação.

  \item [Valores Culturais para o Empreendedorismo:] Fator baseado no índice de suporte cultural definido por \citeonline{Acs2016}. Elemento(s) relacionado(s) no arcabouço: Cultura, Sociedade e Família.

  \item [Processos de Transferência de Tecnologia:] Índice baseado nos fatores de Inovação e Sofisticação definidos por \citeonline{schwab2015}. Elemento(s) relacionado(s) no arcabouço: Universidades, Centros de Pesquisa e Ambiente Regulatório.

  \item [Conhecimento das Metodologias:] Porcentagem de Empreendedores que possuem conhecimento de diversas metodologias comumente utilizadas pelo mercado como Métodos Ágeis, Lean Startup, Canvas, Design Thinking, etc. Por ser um fator difícil de ser mensurado, os autores da Metodologia sugerem utilizar a quantidade de eventos relacionados no Ecossistema. Elemento(s) relacionado(s) no arcabouço: Metodologias.

  \item [Atores da Mídia com foco no Empreendedorismo:] A participação da mídia é muito importante para a promoção do Ecossistema como um todo e de seus Empreendedores, portanto a presença de profissionais engajados e que entendam o contexto do mercado local é de extrema importância. Elemento(s) relacionado(s) no arcabouço: Mídia.

  \item [Dados do Ecossistema e Pesquisas:] As universidades e os institutos de pesquisas são peças triviais em um Ecossistema de Startups, em especial por constantemente levantarem questões, respostas, informações e pontos que devem ser aprimorados em prol de um ambiente mais maduro e preparado. Também é importante que os dados sejam amplamente acessíveis, de forma que diversas peças interessadas possam ter acesso para embasarem suas ações, identificarem pontos em que podem contribuir ou atraírem mais pessoas para o Ecossistema. Elemento(s) relacionado(s) no arcabouço: Centros de Pesquisa, Governo.
  
  \item [Gerações do Ecossistema:] De tempos em tempos o Ecossistema possui uma nova leva de Empreendedores se destacando no mercado e, conforme sua maturidade aumenta, novas gerações são inspiradas, influenciadas e apoiadas pelas anteriores. Elemento(s) relacionado(s) no arcabouço: Empreendedor, Sociedade.  

  \item [Eventos relacionados à Startups:] Quantidade de eventos que acontecem na cidade com temáticas relacionadas ao desenvolvimento de competências ou troca de experiências Empreendedoras na cidade em um determinado período de tempo. Elemento(s) relacionado(s) no arcabouço: Empreendedor, Ecossistema.     
\end{description}

\chapter{Perguntas das Entrevistas}
\label{apendices:perguntas_das_entrevistas}

\begin{description}

  % Perguntas com o objetivo de quebrar o gelo e deixar tanto entrevistado quanto entrevistador mais confortáveis ! 

  \item [Pergunta 01:] Você poderia me contar um pouco da sua trajetória? Como se tornou um empreendedor? Já se envolveu com outras Startups/Empresas antes? Quais as suas motivações? 

  \item [Pergunta 02:] Falando sobre a sua Startup, o que ela faz? O que te motivou a cria-la? Em que fase está, hoje? Como foi o começo? O que mais te ajudou? O que foi mais difícil? Já tem clientes? Como foi o processo para capta-los?

  %! A partir dessa pergunta tento obter algumas informações sobre os fatores socioculturais do Ecossistema!

  \item [Pergunta 03:] Quais erros você já cometeu na sua vida empreendedora? Se pudesse voltar no tempo, o que faria de diferente? Acredita que os erros foram importantes na sua trajetória? Como as pessoas ao seu redor enxergam os erros? 

  \item [Pergunta 04:] Na sua visão, quais são as características essenciais para um Empreendedor na área de tecnologia? Você enxerga essas características nas pessoas da área de tecnologia(empreendedores, profissionais, estudantes, etc) do Distrito Federal? Quais são as principais motivações daqueles que já empreendem com Startups no DF? Dinheiro? Fama? Autoestima? Necessidade?
  
  \item [Pergunta 05:] Quais são as características de times de sucesso? Diversidade é importante? Como? Qual seria a combinação ideal (backgrounds) de um time de fundadores? Qual a sua visão sobre os times das Startups que são formadas no Distrito Federal?

  %! A partir dessa pergunta tento entender como e se o Ecossistema se conecta !

  \item [Pergunta 06:] Qual é a relação da sua Startup e a sua relação, como um Empreendedor, com o Ecossistema do Distrito Federal? Acredita que de alguma forma o Ecossistema poderia te dar suporte para os desafios que vem enfrentando no momento ou já enfrentou?

  \item [Pergunta 07:] Como os membros do Ecossistema de Startups do Distrito Federal interagem e colaboram entre si?

  \item [Pergunta 08:] Como você lida com as dificuldades técnicas e pessoais do seu time? Alguma vez o Ecossistema contribuiu com a formação e o crescimento da sua Startup ou com a resolução de problemas/desafios técnicos? Você já contribuiu ou ajudou alguma outra Startup ou Empreendedor? De forma geral, há troca de experiência entre empreendedores e empresas no Distrito Federal? 

  \item [Pergunta 09:] Como você classificaria a presença de empresas de tecnologia já consolidadas no Ecossistema? Elas de alguma forma apoiam, investem ou influenciam os que estão começando?  

  \item [Pergunta 10:] Quais são os fatores que desencorajam ou criam barreiras para o empreendedor iniciar ou chegar ao sucesso no Distrito Federal? E os que encorajam?

  %! Tentando identificar fatores educacionais que fomentam o Ecossistema !

  \item [Pergunta 11:] Qual o papel da Educação na formação do Empreendedor e no Ecossistema do Distrito Federal? Você pode indicar iniciativas educacionais que alimentam ou nutrem o espírito empreendedor nos brasilienses? Quais elementos poderiam ser melhorados na formação educacional dos jovens com objetivo de fomentar o empreendedorismo no Distrito Federal? Para você, houve algum momento específico na sua formação que foi essencial para a sua formação como Empreendedor?

  %! O Fernando Nandico, um dos empreendedores mais experientes do Distrito Federal, relatou de que não considera saudável criar Startups utilizando linguagens modernas em Brasília, como Rails, pela falta de profissionais capacitados. Aqui, segundo ele, é mais interessante trabalhar com Java, PHP, etc. Quero captar a visão de outros players com essas duas perguntas!

  \item [Pergunta 12:] Como aspectos tecnológicos como linguagens de programação, frameworks, software livre, etc influenciam no sucesso ou fracasso das Startups no Distrito Federal? Como esses fatores no contexto do Distrito Federal se comparam com a realidade de outros Ecossistemas? 

  \item [Pergunta 13:] Qual o nível de qualidade dos profissionais da área de Tecnologia do Distrito Federal? Você possui dificuldade para atrai-los? Acredita que algo poderia poderia ser feito para melhorar a oferta e a qualidade de profissionais?
  
  \item [Pergunta 14:] Como aspectos metodológicos(ágeis, lean startup, customer development, canvas, etc) influenciam no sucesso ou fracasso das Startups do Distrito Federal? Quais práticas vocês utilizam? Como elas impactaram seus negócios? Há algo que não funcionou bem? Como esses fatores no contexto do Distrito Federal se comparam com a realidade de outros Ecossistemas?

  %! Tentando entender os mecânismos institucionais do Distrito Federal !

  \item [Pergunta 15:] Que ações em relação ao Ambiente Regulatório você acredita que deveriam ser tomadas para apoiar o empreendedor do Distrito Federal?

  \item [Pergunta 16:] Há algum mecânismo institucional no Distrito Federal que promove o empreendedorismo? Legislações, ações de universidades, agências e programas do governo, fundos de investimento, ONGs, etc. Você se beneficiou por algum deles? Como os classifica? Algo que poderia ser aprimorado? Considera o governo local como um apoiador do Empreendedorismo?

  \item [Pergunta 17:] Quais fontes de capital estão disponíveis no Distrito Federal? Como você classifica a presença e as ações de investidores, aceleradoras e incubadoras no Distrito Federal? Já se relacionou com algum? Como foi a experiência?

  %! Fechamento da entrevista !
  
  \item [Pergunta 18:] Quais são os elementos chave para um ecossistema de Startups vibrante e saudável? Como você descreveria e classificaria o nosso Ecossistema? Quais os nossos pontos fortes e fracos? Algum Ecossistema ao redor do mundo que seja similar ao nosso?

  \item [Pergunta 19:] E o que tem sido feito no Distrito Federal para estimular o Ecossistema de Startups? O que mais precisa ser feito?
\end{description}

\chapter{Tabelas de Entrevistados}
\label{apendices:tabelas_de_entrevistados}

\begin{table}[!htb]
	\centering
	\begin{tabular}{ | p{3cm} | p{8cm} | p{4cm} | }
		\hline
		Categoria & Nome & Empresa \\ \hline
		Empreendedor & Alexandre Gomes & SEA \\ \hline
		Empreendedor & André Eloy & São \\ \hline
		Empreendedor & André Macedo & ZeroPaper \\ \hline
		Empreendedor & Arthur Furlan & Configr \\ \hline
		Empreendedor & Bruno Kenj & Owl Docs \\ \hline
		Empreendedor & Bruno Rossi & Apetitar \\ \hline
		Empreendedor & Bruno Torquato & PDVend \\ \hline
		Empreendedor & Daniel Bordin & Mirante  \\ \hline
		Empreendedor & Daniel Sandoval & Loop \\ \hline
		Empreendedor & Fabrício Buzeto & Buzeto Tecnologia \\ \hline
		Empreendedor & Fernando Aquino & MovaMais \\ \hline
		Empreendedor & Flávio & Startaê \\ \hline	
		Empreendedor & Flávio Fonseca & Novatics \\ \hline
		Empreendedor & Gustavo Goreinstein & Poup \\ \hline	
		Empreendedor & Henrique Santana & Integrah \\ \hline
		Empreendedor & Iuri Costa & Axies \\ \hline
		Empreendedor & Jéssica Behrens & Tradr \\ \hline
		Empreendedor & Joaquim Venâncio & Ticies \\ \hline
		Empreendedor & Luis Sampaio & Preditiva  \\ \hline
		Empreendedor & Marcos Beto & Urbanizo \\ \hline
		Empreendedor & Marcos Nascimento & Izie \\ \hline
		Empreendedor & Maximiliano ou Jens & WriteWork.com \\ \hline
		Empreendedor & Michele Protzek & Flama \\ \hline
		Empreendedor & Pedro Salum & Loop \\ \hline
		Empreendedor & Rafael & Startaê \\ \hline	
		Empreendedor & Renato & Startaê \\ \hline	
		Empreendedor & Ricardo & Funnifier \\ \hline
		Empreendedor & Roberto Mascarenhas & IPê \\ \hline
		Empreendedor & Saulo Camarotti & Behold Studios \\ \hline	
	\end{tabular}
	\caption{Mapeamento de Empreendedores para serem entrevistados}
	\label{table:sugestao_de_empreendedores_para_entrevista}
\end{table}


\begin{table}[!htb]
	\centering
	\begin{tabular}{ | p{3cm} | p{8cm} | p{4cm} | }
		\hline
		Categoria & Nome & Empresa \\ \hline
		CW/Incu & Cristiane Pereira & Multiplicidade \\ \hline
		CW/Incu & Fernando Santiago & 4Legal \\ \hline
		CW/Incu & Heloísa & 4Legal \\ \hline
		CW/Incu & Juliana Guimarães & 4Legal \\ \hline
	\end{tabular}
	\caption{Mapeamento de Coworkings/Incubadoras/Aceleradoras para serem entrevistados}
	\label{table:sugestao_de_coworkings_para_entrevista}
\end{table}

\begin{table}[!htb]
	\centering
	\begin{tabular}{ | p{3cm} | p{8cm} | p{4cm} | }
		\hline
		Categoria & Nome & Empresa \\ \hline
		Governo & Cris Vieira & SEBRAE \\ \hline
		Governo & Marcio Brito & SEBRAE \\ \hline
		Governo & Marcos Vinicius & MDIC \\ \hline
		Governo & Manoel & FAPDF \\ \hline
		Governo & Sheila Oliveira Pires & ANPROTEC \\ \hline
		Governo & Thiago Jarjour & GDF \\ \hline
		Associacao & Hugo Giallanza & ASTEPS \\ \hline
		Associacao & Antonio Ventura & ASTEPS \\ \hline

	\end{tabular}
	\caption{Mapeamento de Instituições de Apoio para serem entrevistados}
	\label{table:sugestao_de_instituicoes_para_entrevista}
\end{table}

\begin{table}[!htb]
	\centering
	\begin{tabular}{ | p{3cm} | p{8cm} | p{4cm} | }
		\hline
		Categoria & Nome & Empresa \\ \hline
		Investidor & Carlos Augusto Ferraz & Anjos do Brasil \\ \hline
		Investidor & Bruno Brito & Cedro Capital \\ \hline
		Investidor & Rafael Moraes & Garan Ventures \\ \hline
		Aceleradora & Wesley Almeida & Cotidiano \\ \hline
		Aceleradora & Mariana & Techmall \\ \hline
		Aceleradora & Hélio & Acceleratus \\ \hline	
	\end{tabular}
	\caption{Mapeamento de Investidores para serem entrevistados}
	\label{table:sugestao_de_investidores_para_entrevista}
\end{table}

\begin{table}[!htb]
	\centering
	\begin{tabular}{ | p{3cm} | p{8cm} | p{4cm} | }
		\hline
		Categoria & Nome & Empresa \\ \hline
		Universidade & Cristina Castro & UnB \\ \hline
		Universidade & Érika Lisboa & UniCeub \\ \hline
		Universidade & Fabricio Costa & Catolica \\ \hline
		Universidade & Gabriel Cardoso & UDF \\ \hline
		Universidade & Jonathan Medeiros & Marco Zero \\ \hline		
		Universidade & Victor Medeiros & Concentro \\ \hline
	\end{tabular}
	\caption{Mapeamento de Representantes das Universidades para serem entrevistados}
	\label{table:sugestao_de_universidades_para_entrevista}
\end{table}

\end{apendicesenv}
