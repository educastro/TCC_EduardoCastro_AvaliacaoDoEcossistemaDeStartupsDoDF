\chapter[Metodologia]{Metodologia}
\label{cap-metodologia}

Para cumprir com o objetivo de realizar uma Avaliação do Ecossistema de Startups de Tecnologia do Distrito Federal foi adotada a Metodologia criada por \citeonline{kon2015} como um dos primeiros frutos do Grupo de Pesquisa em Empreendedorismo InovaSampa. 

Este capítulo tem como objetivo descrever diversos pontos explorados por eles e as adaptações que foram feitas para obter uma visão mais adequada com base nos dados disponíveis. A escolha se deu por recomendação do Orientador deste trabalho, Professor Paulo Meirelles, e por acreditar que ela fornece um bom caminho para obtermos uma visão geral e realista do atual estado do Ecossistema local por ter uma forte integração com empreendedores locais e já ter sido executada em três cidades do mundo: Tel-Aviv \cite{kon2014}, São Paulo \cite{monna2015} e Nova Iorque \cite{cukier2016}. 

\section{A Metodologia do InovaSampa}
\label{section:metodologia_do_inovasampa}

\subsection{Desvantagens e Vantagens}
\label{subsection:vantagens_e_desvantagens}

Até a publicação deste trabalho as três cidades que foram objetos de estudo do InovaSampa, e que contribuiram para o desenvolvimento de sua Metodologia, são conhecidas globalmente como grandes centros empreendedores e dispõem de uma vasta quantidade de dados nacionais e globais disponíveis, o que não condiz com a realidade de Ecossistemas menores, como o do Distrito Federal, e torna alguns dos fatores ou medidas sugeridos fora de contexto com o que temos disponível.

Para contornar esse problema, e trazer melhorias à Metodologia para que ela possa ser melhor utilizada no contexto de outras cidades brasileiras que também carecem de dados e querem conhecer e desenvolver seus Ecossistemas, alguns fatores foram excluídos ou adaptados para se adaptarem às informações disponíveis. Tudo está descrito na sessão \ref{subsection:fatores_que_formam_um_ecossistema}.

A maior vantagem da Metodologia proposta se dá por ter, como sua maior base, dados obtidos a partir das visões daqueles que melhor o entendem e lidam com o Ecossistema de Startups local - os próprios empreendedores. Ao permitir que os fatores sejam mensurados de acordo com esses dados torna-se possível visualizar quais são as características mais fortes e fracas do Ecossistema em estudo como um todo e compara-los direta ou indiretamente com outros Ecossistemas.

\subsection{Técnicas Utilizadas}
\label{subsection:tecnicas_utilizadas}

Toda a metodologia foi construida com base nas técnicas de Pesquisa Qualitativa e na Teoria Fundamentada em Dados por oferecerem a possibilidade de obter dados a partir das visões daqueles que melhor o entendem e lidam com o Ecossistema de Startups local e seus pontos fortes e fracos todos os dias - os próprios empreendedores - por meio de entrevistas, dessa forma é possível valorizar e obter respostas a partir de suas experiências individuais.

! TO-DO: estudar sobre pesquisa qualitativa e teoria fundamentada em dados para embasar melhor essa subseção !

\subsection{Fatores que formam um Ecossistema}
\label{subsection:fatores_que_formam_um_ecossistema}

Após vasta pesquisa bibliográfica e entrevistas com mais de 50 pessoas chaves para os Ecossistemas de Tel-Aviv e São Paulo foram definidos cerca de 21 fatores que os compõem e fazem parte do Arcabouço Teórico de um Ecossistema, descrito na subseção \ref{subsection:arcabouco_conceitual_e_modelo}. 

\begin{description}

  \item [Estratégias de Saída:] Quando falamos de Estratégias de Saída falamos de formas de transformar uma empresa em capital, em converter ações em dinheiro real. Investidores não estão em busca de empresas com modelos de negócios conservadores e com taxas de crescimento controladas e tímidas, eles estão em busca de empresas que vão obter uma taxa de crescimento muito alta e proporcionar possibilidades de saída rápido, geralmente com a venda da empresa ou por meio da abertura de capital na bolsa de valores, para que eles possam concretizar o investimento e lucrar. Para um investidor nada é pior do que ter o seu dinheiro investido em uma empresa sem prospecções de saída, mesmo que a empresa demonstre crescimento constante. Se não há como converter o investimento em dinheiro no bolso ele terá sido em vão. Um Ecossistema com diversas opções e uma quantidade alta de saídas bem sucedidas certamente atrairá muitos investidores e contribuirá para o seu crescimento. Elemento(s) relacionado(s) no arcabouço: Startup, Investimentos, Empresas Estabelecidas.

  \item [Mercado Global:] Porcentagem de Startups no Ecossistema com abrangência de mercado global. Elemento(s) relacionado(s) no arcabouço: Mercado.

  \item [Influência Militar nas Tecnologias: ] O quanto o setor Militar influencia no desenvolvimento de novas tecnologias no Ecossistema. Elemento(s) relacionado(s) no arcabouço: Centros de PEsquisa.

  \item [Empreendedorismo nas Universidades:] Porcentagem de ex-alunos que fundaram uma empresa em até 5 anos após a graduação. Elemento(s) relacionado(s) no arcabouço: Universidades, Centros de Pesquisa e Educação.

  \item [Número de Startups:] Número de Startups em atividade por ano de acordo com fontes de dados confiáveis em um dado ano. Elemento(s) relacionado(s) no arcabouço: Mercado.

  \item [Acesso à investimento em US\$ por ano:] Quantidade de dinheiro investido em Startups locais, em dólares americanos, de acordo com fontes confiáveis em um dado ano. Elemento(s) relacionado(s) no arcabouço: Opções de Investimento. 

  \item [Acesso à investimento em quantidade de negócios realizados:] Contagem simples de quantos investimentos foram realizados em Startups locais, independente do valor, de acordo com fontes confiáveis em um dado ano. Elemento(s) relacionado(s) no arcabouço: Opções de Investimento.

  \item [Qualidade dos Mentores:] Um mentor de qualidade é um empreendedor experiente, alguém que já viveu os problemas que o novo empreendedor está passando e entende perfeitamente a sua situação, ninguém melhor para orienta-lo do que alguém que já passou por problemas similares ou iguais. Elemento(s) relacionado(s) no arcabouço: Empreendedor.

  \item [Burocracia:] Em sua maioria envolve o ambiente regulatório do Ecossistema Local e representa o quanto a burocrácia impacta as Startups como, por exemplo, envolvendo o tempo, custo médio e a complexidade tributária para se abrir e manter uma empresa. Elemento(s) relacionado(s) no arcabouço: Ambiente Regulatório.

  \item [Gastos com Impostos:] Baseado no ranking de impostos entre países criado por \citeonline{schwab2015}. Elemento(s) relacionado(s) no arcabouço: Ambiente Regulatório, Mercado. 

  \item [Incubadoras, Aceleradoras e Parques Tecnológicos:] Representação da quantidade de incubadoras e parques tecnológicos presentes no Ecossistema. Elemento(s) relacionado(s) no arcabouço: Incubadoras, Aceleradoras.

  \item [Qualidade das Aceleradoras:] Porcentagem das Startups que passaram por algum programa de Aceleração ou Incubação e se estabeleceram bem no mercado ou avançaram com sucesso para a fase de captação de investimento de terceiros. Elemento(s) relacionado(s) no arcabouço: Aceleradoras, Incubadoras e Parques Tecnológicos.

  \item [Presença de Empresas de Alta Tecnologia:] Quantidade de empresas de alta tecnologia presentes no Ecossistema. Elemento(s) relacionado(s) no arcabouço: Empresas Estabelecidas. (mas qual o parâmetro para considerar uma empresa de alta tecnologia? talvez uma multinacional ou empresa com valor acima de X milhões)

  \item [Influência de Empresas já estabelecidas:] A quantidade de empresas estabelecidas e engajadas em movimentar o Ecossistema por meio de eventos, líderança, mentoria e apoio, investimentos ou programas de aceleração para Startups locais. Elemento(s) relacionado(s) no arcabouço: Eventos, Empresas Estabelecidas, Aceleradoras, Empreendedores.

  \item [Qualidade do Capital Humano:] Fator baseado no índice de talentos definido por \citeonline{compass2015}. Elemento(s) relacionado(s) no arcabouço: Empreendedor, Educação.

  \item [Valores Culturais para o Empreendedorismo:] Fator baseado no índice de suporte cultural definido por \citeonline{gedi2016}. Elemento(s) relacionado(s) no arcabouço: Cultura, Sociedade e Família.

  \item [Processos de Transferência de Tecnologia:] Índice baseado nos fatores de Inovação e Sofisticação definidos por \citeonline{schwab2015}. Elemento(s) relacionado(s) no arcabouço: Universidades, Centros de Pesquisa e Ambiente Regulatório.

  \item [Conhecimento das Metodologias:] Porcentagem de Empreendedores que possuem conhecimento de diversas metodologias comumente utilizadas pelo mercado como Métodos Ágeis, Lean Startup, Canvas, Design Thinking, etc. Por ser um fator difícil de ser mensurado, os autores da Metodologia sugerem utilizar a quantidade de eventos relacionados no Ecossistema. Elemento(s) relacionado(s) no arcabouço: Metodologias.

  \item [Mídia Especializada:] A participação da mídia é muito importante para a promoção do Ecossistema como um todo e de seus Empreendedores, portanto a presença de profissionais engajados e que entendam o contexto do mercado local é de extrema importância. Elemento(s) relacionado(s) no arcabouço: Mídia.

  \item [Dados do Ecossistema e Pesquisas:] As universidades e os institutos de pesquisas são peças triviais em um Ecossistema de Startups, em especial por constantemente levantarem questões, respostas, informações e pontos que devem ser aprimorados em prol de um ambiente mais maduro e preparado. Também é importante que os dados sejam amplamente acessíveis, de forma que diversas peças interessadas possam ter acesso para embasarem suas ações, identificarem pontos em que podem contribuir ou atraírem mais pessoas para o Ecossistema. Elemento(s) relacionado(s) no arcabouço: Centros de Pesquisa, Governo.
  
  \item [Gerações do Ecossistema:] De tempos em tempos o Ecossistema possui uma nova leva de Empreendedores se destacando no mercado e, conforme sua maturidade aumenta, novas gerações são inspiradas, influenciadas e apoiadas pelas anteriores. Elemento(s) relacionado(s) no arcabouço: Empreendedor, Sociedade.       
\end{description}

Com o objetivo de classifica-los entre níveis para facilitar as comparações e o cálculo final da maturidade do Ecossistema foram definidas as seguintes métricas para cada um dos fatores representados na Tabela \ref{tabela:metricas_de_classificacao_dos_fatores}, vale ressaltar que os fatores que contém o símbolo \"*\" antes de seu nome são os fatores essenciais, os restantes são os fatores derivados.

\begin{table}[!htb]
\centering
\caption{Métricas de classificação dos Fatores que compõem um Ecossistema}
\label{tabela:metricas_de_classificacao_dos_fatores}
\begin{tabular}{llll}
Fator                                                      &     L1     &     L2     &     L3   \\
Estratégias de Saída*                                      &     00     &     01     &    >=2   \\
Mercado Global*                                            &    <10\%   &   10-50\%  &    >50\% \\
Influência Militar nas Tecnologias                         &    <10\%   &   10-50\%  &   >=50\% \\
Empreendedorismo nas Universidades                         &    <02\%   &   02-10\%  &    >10\% \\
Número de Startups*                                        &   <500k    &   500-3k   &    >3k   \\
Acesso à investimento em quantidade de negócios realizados*&   200     &   200-1000 &    >1000 \\
Qualidade de Mentores                                      &    <10\%   &   10-50\%  &    >50\% \\
Burocracia                                                 &    >40\%   &   10-40\%  &    <10\% \\
Gastos com impostos                                        &    >50\%   &   30-50\%  &    <30\% \\
Incubadoras, Aceleradoras e Parques Tecnológicos           &     02     &    02-10   &    >10   \\
Qualidade das Aceleradoras                                 &    <10\%   &   10-50\%  &    >50\% \\
Presença de Empresas de Alta Tecnologia*                   &    <10     &   10-50    &    >50   \\
Influência de Empresas já estabelecidas                    &    <20     &   20-80    &    >80\% \\
Qualidade do Capital Humano*                               &    >20th   &   15-20th  &    <15th \\
Valores Culturais para o Empreendedorismo*                 &    <0.5    &   0.5-0.75 &    >0.75 \\
Processos de Transferência de Tecnologia                   &    <4.0    &   4.0-5.0  &    >5.0  \\
Conhecimento das Metodologias                              &    20\%    &   20-60\%  &    >60\% \\
Mídia Especializada                                        &    <03     &   03-05    &    > 05  \\
Dados do Ecossistema e Pesquisas*                          &    nada    & parcial &disponíveis  \\
Gerações do Ecossistema*                                   &     00     &    0.1     &    02    \\
\end{tabular}
\end{table}

\subsection{O arcabouço conceitual e o Mapa de um Ecossistema}
\label{subsection:arcabouco_conceitual_e_modelo}

Com base nesses mesmos fatores descritos na subseção \ref{subsection:fatores_que_formam_um_ecossistema} e na relevância de cada um deles de acordo com a visão das pessoas que compõem o próprio Ecossistema e nas informações disponibilizadas por outras pesquisadores ou bases de dados foi elaborado um arcabouço conceitual de um Ecossistema, representado pela Figura 01. Na Figura 02 está representado o Mapa do Ecossistema de Tel-Aviv, Israel, e na Figura 03 o Mapa do Ecossistema de São Paulo, ambos tendo como base o mesmo arcabouço conceitual.

\begin{figure}[!htb]
\centering
\includegraphics[width=11cm,angle=0]{figuras/arcabouco_teorico_de_um_ecossistema}
\caption{Arcabouço Conceitual de um Ecossistema de Startups}
\label{Rotulo}
\end{figure}

\begin{figure}[!htbp]
\centering
\includegraphics[width=13cm,angle=0]{figuras/mapa_conceitual_tel_aviv}
\caption{Mapa do Ecossistema de Tel-Aviv, Israel}
\label{Rotulo}
\end{figure}

\begin{figure}[!htbp]
\centering
\includegraphics[width=13cm,angle=0]{figuras/mapa_conceitual_sao_paulo}
\caption{Mapa do Ecossistema de Tel-Aviv, Israel}
\label{Rotulo}
\end{figure}

\subsection{Os níveis de maturidade de um Ecossistema}
\label{subsection:niveis_de_maturidade_de_um_ecossistema}

Além de elaborar o mapa do ecossistema a Metodologia tem como um dos seus objetivos classificar Ecossistemas entre quatro diferentes níveis de maturidade. Os níveis são os seguintes:

\begin{description}
  \item [Nascente (M1):] quando há um Ecossistema com algumas Startups presentes no mercado, alguns investimentos concretizados e algumas iniciativas com o objetivo de estimular ou fomentar o Ecossistema sendo realizadas mas não há reconhecimento ou as Startups não possuem representatividade nos índices de geração de emprego e renda da região.

  \item [Crescente (M2):] quando há algumas Startups estabelecidas como empresas sólidas e o Ecossistema como um todo possui representatividade notável na economia regional e nos índices de empregos. Para se enquadrar como Crescente todos fatores essenciais e cerca de 30\% dos fatores derivados deverão ser classificadas como nível L2.

  \item [Maduro (M3):] quando existem algumas centenas de Startups em atividade, sendo algumas reconhecidas internacionalmente e com negócios realizados globalmente, um histórico relevante de investimentos concretizados dentro do Ecossistema e pelo menos uma geração de empreendedores bem sucedidos que se tornaram líderes, mentores, referências e investidores-anjo para os novos empreendedores, ajudando-os a crescer. Além dessas características, para ser considerado como um Ecossistema Maduro, todos os fatores essenciais e pelo menos 50\% dos fatores derivados devem ser classificadas como nível L2 e, no mínimo, 30\% de todos os fatores devem estar enquadrados no nível L3.

  \item [Sustentável (M4):] quando o número de Startups em atividade e de aquisições e/ou investimentos dentro do Ecossistema ultrapassam a casa dos milhares, há no mínimo duas gerações de empreendedores bem sucedidos que iniciaram suas carreiras com Startups de tecnologia presentes, uma rede de empreendedores comprometidos com o desenvolvimento do Ecossistema à longo prazo, um ambiente inclusivo com muitos eventos envolvendo temáticas que fomentem a cultura empreendedora e o mercado local e a presença de uma alta quantidade de profissionais de alta qualidade técnica. Para possuir esse estágio de maturidade, todos os fatores essenciais devem ser classificados como nível L3 e pelos menos 80\% dos fatores derivados também como nível L3.
\end{description}

\section{Aplicação da Metodologia e Protocolo}
\label{section:aplicacao_da_metodologia}

A aplicação da Metodologia foi dividida em três etapas:

\begin{enumerate}
  \item Entrevistas e Observações
  \item Codificação dos Dados
  \item Análises e Conclusões
\end{enumerate}

A primeira, de Entrevistas e Observações, se deu por meio da observação de diversos eventos que movimentam o Ecossistema de Startups do Distrito Federal e das pessoas que o compõem e por meio de entrevistas individuais com pessoas atuantes no Ecossistema com objetivo de entender seu contexto pessoal e profissional, bem como suas visões sobre a realidade e as dinâmicas do Ecossistema como um todo, quais os seus pontos fortes e fracos, seus maiores problemas, como diversas instituições e pessoas interagem entre si afim de fomenta-lo e quais ações poderiam ser tomadas afim de melhora-lo.

Com a Codificação dos Dados todas as informações levantadas pela primeira etapa foram catalogadas em tabelas com o objetivo de se tornarem referências para as etapas de Análises e Conclusões e futuras pesquisas bem como documentar todo o processo que foi realizado. 

Com as Análises dos dados e sua adequação nos fatores pré-definidos será possível mensurar a maturidade do Ecossistema com o objetivo de gerar as Conclusões da pesquisa, que se concentrarão em explicitar o atual estágio do Ecossistema de acordo com a Metodologia utilizada, em realizar comparações com outros Ecossistemas e identificar uma série de ações que podem ser tomadas para aprimorar determinados pontos.

\subsection{Escolha dos Entrevistados}
\label{subsection:escolha_dos_entrevistados}

É de extrema importância que os entrevistados sejam pessoas atuantes e bem conectados com o Ecossistema de Startups do Distrito Federal como um todo e, em sua maior parte, Empreendedores mas também Professores, Servidores Públicos, Investidores, Representantes de Incubadoras e Aceleradoras e Estudantes. 

Para a escolha dos Entrevistados fora aplicada a metodologia bola de neve. Primeiramente foram definidos algumas pessoas com alto histórico de contribuição ao Ecossistema e que faziam parte da rede de contatos das pessoas envolvidas com a Pesquisa e ao fim de cada entrevista ou conversa informal foram pedidas recomendações de quais pessoas deveriam fazer parte deste pesquisa por serem referências para o Ecossistema, bem como, se possível, solicitado uma introdução entre essas pessoas. 

A meta é que sejam entrevistados cerca de 30 pessoas divididos entre líderes e fomentadores do Ecossistema, empreendedores, membros da comunidade acadêmica, representantes de incubadoras e aceleradoras, investidores e agentes públicos envolvidos com políticas públicas de fomento ao empreendedorismo. 

Até o momento a listagens de pessoas a serem convidadas para a entrevista e justificativa pode ser encontrada em: http://bit.ly/1X9Y33K

\subsection{Condução das Entrevistas}
\label{subsection:conducao_das_entrevistas}

Todas as entrevistas devem ser realizadas, preferencialmente, no ambiente profissional dos Empreendedores de forma a mantê-los à vontade. Caso não seja possível, ela poderá ser conduzida em ambiente escolhido pelo empreendedor, como bibliotecas, cafeterias ou eventos e apenas em último caso de forma remota. Elas também serão gravadas em aúdio caso haja consentimento do empreendedor afim de facilitar a fase de Codificação dos Dados.

As entrevistas não devem ser muito longas, preferencialmente não sendo extendidas por mais de uma hora e meia. Para guiar o Entrevistador foram estabelecidas uma série de Perguntas que devem ser realizadas aos Entrevistados com o objetivo de obter respostas que respondam às Questões de Pesquisa estabelecidas e nos forneçam uma visão geral do Ecossistema. As Questões de Pesquisa foram as mesmas definidas pelo Professor Fabio Kon mas as Perguntas foram adaptadas à realidade do Distrito Federal.

Não necessariamenteas entrevistas devem seguir de forma rígida todas as perguntas definidas no roteiro, o entrevistador poderá ter liberdade de conduzi-la como bem entender. Como a entrevista será conduzida ou a linguagem utilizada não é de grande importância, desde que a maior parte das questões sejam respondidas, mesmo que de forma indireta. Há a possibilidade de que o próprio entrevistado responda algumas delas durante outras perguntas. 

\subsection{Roteiro de Entrevista e Perguntas}
\label{subsection:roteiro_de_entrevista_e_perguntas}

\begin{description}

  % Perguntas com o objetivo de quebrar o gelo e deixar tanto entrevistado quanto entrevistador mais confortáveis ! 

  \item [Pergunta 01:] Você poderia me contar um pouco da sua trajetória? Como se tornou um empreendedor? Já se envolveu com outras Startups/Empresas antes? Quais as suas motivações? 

  \item [Pergunta 02:] Falando sobre a sua Startup, o que ela faz? O que te motivou a cria-la? Em que fase está, hoje? Como foi o começo? O que mais te ajudou? O que foi mais difícil? Já tem clientes? Como foi o processo para capta-los?

  %! A partir dessa pergunta tento obter algumas informações sobre os fatores socioculturais do Ecossistema!

  \item [Pergunta 03:] Quais erros você já cometeu na sua vida empreendedora? Se pudesse voltar no tempo, o que faria de diferente? Acredita que os erros foram importantes para a sua formação? Como as pessoas ao seu redor enxergam os erros? 

  \item [Pergunta 04:] Na sua visão, quais são as características essenciais para um Empreendedor na área de tecnologia? Você enxerga essas características nas pessoas da área de tecnologia(empreendedores, profissionais, estudantes, etc) do Distrito Federal? Quais são as principais motivações daqueles que já empreendem com Startups no DF? Dinheiro? Fama? Autoestima? Necessidade?

  \item [Pergunta 05:] Quais são as características de times de sucesso? Diversidade é importante? Como? Qual seria a combinação ideal (backgrounds) de um time de fundadores? Qual a sua visão sobre os times das Startups que são formadas no Distrito Federal?
  
  %! A partir dessa pergunta tento entender como e se o Ecossistema se conecta !

  \item [Pergunta 06:] Qual é a relação da sua Startup e a sua relação, como um Empreendedor, com o Ecossistema do Distrito Federal? Acredita que de alguma forma o Ecossistema poderia te dar suporte para os desafios que vem enfrentando no momento ou já enfrentou?

  \item [Pergunta 07:] Existem empresas ou empreendedores no Distrito Federal mais experientes que te inspiraram ou te ajudaram a empreender? Quais? 

  \item [Pergunta 08:] Como você classificaria a presença de empresas de tecnologia já consolidadas no Ecossistema? Elas de alguma forma apoiam, investem ou influenciam os que estão começando?  
  
  \item [Pergunta 09:] Como você lida com as dificuldades técnicas e pessoais do seu time? Alguma vez o Ecossistema contribuiu com a formação e o crescimento da sua Startup ou com a resolução de problemas/desafios técnicos? Você já contribuiu ou ajudou alguma outra Startup ou Empreendedor? De forma geral, há troca de experiência entre empreendedores e empresas no Distrito Federal? 

  \item [Pergunta 10:] Quais são os fatores que desencorajam ou criam barreiras para o empreendedor iniciar ou chegar ao sucesso no Distrito Federal? E os que encorajam?

  %! Tentando identificar fatores educacionais que fomentam o Ecossistema !

  \item [Pergunta 11:] Qual o papel da Educação na formação do Empreendedor e no Ecossistema do Distrito Federal? Você pode indicar iniciativas educacionais que alimentam ou nutrem o espírito empreendedor nos brasilienses? Quais elementos poderiam ser melhorados na formação educacional dos jovens com objetivo de fomentar o empreendedorismo no Distrito Federal? Para você, houve algum momento específico na sua formação que foi essencial para a sua formação como Empreendedor?

  %! O Fernando Nandico, um dos empreendedores mais experientes do Distrito Federal, relatou de que não considera saudável criar Startups utilizando linguagens modernas em Brasília, como Rails, pela falta de profissionais capacitados. Aqui, segundo ele, é mais interessante trabalhar com Java, PHP, etc. Quero captar a visão de outros players com essas duas perguntas!

  \item [Pergunta 12:] Como aspectos tecnológicos como linguagens de programação, frameworks, software livre, etc influenciam no sucesso ou fracasso das Startups no Distrito Federal? Como esses fatores no contexto do Distrito Federal se comparam com a realidade de outros Ecossistemas? 

  \item [Pergunta 13:] Você possui dificuldade para atrair profissionais para trabalhar na sua Startup? O que poderia ser feito para melhorar a oferta e a qualidade de profissionais?
  
  \item [Pergunta 14:] Como aspectos metodológicos(ágeis, lean startup, customer development, canvas, etc) influenciam no sucesso ou fracasso das Startups do Distrito Federal? Quais práticas vocês utilizam? Como elas impactaram seus negócios? Há algo que não funcionou bem? Como esses fatores no contexto do Distrito Federal se comparam com a realidade de outros Ecossistemas?

  \item [Pergunta 15:] O que você acredita que é mais importante para o sucesso de uma Startup: um planejamento sistêmico ou tomada de decisões baseadas na intuição dos membros? Como você acha que esses dois fatores se relacionam durante o ciclo de vida de uma Startup? 

  %! Tentando entender os mecânismos institucionais do Distrito Federal !

  \item [Pergunta 16:] Que ações em relação ao Ambiente Regulatório você acredita que deveriam ser tomadas para apoiar o empreendedor do Distrito Federal?

  \item [Pergunta 17:] Há algum mecânismo institucional no Distrito Federal que promove o empreendedorismo? Legislações, ações de universidades, agências e programas do governo, fundos de investimento, ONGs, etc. Você se beneficiou por algum deles? Como os classifica? Algo que poderia ser aprimorado? Considera o governo local como um apoiador do Empreendedorismo?

  \item [Pergunta 18:] Como você classifica a presença e as ações de investidores, aceleradoras e incubadoras no Distrito Federal? Já se relacionou com algum? Como foi a experiência?

  \item [Pergunta 19:] Como você classificaria as oportunidades de exit para Startups/Empreendedores/Investidores no Distrito Federal? 

  %! Fechamento da entrevista !
  
  \item [Pergunta 20:] Na sua opinião, quais são os elementos chave para um ecossistema de Startups vibrante e saudável? Eles existem no Distrito Federal? Se não, porque? Como você descreveria e classificaria o nosso Ecossistema? Quais os nossos pontos fortes e fracos?
\end{description}

\subsection{Questões de Pesquisa}
\label{subsection:questoes_de_pesquisa}

As perguntas da subseção anterior visam responder as seguintes Questões que representam o objetivo final deste trabalho como um primeiro passo para conhecer, mapear e mensurar o Ecossistema de Startups do Distrito Federal.

\begin{itemize}
  \item Questão de Pesquisa 1: Quais são as características socioculturais do Distrito Federal que promovem ou inibem o espirito empreendedor?
  \item Questão de Pesquisa 2: Quais são os mecânismos institucionais do Distrito Federal que promovem ou dificultam o Empreendedorismo?
  \item Questão de Pesquisa 3: Quais são os mecânismos educacionais do Distrito Federal que promovem o Empreendedorismo?
  \item Questão de Pesquisa 4: Como os fatores tecnológicos influenciam o sucesso ou fracasso das Startups do Distrito Federal? Qual o papel executado pela comunidade e pelo Software Livre?
  \item Questão de Pesquisa 5: Qual a relação do empreendedor do Distrito Federal com as opções de investimento disponíveis e como elas influenciam o Ecossistema?
  \item Questão de Pesquisa 6: Quais ações devem ser tomadas no Ecossistema do Distrito Federal para que ele cresça? 

  ! Me pergunto se já há muitas questões e talvez as duas seguintes possam ser cortadas !
  \item Questão de Pesquisa 7: Quais são os aspectos metodológicos que influenciam o sucesso ou fracassos das Startups? Como? Qual o nível de adoção de métodos de desenvolvimento de negócios e de gerenciamento de equipes como Scrum, Startup Enxuta, etc no Ecossistema? Essa relação muda conforme amadurecimento das Startups?
  \item Questão de Pesquisa 8: Quais são as características de times inovadores e empreendedores de sucesso no Ecossistema do Distrito Federal? Qual é a principal motivação desse Empreendedor?


\end{itemize}

\subsection{Codificação e Interpretação dos Dados}
\label{subsection:codificacao_e_interpretacao_dos_dados}

Após a realização de cada Entrevista a codificação dos dados será feita utilizando um software CAQDAS(Computer-Assisted/Aided Qualitative Data Analysis Software), preferencialmente de Código Aberto e Gratuito, com o objetivo de facilitarem o trabalho de transcrição e análise de conteúdo. 

! RQDA e MaxQDA parecem ser os mais adequados, porém o segundo é pago ! 
! TO-DO: Estudar Miles Huberman sobre fases de análises qualitativas de dados !