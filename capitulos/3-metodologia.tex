/ \chapter[Metodologia]{Metodologia}
\label{cap-metodologia}

Este capítulo aborda sobre a Metodologia de Pesquisa utilizada para a Avaliação da Maturidade do Ecossistema do Distrito Federal com base no modelo desenvolvido por \citeonline{kon2015} na Universidade de São Paulo.

\section{Trabalhos Relacionados}
\label{section:trabalhos_relacionados}

\section{Metodologia de Avaliação adotada}
\label{section:framework_adotado}

\citeonline{kon2015} criaram uma metodologia de avaliação que não tem como única utilidade a definição do nível de maturidade de um determinado Ecossistema de Startups com base em um arcabouço estabelecido mas, também, levantar indicadores que possam ser utilizados para comparações precisas entre diversos Ecossistemas, de forma que seja possível identificar pontos fortes e fracos para que sejam propostas uma série de ações contextualizadas para cada realidade individual.

Os Ecossistemas são classificados em quatro diferentes níveis de maturidade de acordo com os resultados obtidos com a coleta de dados e sua classificação de acordo com vinte indicadores que compõem o arcabouço desenvolvido pelos pesquisadores. Os níveis são os seguintes:

\begin{description}
  \item [Nascente (M1):] quando há um Ecossistema com algumas Startups presentes no mercado, alguns investimentos concretizados e algumas iniciativas com o objetivo de estimular ou fomentar o Ecossistema sendo realizadas mas não há reconhecimento ou as Startups não possuem representatividade nos índices de geração de emprego e renda da região.

  \item [Crescente (M2):] quando há algumas Startups estabelecidas como empresas sólidas e o Ecossistema como um todo possui representatividade notável na economia regional e nos índices de empregos. Para se enquadrar como Crescente todos indicadores essenciais e cerca de 30\% dos indicadores derivados deverão ser classificadas como nível L2. Os indicadores serão melhor explanados na sessão de Indicadores deste capítulo.

  \item [Maduro (M3):] quando existem algumas centenas de Startups em atividade, sendo algumas reconhecidas internacionalmente e com negócios realizados globalmente, um histórico relevante de investimentos concretizados dentro do Ecossistema e pelo menos uma geração de empreendedores bem sucedidos que se tornaram líderes, mentores, referências e investidores-anjo para os novos empreendedores, ajudando-os a crescer. Além desses fatores, para ser considerado como um Ecossistema Maduro, todos os indicadores essenciais e pelo menos 50\% dos indicadores derivados devem ser classificadas como nível L2 e, no mínimo, 30\% de todos os indicadores devem estar enquadrados no nível L3.

  \item [Sustentável (M4):] quando o número de Startups em atividade e de aquisições e/ou investimentos dentro do Ecossistema ultrapassam a casa dos milhares, há no mínimo duas gerações de empreendedores bem sucedidos que iniciaram suas carreiras com Startups de tecnologia presentes, uma rede de empreendedores comprometidos com o desenvolvimento do Ecossistema à longo prazo, um ambiente inclusivo com muitos eventos envolvendo temáticas que fomentem a cultura empreendedora e o mercado local e a presença de uma alta quantidade de profissionais de alta qualidade técnica. Para possuir esse estágio de maturidade, todos os indicadores essenciais devem ser classificados como nível L3 e pelos menos 80\% dos indicadores derivados também como nível L3.
\end{description}

Para isso, foram definidos cerca de 20 indicadores que devem ser qualificados entre os níveis L1, L2 ou L3 de acordo com as visões obtidas durante as Entrevistas com os Empreendedores locais, esses indicadores e o formato das entrevistas, bem como a escolha dos empreendedores, serão descritos nas sessões subsequentes. 

\section{Indicadores Utilizados}
\label{section:indicadores_utilizados}

\begin{description}
  \item Acesso à investimento em US\$ por ano: Quantidade de dinheiro investido em Startups locais, em dólares americanos, de acordo com fontes confiáveis em um dado ano. Para fins de classificação com um total de até US\$200 mil em investimentos realizados em um determinado ano o Ecossistema será considerado no Nível L1, entre US\$200 mil e US\$1 bilhão nível L2 e a partir de US\$1 bilhão nível L3. Elemento(s) relacionado(s) no arcabouço: Investimentos. 

  \item Acesso à investimento em quantidade de negócios realizados: Contagem simples de quantos investimentos foram realizados em Startups locais, independente do valor. Para fins de classificação um Ecossistema que possua até 200 investimentos concretizados será considerado como nível L1, entre 200 e mil investimentos concretizados nível L2 e acima de mil investimentos concretizados a Startup será compreendida como nível L3.

  \item Burocracia: Em sua maioria envolve o ambiente regulatório do Ecossistema Local e representa o quanto a burocrácia impacta negativamente as Startups como, por exemplo, o tempo e custo médio para se abrir uma emresa em determinado Estado. Para fins de comparação, um Ecossistema em que 40\% dos empreendedores entrevistados responderam que a Burocracia envolvendo a criação e manutenção de uma empresa é alta e um fator limitante será considerado como nível L1, um Ecossistema em que de 10 a 40\% responderem que a Burocracia é alta e limitante será nível L2 e com menos de 10\% de respostas negativas relativos à Burocracia consideraremos como nível L3.

  \item Dados do Ecossistema e Pesquisas:
  
  \item Conhecimento das Metodologias:
  
  \item Empreendedorismo nas Universidades:
  
  \item Estratégias de Saída: Quando falamos de Estratégias de Saída falamos de formas de transformar uma empresa em capital, em converter ações em dinheiro real. Investidores não estão em busca de empresas com modelos de negócios conservadores e com taxas de crescimento controladas e tímidas, eles estão em busca de empresas que vão obter uma taxa de crescimento muito alta e proporcionar possibilidades de saída rápido, geralmente com a venda da empresa ou por meio da abertura de capital na bolsa de valores, para que eles possam concretizar o investimento e lucrar. Para um investidor nada é pior do que ter o seu dinheiro investido em uma empresa sem prospecções de saída, mesmo que a empresa demonstre crescimento constante. Se não há como converter o investimento em dinheiro no bolso ele terá sido em vão. Um Ecossistema com diversas opções e uma quantidade alta de saídas concretizadas certamente atrairá muitos investidores e contribuirá para o seu crescimento. Para fins de classificação, caso não exista nenhuma opção de saída presente no Ecossistema em estudo, seu nível será L1. Caso possua uma opção nível L2 e para duas ou mais opções de saída nível L3. Elemento(s) relacionado(s) no arcabouço: Startup, Investimentos, Empresas Estabelecidas.
  
  \item Gerações do Ecossistema:
  
  \item Impostos:
  
  \item Incubadoras e Parques Tecnológicos:
  
  \item Influência de Empresas já estabelecidas:
  
  \item Mercado Global:
  
  \item Mídia Especializada:
  
  \item Número de Startups:
  
  \item Presença de Empresas de Alta Tecnologia:
  
  \item Processos de Transferência de Tecnologia:
  
  \item Qualidade das Aceleradoras:
  
  \item Qualidade do Capital Humano:
  
  \item Qualidade dos Mentores:
  
  \item Valores Culturais para o Empreendedorismo:
\end{description}

\section{Técnicas Utilizadas}
\label{section:tecnicas_utilizadas}

A escolha da Pesquisa Qualitativa como metodologia de pesquisa para avaliação do Ecossistema de Startups do Distrito Federal se deu pela falta de dados disponíveis sobre o atual estado do mesmo,
principalmente relativo à quantidade de empresas e grupos que se enquadram como Startups ativas, empregos e faturamento gerados, etc. Vale ressaltar que mesmo em um Ecossistema maduro e com um histórico
alto de pesquisas e levantamentos já realizados muitos desses dados são muito difíceis de serem levantados e mantidos, visto que Startups são empresas que nascem em ambientes de alta incerteza e
com ciclos de vida muito acelerados, de forma que é muito difícil conseguir saber quantas se tornaram empresas, quantas se mantém vivas após 6 meses de atividade, quantos empregos foram gerados, etc.

Por esse motivo acredita-se que uma abordagem Quantitativa não nos ofereceria uma visão precisa do Ecossistema que ainda é novo, é necessário um entendimento mais profundo do mesmo por meio das
visões dos empreendedores atuantes que o constróem e como suas empresas interagem entre si. Em retrospecto, uma abordagem qualitativa, que contempla técnicas para levantamento de dados por meio
de observações de grupos como um todo e também por meio de abordagens individuais, se mostra mais adequada. Também utilizaremos a Teoria Fundamentada de Dados com o objetivo de construir novas hipóteses a partir dos dados coletados, afim de conseguir enquadrar o Ecossistema em níveis e sugerir ações
de melhora.

A pesquisa foi dividida em três etapas:

\begin{enumerate}
  \item Entrevistas e Observações
  \item Codificação dos Dados
  \item Análises e Conclusões
\end{enumerate}

A primeira, de Entrevistas e Observações, dar-se-à por meio da participação em diversos eventos que movimentam o Ecossistema de Startups do Distrito Federal e as pessoas que o compõem e por meio
de entrevistas individuais com Empreendedores atuantes com objetivo de entender seu contexto pessoal e profissional, bem como suas visões sobre a realidade e as dinâmicas do Ecossistema como um todo,
quais os seus pontos fortes e fracos, seus maiores problemas, como diversas instituições e pessoas interagem entre si afim de fomenta-lo e quais ações poderiam ser tomadas afim de melhora-lo.

Com a Codificação dos Dados todas as informações levantadas pela primeira etapa serão catalogadas em tabelas com o objetivo de se tornarem referências para as etapas de Análises e Conclusões e
futuras pesquisas bem como documentar todo o processo que foi realizado. Com as Análises desses dados será possível mensurar a maturidade do Ecossistema como um todo por meio de diversos indicadores
com o objetivo de gerar as Conclusões da pesquisa, que se concentrarão no atual estágio do Ecossistema e comparações com outros Ecossistemas com o objetivo de identificar ações que poderão ser tomadas.

\section{Condução das Entrevistas}
\label{section:conducao_das_entrevistas}

Para as entrevistas foram estabelecidas uma série de Questões de Pesquisa que devem ser respondidas ao fim dessa pesquisa e as Perguntas que devem ser realizadas aos Empreendedores com o objetivo de
obter respostas que respondam à essas Questões de Pesquisa. Essas Questões foram as mesmas definidas pelo Professor Fabio Kon mas as perguntas foram adaptadas à realidade do Distrito Federal.

Todas as entrevistas devem ser realizadas, preferencialmente, no ambiente profissional dos Empreendedores de forma a mantê-los à vontade. Caso não seja possível, ela poderá ser conduzida em ambiente
escolhido pelos empreendedor, como bibliotecas, cafeterias ou eventos e apenas em último caso de forma remota. Elas também serão gravadas em aúdio caso haja consentimento do empreendedor afim de
facilitar a fase de Codificação dos Dados.

Não necessariamenteas entrevistas devem seguir de forma rígida todas as perguntas, o entrevistador poderá ter liberdade de conduzi-la como bem entender, o objetivo final é que todas as Questões de Pesquisa
sejam respondidas, como a entrevista foi conduzida ou a forma de linguagem utilizada não é de grande importância. As entrevistas não devem ser muito longas, preferencialmente não sendo extendidas por mais
de uma hora e meia.

\section{Questões de Pesquisa e Perguntas a serem feitas}
\label{section:questoes_de_pesquisa_e_perguntas}

Questões de Pesquisa:
\begin{itemize}
  \item Questão de Pesquisa 1: Quais são as características socioculturais do Distrito Federal que promovem ou inibem o espirito empreendedor?
  \item Questão de Pesquisa 2: Quais são os mecânismos institucionais e governamentais do Distrito Federal que promovem ou dificultam o Empreendedorismo?
  \item Questão de Pesquisa 3: Quais são os mecânismos educacionais do Distrito Federal que promovem ou dificultam o Empreendedorismo?
  \item Questão de Pesquisa 4: Quais são as características de times inovadores e empreendedores de sucesso no Ecossistema do Distrito Federal? Qual é a principal motivação desse Empreendedor?
  \item Questão de Pesquisa 5: Quais são os fatores tecnológicos que influenciam o sucesso ou fracasso das Startups? Como? Qual o papel executado pela comunidade e pelo Software Livre?
  \item Questão de Pesquisa 6: Quais são os aspectos metodológicos que influenciam o sucesso ou fracassos das Startups? Como? Qual o nível de adoção de métodos de desenvolvimento de negócios
  e de gerenciamento de equipes como Scrum, Startup Enxuta, etc no Ecossistema? Essa relação muda conforme amadurecimento das Startups?
\end{itemize}

Perguntas para Entrevistas:
\begin{itemize}
  \item Pergunta 01: Quais são os fatores no Distrito Federal que promovem ou inibem a criação de um espírito empreendedor?
  \item Pergunta 02: Quais são os fatores no Distrito Federal que desencorajam ou criam barreiras para o empreendedor?
  \item Pergunta 03: Quais são os mecânismos institucionais do Distrito Federal que promovem ou inibem o empreendedorismo?
  \item Pergunta 04: Qual o papel da Educação na formação do Empreendedor? Como ela acontece no Distrito Federal? Você pode indicar iniciativas que alimentam o espírito empreendedor nos estudantes?
  Você já participou de alguma atividade de educação empreendedora enquanto estudante? Quais elementos poderiam ser melhorados na formação educacional dos jovens com objetivo de fomentar o
  empreendedorismo no Distrito Federal?
  \item Pergunta 05: Quais são as características de um Empreendedor?
  \item Pergunta 06: Quais são as características de times de sucesso? Quais são os papéis de diferentes tipos de pessoas em um time e como eles se complementam? Diversidade é importante? Como?
  Como lida com as dívidas técnicas? Como o Ecossistema contribui com a formação e o crescimento do seu time?
  \item Pergunta 07: Quais as principais motivações do empreendedor do Distrito Federal? Dinheiro? Fama? Autoestima?
  \item Pergunta 08: Quais aspectos tecnológicos influenciam no sucesso das Startups? Qual o papel executado por outras tecnologias e pelo Software Livre? Como o Ecossistema contribui com a resolução
  de problemas e desafios técnicos? Como esses fatores no contexto do Distrito Federal se comparam com a realidade de outros Ecossistemas? Algumas dessas relações se alteraram conforme amadurecimento
  da sua Startup? Você já contribuíu com o desenvolvimento ou resolução de problemas técnicos de outras Startups ou de soluções Livres que foram importantes para você? Você já foi ajudado por outros
  empreendedores?
  \item Pergunta 09: Quais aspectos metodológicos influenciam no sucesso das Startups do Distrito Federal? Como? Qual o papel executado por Métodos Ágeis, Lean Startup, Business Canvas, Customer
  Development, etc? Quais práticas você utiliza? Como elas impactaram seus negócios? Há algo que não funcionou bem? Como esses fatores no contexto do Distrito Federal se comparam com a realidade
  de outros Ecossistemas? Algumas dessas relações se alteraram conforme amadurecimento da sua Startup? Você já contribuíu com o desenvolvimento do negócio de outras Startups de alguma forma? Você
  já foi ajudado por outros empreendedores?
  \item Pergunta 10: Quais erros você já cometeu na sua vida empreendedora? Se pudesse voltar no tempo, o que faria de diferente?
  \item Pergunta 11: Qual é a relação da sua Startups e sua relação como Empreendedor com o Ecossistema?
  \item Pergunta 12: Na sua opinião, quais são os elementos chave para um ecossistema de Startups vibrante? Eles existem no Distrito Federal? Se não, porque? Há algum fator no Distrito Federal que
  limite o Ecossistema? Qual?
  \item Pergunta 13: Na sua opinião, quais são as melhores e piores características do Ecossistema de Startups e dos Empreendedores do Distrito Federal? Quais soluções você vê para o crescimento do
  Ecossistema?
\end{itemize}

\section{Escolha dos Entrevistados}
\label{section:escolha_dos_entrevistados}

É muito importante que os entrevistados sejam pessoas atuantes e bem conectados com o Ecossistema de Startups do Distrito Federal como um todo e, em sua maior parte, Empreendedores mas também
Professores, Servidores Públicos, Investidores, Representantes de Incubadoras e Aceleradoras e Estudantes. Primeiramente foram definidos algumas pessoas com alto poder de contribuição no Ecossistema
e fazem parte da rede de contatos das pessoas envolvidas com a Pesquisa e ao fim de cada entrevista é recomendado que o Entrevistador pergunte por quais pessoas o Entrevistado consideram que são
referência, bem como pedido uma introdução entre as pessoas.

A meta é que sejam entrevistados cerca de 35 pessoas, sendo, preferencialmente, cerca de cinco líderes do Ecossistema, três professores envolvidos com Empreendedorismo,
dois Estudantes, cinco Representantes de Incubadoras e Aceleradoras, cinco Investidores e quinze Empreendedores. Até o momento as listagens são as seguintes:

\section{Codificação e Interpretação dos Dados}
\label{section:codificacao_e_interpretacao_dos_dados}

% TODO:!!!! Estudar Miles  Huberman sobre fases de análises qualitativas de dados !!!!

Após a realização de cada Entrevista é recomendado que o Entrevistador faça a Codificação de Dados em uma tabela pontuando os pontos-chave o mais rápido possível, de forma a ainda possuir na memória
boa parte das informações obtidas de cada Entrevistador,
