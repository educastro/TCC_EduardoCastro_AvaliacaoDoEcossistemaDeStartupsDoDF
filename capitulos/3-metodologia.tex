\chapter[Metodologia]{Metodologia}
\label{cap-metodologia}

Para cumprir com o proposto de realizar uma Avaliação da Maturidade do Ecossistema de Startups de Tecnologia do Distrito Federal foi adotada a Metodologia criada por \cite{kon2015}, este capítulo tem como objetivo descrever diversos pontos explorados por eles e algumas outras metodologias que também tem como proposta avaliar Ecossistemas.

A escolha se deu por recomendação do Orientador deste trabalho, Professor Paulo Meirelles, e por acreditar que ela fornece um bom caminho para obtermos uma visão geral e realista do atual estado do Ecossistema local por ter uma forte integração com empreendedores locais e já ter sido executada em três cidades do mundo: Tel-Aviv \cite{kon2014}, São Paulo \cite{monna2015} e Nova Iorque \cite{cukier2016}. 

Como essas três avaliações também foram feitas com base nos mesmos fatores em cima do mesmo arcabouço conceitual e muitos de seus dados foram disponibilizados publicamente nos artigos citados é possível realizar comparações diretas entre os quatro Ecossistemas, facilitando a idenficação de pontos a serem aprimorados no contexto do Distrito Federal.

\section{A Metodologia criada por Pesquisadores da Universidade de São Paulo}
\label{section:metodologia_de_avaliacao_adotada}

\subsection{Técnicas Utilizadas}
\label{subsection:tecnicas_utilizadas}

Toda a metodologia é construida com base nas técnicas de Pesquisa Qualitativa e na Teoria Fundamentada em Dados por oferecerem a possibilidade de obter dados a partir das visões daqueles que melhor o entendem e lidam com o Ecossistema de Startups local e seus pontos fortes e fracos todos os dias - os próprios empreendedores - por meio de entrevistas, dessa forma é possível valorizar e obter respostas a partir de suas experiências individuais.

! TO-DO: estudar sobre pesquisa qualitativa e teoria fundamentada em dados para embasar melhor essa subseção !

\subsection{O arcabouço conceitual de um Ecossistema e um modelo conceitual}
\label{subsection:arcabouco_conceitual_e_modelo}

Após profunda análise bibliográfica e estudos qualitativos por meio de entrevistas com mais de cem pessoas chaves em três ecossistemas explorados, \citeonline{kon2014} mapearam diversos elementos do Ecossistema e seus respectivos relacionamentos e como um dos resultados criaram um arcabouço conceitual, representado pela Figura 01, de um Ecossistema de Startups e como seus diversos fatores interagem entre si.

\begin{figure}[!htb]
\centering
\includegraphics{figuras/arcabouco_teorico_de_israel}
\caption{Arcabouço criado a partir do estudo feito em Israel}
\label{Rotulo}
\end{figure}

\subsection{Fatores que fazem parte de um Ecossistema}
\label{subsection:indicadores_utilizados}

Foram definidos cerca de 20 fatores que compõem um Ecossistema e para que seja realizada a avaliação de maturidade eles devem ser qualificados entre os níveis L1, L2 ou L3 de acordo com as visões obtidas durante as Entrevistas com as pessoas chaves do Ecossistema em análise.

\begin{description}
  \item [Acesso à investimento em US\$ por ano:] Quantidade de dinheiro investido em Startups locais, em dólares americanos, de acordo com fontes confiáveis em um dado ano. Para fins de classificação com um total de até US\$200 mil em investimentos realizados em um determinado ano o Ecossistema será considerado no Nível L1, entre US\$200 mil e US\$1 bilhão nível L2 e a partir de US\$1 bilhão nível L3. Elemento(s) relacionado(s) no arcabouço: Opções de Investimento. 

  \item [Acesso à investimento em quantidade de negócios realizados:] Contagem simples de quantos investimentos foram realizados em Startups locais, independente do valor, de acordo com fontes confiáveis em um dado ano. Para fins de classificação um Ecossistema que possua até 200 investimentos concretizados será considerado como nível L1, entre 200 e mil investimentos concretizados nível L2 e acima de mil investimentos concretizados a Startup será compreendida como nível L3. Elemento(s) relacionado(s) no arcabouço: Opções de Investimento.

  \item [Burocracia:] Em sua maioria envolve o ambiente regulatório do Ecossistema Local e representa o quanto a burocrácia impacta negativamente as Startups como, por exemplo, o tempo e custo médio para se abrir uma emresa em determinado Estado. Para fins de comparação, um Ecossistema em que 40\% dos empreendedores entrevistados responderam que a Burocracia envolvendo a criação e manutenção de uma empresa é alta e um fator limitante será considerado como nível L1, um Ecossistema em que de 10 a 40\% responderem que a Burocracia é alta e limitante será nível L2 e com menos de 10\% de respostas negativas relativos à Burocracia consideraremos como nível L3. Elemento(s) relacionado(s) no arcabouço: Ambiente Regulatório. (usar como base tempo médio de abertura de empresas, ICI, etc)

  \item [Dados do Ecossistema e Pesquisas:] As universidades e os institutos de pesquisas são peças triviais em um Ecossistema de Startups, em especial por constantemente levantarem questões, respostas, informações e pontos que devem ser aprimorados em prol de um ambiente mais maduro e preparado. Também é importante que os dados sejam amplamente acessíveis, de forma que diversas peças interessadas possam ter acesso para embasarem suas ações, identificarem pontos em que podem contribuir ou atraírem mais pessoas para o Ecossistema. Se não há dados ou centros de pesquisa disponíveis no Ecossistema ele será classificado como nível L1, se estão parcialmente disponíveis como nível L2 e se estiverem disponíveis será nível L3. Elemento(s) relacionado(s) no arcabouço: Centros de Pesquisa, Governo.
  
  \item [Conhecimento das Metodologias:] Porcentagem de Empreendedores que possuem conhecimento de diversas metodologias comumente utilizadas pelo mercado como Métodos Ágeis, Lean Startup, Canvas, Design Thinking, etc. Se menos de 20\% dos Empreendedores Entrevistados estiverem familiarizados com algumas das Metodologias, o nível do Ecossistema será L1. Se entre 20 e 60\% conhecerem classificaremos como L2 e acima de 60\% como L3. Elemento(s) relacionado(s) no arcabouço: Metodologias. (talvez fosse interessante envolver a quantidade de eventos, como sugerido pelo paper, e entrevistas com alunos próximos de se formar)
  
  \item [Empreendedorismo nas Universidades:] Porcentagem de ex-alunos que fundaram uma empresa em até 5 anos após a graduação. Se menos de 2\% dos ex-alunos fundaram uma empresa o Ecossistema estará no nível L1, entre 2 e 10\% no nível L2 e acima de 10\% no nível L3. Elemento(s) relacionado(s) no arcabouço: Universidades, Centros de Pesquisa e Educação. (talvez fosse mais interessante entrevistar alunos próximos da formatura)
  
  \item [Estratégias de Saída:] Quando falamos de Estratégias de Saída falamos de formas de transformar uma empresa em capital, em converter ações em dinheiro real. Investidores não estão em busca de empresas com modelos de negócios conservadores e com taxas de crescimento controladas e tímidas, eles estão em busca de empresas que vão obter uma taxa de crescimento muito alta e proporcionar possibilidades de saída rápido, geralmente com a venda da empresa ou por meio da abertura de capital na bolsa de valores, para que eles possam concretizar o investimento e lucrar. Para um investidor nada é pior do que ter o seu dinheiro investido em uma empresa sem prospecções de saída, mesmo que a empresa demonstre crescimento constante. Se não há como converter o investimento em dinheiro no bolso ele terá sido em vão. Um Ecossistema com diversas opções e uma quantidade alta de saídas concretizadas certamente atrairá muitos investidores e contribuirá para o seu crescimento. Para fins de classificação, caso não exista nenhuma opção de saída presente no Ecossistema em estudo, seu nível será L1. Caso possua uma opção nível L2 e para duas ou mais opções de saída nível L3. Elemento(s) relacionado(s) no arcabouço: Startup, Investimentos, Empresas Estabelecidas.
  
  \item [Gerações do Ecossistema:] De tempos em tempos o Ecossistema possui uma nova leva de Empreendedores se destacando no mercado e, conforme sua maturidade aumenta, novas gerações são inspiradas, influenciadas e apoiadas pelas anteriores. Se um dado Ecossistema não possuir nenhuma geração anterior de Empreendedores bem sucedidos investindo e mentorando novas Startups seu nível nesse fator será L1, se houver pelo menos uma geração atuante será L2 e acima de duas gerações L3. Elemento(s) relacionado(s) no arcabouço: Empreendedor, Sociedade.
  
  \item [Impostos:] Baseado no ranking de impostos entre países criado por \citeonline{schwab2015}. Acima de 50\% no valor proposto pelo ranking o Ecossistema será classificado no nível L1, entre 30 e 50\% no nível L2 e abaixo de 30\% no nível L1. Elemento(s) relacionado(s) no arcabouço: Ambiente Regulatório, Mercado.
  
  \item [Incubadoras, Aceleradoras e Parques Tecnológicos:] Representação da quantidade de incubadoras e parques tecnológicos presentes no Ecossistema. Caso o valor seja até 2, o nível será L1. Entre 2 e 10 o nível será L2 e acima de 10 o fator será classificado como L3. Elemento(s) relacionado(s) no arcabouço: Incubadoras, Aceleradoras.
  
  \item [Influência de Empresas já estabelecidas:] A quantidade de empresas estabelecidas e engajadas em movimentar o Ecossistema por meio de eventos, líderança, mentoria e apoio, investimentos ou programas de aceleração para Startups locais. Se forem menos de 20 o nível do fator será L1, entre 20 e 80 será nível L2 e acima de 80 será nível L3. Elemento(s) relacionado(s) no arcabouço: Eventos, Empresas Estabelecidas, Aceleradoras, Empreendedores.
  
  \item [Mercado Global:] Porcentagem de Startups ativas no Ecossistema com abrangência de mercado global. Se menos de 10\% das Startups forem globais o nível do fator será L1, entre 10 e 50\% o nível será L2 e acima de 50\% com atividades e clientes em pelo menos mais de um país fora o Brasil o nível será L3. Elemento(s) relacionado(s) no arcabouço: Mercado.
  
  \item [Mídia Especializada:] A participação da mídia é muito importante para a promoção do Ecossistema como um todo e de seus Empreendedores, portanto a presença de profissionais engajados e que entendam o contexto do mercado local é de extrema importância. Se os Empreendedores entrevistados indicarem menos de três representantes da mídia que são atuantes no Ecossistema esse fator será classificado como nível L1, entre 3 e 5 representantes da mídia nível L2 e acima de 5 representantes nível L3. Elemento(s) relacionado(s) no arcabouço: Mídia.
  
  \item [Número de Startups:] Número de Startups em atividade por ano de acordo com fontes de dados confiáveis em um dado ano. Caso exista menos de 500, o Ecossistema será classificado nesse fator como nível L1, caso existe entre 500 e 3000 como nível L2 e acima de 3000 como nível L3. Elemento(s) relacionado(s) no arcabouço: Mercado.
  
  \item [Presença de Empresas de Alta Tecnologia:] Quantidade de empresas de alta tecnologia presentes no Ecossistema. Elemento(s) relacionado(s) no arcabouço: Empresas Estabelecidas.
  
  \item [Processos de Transferência de Tecnologia:] Índice baseado nos fatores de Inovação e Sofisticação definidos por \citeonline{schwab2015}. Elemento(s) relacionado(s) no arcabouço: Universidades, Centros de Pesquisa e Ambiente Regulatório.
  
  \item [Qualidade das Aceleradoras:] Porcentagem das Startups que passaram por algum programa de Aceleração ou Incubação e se estabeleceram bem no mercado ou avançaram com sucesso para a fase de captação de investimento de terceiros. Se menos de 10\% das Startups atingiram os estágios descritos esse fator será classificado como L1, se entre 10 e 50\% atingiram a descrição a classificação será L2 e se mais de 50\% alcançaram esse estágio o nível será L3. Elemento(s) relacionado(s) no arcabouço: Aceleradoras, Incubadoras e Parques Tecnológicos.
  
  \item [Qualidade do Capital Humano:] Fator baseado no índice de talentos definido por \citeonline{compass2015}. Se o Ecossistema em questão estiver posicionado acima da 20ª posição no índice este fator será classificado como L1, se estiver posicionado entre a 15ª e 20ª colocações será classificado como L2 e se estiver entre os 15 melhores será classificado como L3. Elemento(s) relacionado(s) no arcabouço: Empreendedor, Educação.
  
  \item [Qualidade dos Mentores:] Um mentor de qualidade é um empreendedor experiente, alguém que já viveu os problemas que o novo empreendedor está passando e entende perfeitamente a sua situação, ninguém melhor para orienta-lo do que alguém que já passou por problemas similares ou iguais. Para esse fator consideraremos como um mentor de sucesso aquele que já construiu ao menos uma startup bem sucedida ou possui pelo menos 10 anos de experiência com Startups. Se menos de 10\$ dos mentores citados pelos Empreendedores se enquadrarem nessas características o nível desse fator será L1, se entre 10 e 50\% dos mentores citados se enquadrarem o nível será L2 e se mais de 50\% dos mentores se enquadrarem, L3. Elemento(s) relacionado(s) no arcabouço: Empreendedor.
  
  \item [Valores Culturais para o Empreendedorismo:] Fator baseado no índice de suporte cultural definido por \citeonline{gedi2016}. Se o valor do índice for menor do que 0.5, o fator será classificado como nível L1, se estiver entre 0.5 e 0.75 será nível L2 e acima de 0.75 nível L3. Elemento(s) relacionado(s) no arcabouço: Cultura, Sociedade e Família.
\end{description}

\subsection{Níveis de Maturidade de um Ecossistema}
\label{subsection:niveis_de_maturidade_de_um_ecossistema}

Além do arcabouço conceitual a Metodologia tem como objetivo classificar cada Ecossistemas entre quatro diferentes níveis de maturidade de acordo com os resultados obtidos com a coleta de dados e sua classificação de acordo com vinte indicadores indicados pelos pesquisadores. Os níveis são os seguintes:

\begin{description}
  \item [Nascente (M1):] quando há um Ecossistema com algumas Startups presentes no mercado, alguns investimentos concretizados e algumas iniciativas com o objetivo de estimular ou fomentar o Ecossistema sendo realizadas mas não há reconhecimento ou as Startups não possuem representatividade nos índices de geração de emprego e renda da região.

  \item [Crescente (M2):] quando há algumas Startups estabelecidas como empresas sólidas e o Ecossistema como um todo possui representatividade notável na economia regional e nos índices de empregos. Para se enquadrar como Crescente todos indicadores essenciais e cerca de 30\% dos indicadores derivados deverão ser classificadas como nível L2. Os indicadores serão melhor explanados na sessão de Indicadores do Ecossistema deste capítulo.

  \item [Maduro (M3):] quando existem algumas centenas de Startups em atividade, sendo algumas reconhecidas internacionalmente e com negócios realizados globalmente, um histórico relevante de investimentos concretizados dentro do Ecossistema e pelo menos uma geração de empreendedores bem sucedidos que se tornaram líderes, mentores, referências e investidores-anjo para os novos empreendedores, ajudando-os a crescer. Além desses indicadores, para ser considerado como um Ecossistema Maduro, todos os indicadores essenciais e pelo menos 50\% dos indicadores derivados devem ser classificadas como nível L2 e, no mínimo, 30\% de todos os indicadores devem estar enquadrados no nível L3.

  \item [Sustentável (M4):] quando o número de Startups em atividade e de aquisições e/ou investimentos dentro do Ecossistema ultrapassam a casa dos milhares, há no mínimo duas gerações de empreendedores bem sucedidos que iniciaram suas carreiras com Startups de tecnologia presentes, uma rede de empreendedores comprometidos com o desenvolvimento do Ecossistema à longo prazo, um ambiente inclusivo com muitos eventos envolvendo temáticas que fomentem a cultura empreendedora e o mercado local e a presença de uma alta quantidade de profissionais de alta qualidade técnica. Para possuir esse estágio de maturidade, todos os indicadores essenciais devem ser classificados como nível L3 e pelos menos 80\% dos indicadores derivados também como nível L3.
\end{description}

\subsection{Aplicação da Metodologia}
\label{subsection:aplicacao_da_metodologia}

A aplicação da Metodologia foi dividida em três etapas:

\begin{enumerate}
  \item Entrevistas e Observações
  \item Codificação dos Dados
  \item Análises e Conclusões
\end{enumerate}

A primeira, de Entrevistas e Observações, dar-se-à por meio da observação de diversos eventos que movimentam o Ecossistema de Startups do Distrito Federal e das pessoas que o compõem e por meio
de entrevistas individuais com Empreendedores atuantes com objetivo de entender seu contexto pessoal e profissional, bem como suas visões sobre a realidade e as dinâmicas do Ecossistema como um todo, quais os seus pontos fortes e fracos, seus maiores problemas, como diversas instituições e pessoas interagem entre si afim de fomenta-lo e quais ações poderiam ser tomadas afim de melhora-lo.

Com a Codificação dos Dados todas as informações levantadas pela primeira etapa serão catalogadas em tabelas com o objetivo de se tornarem referências para as etapas de Análises e Conclusões e
futuras pesquisas bem como documentar todo o processo que foi realizado. 

Com as Análises desses dados será possível mensurar a maturidade do Ecossistema como um todo por meio de diversos indicadores com o objetivo de gerar as Conclusões da pesquisa, que se concentrarão em explicitar o atual estágio do Ecossistema de acordo com a Metodologia utilizada, comparações com outros Ecossistemas e uma série de ações que podem ser tomadas para aprimorar determinados pontos.

\subsection{Condução das Entrevistas}
\label{subsection:conducao_das_entrevistas}

Todas as entrevistas devem ser realizadas, preferencialmente, no ambiente profissional dos Empreendedores de forma a mantê-los à vontade. Caso não seja possível, ela poderá ser conduzida em ambiente
escolhido pelos empreendedor, como bibliotecas, cafeterias ou eventos e apenas em último caso de forma remota. Elas também serão gravadas em aúdio caso haja consentimento do empreendedor afim de
facilitar a fase de Codificação dos Dados.

Não necessariamenteas entrevistas devem seguir de forma rígida todas as perguntas, o entrevistador poderá ter liberdade de conduzi-la como bem entender, o objetivo final é que todas as Questões de Pesquisa
sejam respondidas, como a entrevista foi conduzida ou a forma de linguagem utilizada não é de grande importância. As entrevistas não devem ser muito longas, preferencialmente não sendo extendidas por mais
de uma hora e meia.

Para as entrevistas foram estabelecidas uma série de Questões de Pesquisa que devem ser respondidas ao fim dessa pesquisa e as Perguntas que devem ser realizadas aos Empreendedores com o objetivo de
obter respostas que respondam à essas Questões de Pesquisa. Essas Questões foram as mesmas definidas pelo Professor Fabio Kon mas as perguntas foram adaptadas à realidade do Distrito Federal.

\subsection{Questões de Pesquisa e Perguntas a serem feitas}
\label{subsection:questoes_de_pesquisa_e_perguntas}

Questões de Pesquisa:
\begin{itemize}
  \item Questão de Pesquisa 1: Quais são as características socioculturais do Distrito Federal que promovem ou inibem o espirito empreendedor?
  \item Questão de Pesquisa 2: Quais são os mecânismos institucionais e governamentais do Distrito Federal que promovem ou dificultam o Empreendedorismo?
  \item Questão de Pesquisa 3: Quais são os mecânismos educacionais do Distrito Federal que promovem ou dificultam o Empreendedorismo?
  \item Questão de Pesquisa 4: Quais são as características de times inovadores e empreendedores de sucesso no Ecossistema do Distrito Federal? Qual é a principal motivação desse Empreendedor?
  \item Questão de Pesquisa 5: Quais são os fatores tecnológicos que influenciam o sucesso ou fracasso das Startups? Como? Qual o papel executado pela comunidade e pelo Software Livre?
  \item Questão de Pesquisa 6: Quais são os aspectos metodológicos que influenciam o sucesso ou fracassos das Startups? Como? Qual o nível de adoção de métodos de desenvolvimento de negócios
  e de gerenciamento de equipes como Scrum, Startup Enxuta, etc no Ecossistema? Essa relação muda conforme amadurecimento das Startups?
\end{itemize}

Perguntas para Entrevistas:
\begin{itemize}
  \item Pergunta 01: Quais são os fatores no Distrito Federal que promovem ou inibem a criação de um espírito empreendedor?
  \item Pergunta 02: Quais são os fatores no Distrito Federal que desencorajam ou criam barreiras para o empreendedor?
  \item Pergunta 03: Quais são os mecânismos institucionais do Distrito Federal que promovem ou inibem o empreendedorismo?
  \item Pergunta 04: Qual o papel da Educação na formação do Empreendedor? Como ela acontece no Distrito Federal? Você pode indicar iniciativas que alimentam o espírito empreendedor nos estudantes?
  Você já participou de alguma atividade de educação empreendedora enquanto estudante? Quais elementos poderiam ser melhorados na formação educacional dos jovens com objetivo de fomentar o
  empreendedorismo no Distrito Federal?
  \item Pergunta 05: Quais são as características de um Empreendedor?
  \item Pergunta 06: Quais são as características de times de sucesso? Quais são os papéis de diferentes tipos de pessoas em um time e como eles se complementam? Diversidade é importante? Como?
  Como lida com as dívidas técnicas? Como o Ecossistema contribui com a formação e o crescimento do seu time?
  \item Pergunta 07: Quais as principais motivações do empreendedor do Distrito Federal? Dinheiro? Fama? Autoestima?
  \item Pergunta 08: Quais aspectos tecnológicos influenciam no sucesso das Startups? Qual o papel executado por outras tecnologias e pelo Software Livre? Como o Ecossistema contribui com a resolução
  de problemas e desafios técnicos? Como esses fatores no contexto do Distrito Federal se comparam com a realidade de outros Ecossistemas? Algumas dessas relações se alteraram conforme amadurecimento
  da sua Startup? Você já contribuíu com o desenvolvimento ou resolução de problemas técnicos de outras Startups ou de soluções Livres que foram importantes para você? Você já foi ajudado por outros
  empreendedores?
  \item Pergunta 09: Quais aspectos metodológicos influenciam no sucesso das Startups do Distrito Federal? Como? Qual o papel executado por Métodos Ágeis, Lean Startup, Business Canvas, Customer
  Development, etc? Quais práticas você utiliza? Como elas impactaram seus negócios? Há algo que não funcionou bem? Como esses fatores no contexto do Distrito Federal se comparam com a realidade
  de outros Ecossistemas? Algumas dessas relações se alteraram conforme amadurecimento da sua Startup? Você já contribuíu com o desenvolvimento do negócio de outras Startups de alguma forma? Você
  já foi ajudado por outros empreendedores?
  \item Pergunta 10: Quais erros você já cometeu na sua vida empreendedora? Se pudesse voltar no tempo, o que faria de diferente?
  \item Pergunta 11: Qual é a relação da sua Startups e sua relação como Empreendedor com o Ecossistema?
  \item Pergunta 12: Na sua opinião, quais são os elementos chave para um ecossistema de Startups vibrante? Eles existem no Distrito Federal? Se não, porque? Há algum fator no Distrito Federal que
  limite o Ecossistema? Qual?
  \item Pergunta 13: Na sua opinião, quais são as melhores e piores características do Ecossistema de Startups e dos Empreendedores do Distrito Federal? Quais soluções você vê para o crescimento do
  Ecossistema?
\end{itemize}

\subsection{Escolha dos Entrevistados}
\label{subsection:escolha_dos_entrevistados}

É muito importante que os entrevistados sejam pessoas atuantes e bem conectados com o Ecossistema de Startups do Distrito Federal como um todo e, em sua maior parte, Empreendedores mas também
Professores, Servidores Públicos, Investidores, Representantes de Incubadoras e Aceleradoras e Estudantes. Primeiramente foram definidos algumas pessoas com alto poder de contribuição no Ecossistema
e fazem parte da rede de contatos das pessoas envolvidas com a Pesquisa e ao fim de cada entrevista é recomendado que o Entrevistador pergunte por quais pessoas o Entrevistado consideram que são
referência, bem como pedido uma introdução entre as pessoas.

A meta é que sejam entrevistados cerca de 35 pessoas, sendo, preferencialmente, cerca de cinco líderes do Ecossistema, três professores envolvidos com Empreendedorismo,
dois Estudantes, cinco Representantes de Incubadoras e Aceleradoras, cinco Investidores e quinze Empreendedores. Até o momento as listagens são as seguintes:

\subsection{Codificação e Interpretação dos Dados}
\label{subsection:codificacao_e_interpretacao_dos_dados}

% TODO:!!!! Estudar Miles  Huberman sobre fases de análises qualitativas de dados !!!!

Após a realização de cada Entrevista é recomendado que o Entrevistador faça a Codificação de Dados em uma tabela pontuando os pontos-chave o mais rápido possível, de forma a ainda possuir na memória
boa parte das informações obtidas de cada Entrevistador,

\section{Trabalhos Relacionados}
\label{section:trabalhos_relacionados}