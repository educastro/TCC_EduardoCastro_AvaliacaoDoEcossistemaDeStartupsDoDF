\chapter{Metodologia}
\label{cap-metodologia}

\citeonline{Kon2014} explora uma abordagem mista, tendo como base técnicas de análise qualitativas e quantitativas, com o objetivo de mensurar a maturidade de um determinado ecossistema de startups por meio de entrevistas com atores dos ecossistemas locais juntamente com a exploração de dados estatísticos acerca do ecossistema. Assim como \citeonline{Frenkel2014}, Kon também se propôs a criar um arcabouço conceitual do ecossistema em estudo e aplicou a referida metodologia nas cidades de Tel Aviv (Israel), Nova Iorque (EUA) e São Paulo.

A forma como a metodologia foi construída permite que os pesquisadores realizem comparações diretas entre ecossistemas ao classificá-los por níveis de maturidade (nascente, crescente, maduro e autosustentável). Uma síntese dos resultados das aferições realizadas em Tel Aviv e São Paulo estão disponíveis nas Figuras \ref{figure:mapa_conceitual_tel_aviv}, \ref{figure:mapa_conceitual_sao_paulo} e \ref{figure:tabela_metricas}, todas no Anexo C deste trabalho. As duas primeiras figuras são uma representação gráfica de como os diversos elementos que compõem os ecossistemas estudados (educação, demografia, investimentos, mercado, ambiente regulatório, tecnologias, etc.) interagem entre si e afetam as \textit{startups} que os compõem, a terceira figura trás o resultado das análises dos indicadores explorados pela metodologia.

A maior vantagem da metodologia proposta é oferecer uma forma de obter conhecimento a partir das visões daqueles que melhor o entendem e lidam com o ecossistema de startups local - os próprios empreendedores. Ao dar uma maior prioridade a esse tipo de abordagem ao invés de uma análise puramente quantitativa torna-se possível obter uma visualização mais realista e próxima de quais são as características do ecossistema em estudo como um todo, além de contornar a falta de bases de dados alimentadas de forma sistemática de ecossistemas menos estruturadas e maduros. Essa abordagem também aumenta as chances de que a atividade de pesquisa entregará valor ao pesquisador mesmo quando alguns dos fatores propostos pela metodologia não são aplicáveis no ecossistema em análise.

Dentre as técnicas de pesquisa qualitativa que serão utilizadas está a Teoria Fundamentada em Dados (\textit{Grounded Theory}) por ser uma forma de se gerar conhecimento a respeito do objeto de estudo e das questões de pesquisa de forma indutiva, como relatado por \citeonline{Glaser1999}. No caso deste trabalho se deu por meio de uma observação participante e por meio de interações com os atores locais que compõem o ecossistema de startups do Distrito Federal.

\section{Os níveis de maturidade de um Ecossistema}
\label{subsection:niveis_de_maturidade_de_um_ecossistema}

Além de elaborar o mapa do ecossistema a Metodologia tem como um dos seus objetivos classificar Ecossistemas entre quatro diferentes níveis de maturidade. Os níveis são os seguintes:

\begin{description}
  \item [Nascente (M1):] quando há um Ecossistema com algumas Startups presentes no mercado, alguns investimentos concretizados e algumas iniciativas com o objetivo de estimular ou fomentar o Ecossistema sendo realizadas mas não há reconhecimento ou as Startups não possuem representatividade nos índices de geração de emprego e renda da região.

  \item [Crescente (M2):] quando há algumas Startups estabelecidas como empresas sólidas e o Ecossistema como um todo possui representatividade notável na economia regional e nos índices de empregos. Para se enquadrar como Crescente todos fatores essenciais e cerca de 30\% dos fatores derivados deverão ser classificadas como nível L2.

  \item [Maduro (M3):] quando existem algumas centenas de Startups em atividade, sendo algumas reconhecidas internacionalmente e com negócios realizados globalmente, um histórico relevante de investimentos concretizados dentro do Ecossistema e pelo menos uma geração de empreendedores bem sucedidos que se tornaram líderes, mentores, referências e investidores-anjo para os novos empreendedores, ajudando-os a crescer. Além dessas características, para ser considerado como um Ecossistema Maduro, todos os fatores essenciais e pelo menos 50\% dos fatores derivados devem ser classificadas como nível L2 e, no mínimo, 30\% de todos os fatores devem estar enquadrados no nível L3.

  \item [Sustentável (M4):] quando o número de Startups em atividade e de aquisições e/ou investimentos dentro do Ecossistema ultrapassam a casa dos milhares, há no mínimo duas gerações de empreendedores bem sucedidos que iniciaram suas carreiras com Startups de tecnologia presentes, uma rede de empreendedores comprometidos com o desenvolvimento do Ecossistema à longo prazo, um ambiente inclusivo com muitos eventos envolvendo temáticas que fomentem a cultura empreendedora e o mercado local e a presença de uma alta quantidade de profissionais de alta qualidade técnica. Para possuir esse estágio de maturidade, todos os fatores essenciais devem ser classificados como nível L3 e pelos menos 80\% dos fatores derivados também como nível L3.
\end{description}

\section{Fatores que formam um Ecossistema}
\label{subsection:fatores_que_formam_um_ecossistema}

Após vasta pesquisa bibliográfica e entrevistas com mais de 50 pessoas chaves para os Ecossistemas de Tel-Aviv e São Paulo foram definidos cerca de 21 fatores que os compõem e fazem parte do Arcabouço Teórico de um Ecossistema, descrito na Subseção \ref{subsection:arcabouco_conceitual_e_modelo}. Com o objetivo de classificá-los para facilitar possíveis comparações entre ecossistemas foram definidas as seguintes métricas e níveis para cada um dos fatores, todas dispostas na Tabela \ref{table:metricas_de_classificacao_dos_fatores}. Os fatores que contém o símbolo ``*''  após seu nome são o que Kon chamou de fatores essenciais, aqueles considerados os mais importantes para o sucesso de um ecossistema, e os restantes são considerados fatores derivados, mais relacionados a características que contribuem para o amadurecimento/desenvolvimento de ecossistemas. Uma descrição de cada um dos fatores citados está disponível no Apêndice \ref{apendices:fatores_de_um_ecossistema}.

\begin{table}[H]
\centering
\begin{tabular}{ | c | c | c | c |}
\hline
\thead{Fator} & \thead{L1} & \thead{L2} & \thead{L3} \\
\hline
Estratégias de Saída*                                      &     00     &     01     &    >=2      \\
\hline
Mercado Global*                                            &    <10\%   &   10-40\%  &    >40\%    \\
\hline
Empreendedorismo nas Universidades*                        &    <02\%   &   02-10\%  &    >10\%    \\
\hline
Valores Culturais para o Empreendedorismo*                 &    <0.5    &   0.5-0.75 &    >0.75    \\
\hline
Eventos relacionados à Startups*                           &   mensal  &   semanal   &    diário    \\
\hline
Dados do Ecossistema e Pesquisas*                          &    nada    & parciais    & disponíveis \\
\hline
Gerações do Ecossistema*                                   &     00     &    0.1     &    02       \\
\hline
Qualidade de Mentores                                      &    <10\%   &   10-50\%  &    >50\%    \\
\hline
Burocracia                                                 &    >40\%   &   10-40\%  &    <10\%    \\
\hline
Gastos com impostos                                        &    >50\%   &   30-50\%  &    <30\%    \\
\hline
Qualidade das Aceleradoras                                 &    <10\%   &   10-50\%  &    >50\%    \\
\hline
Acesso à investimento em US\$ por ano                      &    <200M   &   200M-1B  &    >1B      \\
\hline
Qualidade do Capital Humano                                &    >20th   &   15-20th  &    <15th    \\
\hline
Processos de Transferência de Tecnologia                   &    <4.0    &   4.0-5.0  &    >5.0     \\
\hline
Conhecimento das Metodologias                              &    20\%    &   20-60\%  &    >60\%    \\
\hline
Atores da Mídia com foco no Empreendedorismo               &    <03     &   03-05    &    > 05     \\
\hline
Número de Startups*                                        &    <200    &   200-1k   &    >1k      \\
\hline
Acesso à investimento anjo em quantidade/ano*              &    <05     &   05-50    &    >50      \\
\hline
Presença de Empresas de Alta Tecnologia*                   &    <02     &   02-10    &    >10      \\
\hline
\makecell{Acesso à investimento em quantidade \\de negócios/ano}&\makecell{<50}&\makecell{50-300}&\makecell{>300}\\
\hline
Incubadoras e Parques Tecnológicos                         &     01     &    02-05   &    >5       \\
\hline
Influência de Empresas já estabelecidas                    &    <02     &   02-10    &    >10      \\
\hline
\end{tabular}

\caption{Métricas de classificação dos Fatores que compõem um Ecossistema.}
\label{table:metricas_de_classificacao_dos_fatores}
\end{table}

Também foi desenvolvida uma versão mais enxuta do modelo de avaliação proposto, com foco em apenas oito fatores ao invés de 21. Para que um ecossistema seja enquadrado em um determinado nível (nascente, crescente, maduro ou sustentável) sete, dos oito, fatores da versão enxuta devem ser classificados no mesmo nível ou superior. Os parâmetros utilizados estão representados na Tabela \ref{table:metricas_de_classificacao_versao_enxuta} e a importância de cada um dos fatores para cada nível de maturidade na Tabela \ref{table:valor_das_metricas_de_classificacao_versao_enxuta}, vale ressaltar que os níveis de importância de cada fator foram obtidos por meio de ``workshops'' com os empreendedores entrevistados nos três ecossistemas explorados pelo InovaSampa.

A indicação da importância de cada fator para cada nível de maturidade também pode ser de grande utilidade para orientar os atores do ecossistema a identificarem onde devem ser concentrados os esforços para desenvolvê-lo de acordo com o nível de maturidade atual. Por exemplo, para ecossistemas nascentes e crescentes se mostra mais importante investir no fortalecimento da cultura empreendedora e em conexões entre empreendedores por meio de eventos, ambos com classificação de fatores muito importantes (***) enquanto, para esse nível, esforços em estratégias de saída e investimentos não se mostram importantes (classificação *). Por outro lado, em ecossistemas classificados como sustentáveis se mostra mais interessante concentrar energias em desenvolver estratégias de saída e investimentos e obter um forte apoio da mídia local, todos classificados como fatores muito importantes (***) para ecossistemas sustentáveis.

\begin{table}[H]
\centering
\begin{tabular}{ | c | c | c | c | c |}
\hline
\thead{Fator} & \thead{Nascente} & \thead{Crescente} &\thead{Maduro}& \thead{Sustentável} \\
\hline
\makecell{Estratégias\\de saída}&nenhum&poucos&\makecell{várias aquisições\\e fusões mas\\poucos \textit{IPO's}}&\makecell{várias aquisições\\e fusões e\\muitos \textit{IPO's}}\\
\hline
\makecell{Empreendedorismo\\nas universidades}&\makecell{<02\%}&\makecell{02-10\%}&\makecell{10\%}&\makecell{>10\%} \\
\hline
\makecell{Investimento Anjo}&irrelevante &   irrelevante  &  alguns & muitos    \\
\hline
\makecell{Valores culturais\\para o\\empreendedorismo}&<0.5    &   0.5-0.6 &    0.6-0.7 & > 0.7    \\
\hline
\makecell{Atores da mídia\\com foco no\\empreendedorismo}&nenhum     &   alguns    &    muitos & todos     \\
\hline
\makecell{Dados do ecossistema\\e pesquisas}&nenhum    & nenhum & parciais    & completos \\
\hline 
\makecell{Gerações do\\ecossistema}&0& 0     &    1-2     &    >= 3       \\
\hline
\makecell{Eventos}&mensais & semanais & diários  & > diários \\
\hline
\end{tabular}

\caption{Indicadores da versão enxuta.}
\label{table:metricas_de_classificacao_versao_enxuta}
\end{table}

\begin{table}[H]
\centering
\begin{tabular}{ | c | c | c | c | c |}
\hline
\thead{Fator} & \thead{Nascente} & \thead{Crescente} &\thead{Maduro}& \thead{Sustentável} \\
\hline
\makecell{Estratégias\\de saída}& * & * & *** & *** \\
\hline
\makecell{Investimento Anjo}& * & * & ** & *** \\
\hline
\makecell{Cultura Empreendedora}& *** & *** & *** & ** \\
\hline
\makecell{Atores da mídia\\com foco no\\empreendedorismo}& * & ** & *** & *** \\
\hline
\makecell{Dados do ecossistema\\e pesquisas}& * & * & ** & *** \\
\hline 
\makecell{Gerações do\\ecossistema}& * & * & ** & *** \\
\hline
\makecell{Eventos}& *** & *** & ** & * \\
\hline 
\makecell{Incubadoras\\e\\Parques Tecnológicos}& *** & *** & ** & * \\
\hline
\makecell{Ambiente regulatório}& * & ** & *** & *** \\
\hline \hline
\makecell{Legenda}& \makecell{*: Não\\importante}& \makecell{**: Importante}&\makecell{***: Muito\\importante} & \\
\hline
\end{tabular}

\caption{Importância das métricas da versão enxuta.}
\label{table:valor_das_metricas_de_classificacao_versao_enxuta}
\end{table}

\section{Adaptação para o estudo do ecossistema do DF}
\label{section:adaptacoes_para_o_trabalho}

Por se tratar de um ecossistema recente e com menos dados disponíveis em comparação a ecossistemas mais desenvolvidos, como São Paulo, Nova Iorque e Tel-Aviv, para a parte quantitativa do trabalho foi utilizada a versão enxuta da metodologia com pequenas adaptações para melhor se adequar aos dados que foram coletados, os fatores utilizados estão expostos na tabela \ref{table:metricas_de_classificacao_utilizadas}.

Afim de utilizar uma base de dados já existente no contexto do Distrito Federal o fator ``Valores culturais para o empreendedorismo'' foi substituído por ``Cultura empreendedora', dessa forma ao invés de utilizar um dado proveniente do ``\textit{Global Entrepreneurship Index}'', como sugerido pelo grupo InovaSampa e referente ao Brasil como um todo, utilizamos um indicador local produzido pela Endeavor e publicado no Índice de Cidades Empreendedoras. O fator ``Incubadoras e parques tecnológicos'' foi inserido por se tratar de um dado conhecido e que se mostrou relevante durante as entrevistas pela sua contribuição no desenvolvimento e no apoio à novas \textit{startups} no Distrito Federal.

\begin{table}[H]
\centering
\begin{tabular}{ | c | c | c | c | c |}
\hline
\thead{Fator} & \thead{Nascente} & \thead{Crescente} &\thead{Maduro}& \thead{Sustentável} \\
\hline
\makecell{Estratégias\\de saída}&nenhum&poucos&\makecell{várias aquisições\\e fusões mas\\poucos \textit{IPO's}}&\makecell{várias aquisições\\e fusões e\\muitos \textit{IPO's}}\\
\hline
\makecell{Investimento Anjo}&\makecell{irrelevante}&\makecell{irrelevante}  &\makecell{alguns} & \makecell{muitos}    \\
\hline
\makecell{Cultura\\Empreendedora}&\makecell{0-4}&\makecell{4-6}&\makecell{6-8}&\makecell{8-10}\\
\hline
\makecell{Atores da mídia\\com foco no\\empreendedorismo}&\makecell{nenhum}     &   \makecell{alguns}    &    \makecell{muitos} & \makecell{todos}     \\
\hline
\makecell{Dados do ecossistema\\e pesquisas}&\makecell{nenhum}    & \makecell{nenhum} & \makecell{parciais}    & \makecell{completos} \\
\hline 
\makecell{Gerações do\\ecossistema}&\makecell{0}& \makecell{0}     &    \makecell{1-2}     &    \makecell{>= 3}       \\
\hline
\makecell{Eventos}&\makecell{mensais} & \makecell{semanais} & \makecell{diários}  & \makecell{> diários} \\
\hline
\makecell{Incubadoras \\e\\Parques Tecnológicos}    & \makecell{0} &    \makecell{01}     &    \makecell{02-05}   &    \makecell{>5}    \\
\hline
\makecell{Ambiente regulatório}&\makecell{0-3}&\makecell{3-5}&\makecell{5-8}&\makecell{8-10}\\
\hline
\end{tabular}

\caption{Indicadores utilizados na pesquisa}
\label{table:metricas_de_classificacao_utilizadas}
\end{table}

\section{O arcabouço conceitual e o Mapa de um Ecossistema}
\label{subsection:arcabouco_conceitual_e_modelo}

Com base nos mesmos fatores descritos na Seção anterior, na relevância de cada um dos fatores descritos, de acordo com a visão dos empreendedores que compõem o próprio ecossistema e nas informações disponibilizadas por outros pesquisadores o grupo de pesquisadores da Universidade de São Paulo também elaborou um arcabouço conceitual que representa as diversas interações e fatores influenciadores de um ecossistema, representado pela Figura \ref{figure:arcabouco_teorico_de_um_ecossistema}. Nas Figuras \ref{figure:mapa_conceitual_tel_aviv}, \ref{figure:mapa_conceitual_sao_paulo} estão os mapas dos ecossistemas de São Paulo e Tel-Aviv, ambos tendo como base o mesmo arcabouço conceitual. Os três mapas estão disponíveis no Anexo \ref{anexo:mapas_concentuais_do_inovasampa}.

\section{Aplicação da Metodologia e Protocolo}
\label{section:aplicacao_da_metodologia}

A aplicação da Metodologia foi dividida em três etapas:

\begin{enumerate}
  \item Entrevistas e Observações
  \item Codificação dos Dados
  \item Análises e Conclusões
\end{enumerate}

A primeira, de entrevistas e observações, se deu por meio da observação de diversos eventos e espaços com empreendedores que compõem o ecossistema de startups do Distrito Federal e das pessoas que o constroem e por meio de entrevistas individuais com pessoas atuantes no Ecossistema com objetivo de entender seu contexto pessoal e profissional, bem como suas visões sobre a realidade e as dinâmicas do ecossistema como um todo, quais os seus pontos fortes e fracos, seus maiores problemas, como diversas instituições e pessoas interagem entre si afim de fomentá-lo e quais ações poderiam ser tomadas afim de melhorá-lo.

Com a codificação dos dados todas as informações levantadas pela primeira etapa foram catalogadas em tabelas com o objetivo de se tornarem referências para as etapas de análises e conclusões e futuras pesquisas bem como documentar todo o processo que foi realizado. 

Com as análises dos dados e sua adequação nos fatores pré-definidos será possível mensurar a maturidade do ecossistema com o objetivo de gerar as conclusões da pesquisa, que se concentrarão em explicitar o atual estágio do Ecossistema de acordo com a Metodologia utilizada, em realizar comparações com outros ecossistemas e identificar uma série de ações que podem ser tomadas para aprimorar determinados pontos.

\section{Perguntas a serem respondidas}
\label{subsection:questoes_de_pesquisa}

\begin{itemize}
  \item Pergunta 01: Quais são as características socioculturais de Brasília que promovem ou inibem o empreendedorismo?
  \item Pergunta 02: Quais são os mecanismos institucionais de Brasília que promovem ou dificultam o Empreendedorismo?
  \item Pergunta 03: Quais são os mecanismos educacionais de Brasília que promovem o Empreendedorismo?
  \item Pergunta 04: Como aspectos tecnológicos influenciam o sucesso ou fracasso das startups do Distrito Federal? Qual o papel executado pela comunidade e pelo Software Livre?
  \item Pergunta 05: Qual a relação do empreendedor de Brasília com as opções de investimento disponíveis e como elas influenciam o Ecossistema?
  \item Pergunta 06: Quais ações devem ser tomadas no Ecossistema de Brasília para que ele cresça?
\end{itemize}

As quatro primeiras perguntas foram extraídas dos mapeamentos realizados por \citeonline{Kon2014} em Tel-Aviv e \citeonline{MonnaSantos2015} em São Paulo e ajudaram a entender as principais características e dinâmicas do ecossistema de \textit{startups} do Distrito Federal, com a quinta buscamos entender a entrada de capital de investimento e com a sexta pergunta os pontos que devem ser amadurecidos.

\section{Escolha dos Entrevistados}
\label{subsection:escolha_dos_entrevistados}

É importante que os entrevistados sejam atuantes e bem conectados com o ecossistema de startups do Distrito Federal como um todo e, em sua maior parte, empreendedores, mas também devem ser levantadas as visões de outros atores que o compõem como professores, servidores e agentes públicos, investidores, representantes de incubadoras e aceleradoras e estudantes. Ao todo foram realizadas 16 entrevistas com representantes desses grupos.

Assim como sugerido pelos criadores da metodologia, para a escolha dos entrevistados fora aplicada a metodologia ``bola de neve'. Primeiramente, foram definidas algumas pessoas com alto histórico de contribuição e participação no ecossistema e que faziam parte da rede de contatos das pessoas envolvidas com a pesquisa e foram solicitadas recomendações de quais pessoas deveriam fazer parte desta pesquisa e, se possível, solicitado uma introdução entre essas pessoas. Ao fim de cada entrevista esse processo também foi repetido.

\section{Condução das Entrevistas}
\label{subsection:conducao_das_entrevistas}

Foi dada liberdade ao entrevistado de escolher o local da entrevista, em sua maior parte elas aconteceram no próprio local de trabalho da pessoa mas também foram utilizadas cafeterias e softwares de videoconferência como ``\textit{Skype}'' e ``\textit{Google Hangout}''. Todas foram devidamente gravadas em áudio com anuência do empreendedor. Foi tomado cuidado para que as entrevistas não fossem longas, a maior parte não extrapolou o prazo de 40 minutos de duração. 

Para guiar o entrevistador e facilitar a condução foram estabelecidas uma série de perguntas que foram realizadas aos entrevistados com o objetivo de obter respostas que respondam as perguntas estabelecidas na Subseção \ref{subsection:questoes_de_pesquisa}.

Não necessariamente as entrevistas seguiram de forma rígida todas as perguntas sugeridas no roteiro, de acordo com o perfil do entrevistado ou com a evolução das conversas elas foram adaptadas, em muitos casos o próprio entrevistado respondeu algumas delas durante outras perguntas. Como a entrevista foi conduzida ou a linguagem utilizada não são de grande importância, desde que se tenha obtido insumos o suficiente para responder as questões da Subseção \ref{subsection:questoes_de_pesquisa}.

Todas as perguntas e o roteiro sugerido estão disponíveis no Apêndice \ref{apendices:perguntas_das_entrevistas}. Os convites foram realizados via \textit{e-mail} (nos casos de conexão criada por outros empreendedores), redes sociais e em alguns casos pessoalmente. 

\section{Transcrição, Codificação e Interpretação dos Dados}
\label{subsection:codificacao_e_interpretacao_dos_dados}

Após a realização de cada entrevista a transcrição e codificação das entrevistas foi feita utilizando o software ``\textit{MAXQDA}''\footciteref{MaxQDA}, um software proprietário, pago, e que foi escolhido por recomendação de um dos criadores da metodologia, o Dr. Daniel Cukier. 

Embora a codificação das entrevistas realizadas em São Paulo, Nova Iorque e Tel-Aviv tenham sido feitas utilizando a plataforma ``\textit{Google Docs}'', Cukier relatou a dificuldade para se caracterizar os relatos transcritos de acordo com as perguntas a serem respondidas e o tempo despendido desnecessariamente na exploração desses dados por não ter utilizado uma plataforma própria para essa atividade e com uma série de recursos que facilitam a análise, como o ``\textit{MaxQDA}''.

\begin{figure}[!htb]
	\centering
	\includegraphics[width=16cm,angle=0]{figuras/maxqda}
	\caption{Interface de codificação do ``\textit{MAXQDA}}
	\label{figure:maxqda}
\end{figure}

A Figura \ref{figure:maxqda} representa um exemplo de transcrição de uma das entrevistas realizadas no ``\textit{MaxQDA}'', foram criadas etiquetas para cada uma das Perguntas exploradas na Subseção \ref{subsection:questoes_de_pesquisa} e neste ponto se encontra a maior vantagem do software utilizado: graças a estas etiquetas é possível categorizar as transcrições dos textos e resgatar trechos específicos de diversas entrevistas com um motor de busca que permite buscas por palavras-chave, etiquetas ou quaisquer marcações. Dessa forma no momento de compilação dos resultados a tarefa de revisitar apenas os trechos para cada uma das Perguntas da Subseção \ref{subsection:questoes_de_pesquisa} se tornou muito mais fácil.