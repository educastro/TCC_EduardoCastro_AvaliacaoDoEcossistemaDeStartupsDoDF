\chapter[Conclusões Pré-Eliminares]{Conclusões}
\label{cap-conclusoes}

Sociocultural

==== Participar da criação de uma empresa é uma atividade de altíssimo risco, \citeonline{haswell1989estimating} estima que cerca de 70\% das novas empresas não conseguem sobreviver aos dois primeiros anos de vida
==== (relacionar com o livro startup ecosystem e os hubs, da pra relacionar com rainforest tb, na parte sobre afastamento entre ecossistema, empresas, etc)
==== A galera queria algo orgânico, não uma associação. (relacionar com critica a asteps)

Institucional
==== falar da abertura de empresa em 5 dias


A exploração do ecossistema de startups de Brasília tem sido deveras interessante, primeiro pela extensa rede de contatos que venho contruindo graças à essa pesquisa mas também por em tão pouco tempo já permitir uma visualização superficial e uma compreensão de diversos elementos desse Ecossistema e a razão de certas características peculiares. Esse também tem sido uma das primeiras, se não a primeira, pesquisa com foco nas startups de Brasília, o que pode ser de grande utilidade para outros atores e o início de diversos outros trabalhos. O próprio resultado esperado dessa pesquisa já despertou o interesse de muitos atores envolvidos.

A descoberta de outros pesquisadores envolvidos com a temática de ecossistemas e o contato com outros trabalhos similares também tem sido proveitosa, com os devidos ajustes em alguns dos fatores acredito que a metodologia utilizada por esse trabalho pode se mostrar uma das melhores opções disponíveis para avaliação de Ecossistemas e capaz de gerar uma boa base de dados e comparações entre as cidades, como a \citeonline{indiceglobaldoempreendedorismo} já é no Brasil mas com uma abordagem diferente e mais próxima das pessoas e das realidades locais. A abordagem qualitativa também tem se mostrado um grande diferencial de tal forma que a própria metodologia e o pesquisador se adaptam e evoluem conforme seu avanço.
