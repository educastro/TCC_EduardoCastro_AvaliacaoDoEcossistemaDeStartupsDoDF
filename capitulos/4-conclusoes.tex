\chapter[Conclusões]{Conclusões}
\label{cap-conclusoes}

\section{Contexto atual do ecossistema}
\label{contexto_atual_do_ecosssistema}

De acordo com os relatos claramente a figura de uma forte liderança e de um ecossistema unido e integrado era bem clara entre meados de 2010 e 2012, diversos empreendedores citaram um grupo chamado ``Startup Brasília'' composto por empreendedores de empresas como a Intacto, Qual Canal, SEA Tecnologia, Rota dos Concursos, IPê Tecnologia, Trip2gether, etc. Muitos dos participantes desse grupo e do ecossistema em 2012 estão representados na Figura \ref{figure:startups_board_2012}, criada pelo empreendedor Marcos Oliveira. Esse grupo foi responsável por organizar encontros, startup weekends, um ciclo de capacitação chamado startup dojo, etc.

\begin{figure}[!htb]
	\centering
	\includegraphics[width=15cm,angle=0]{figuras/startups_board_2012}
	\caption{Representantes do Ecossistema de Startups de Brasília em 2012}
	\label{figure:startups_board_2012}
\end{figure} 

A união era tamanha que mesmo sem um grande canal para investimentos na cidade um dos empreendedores mencionou que foi uma época em que Brasília entrou no radar como um dos melhores ecossistemas de startups do Brasil. Muitos empreendedores citaram o constante apoio do SEBRAE DF como essencial para o crescimento na época, esse apoio foi essencial para que, em 2012, Brasília tivesse a maior delegação brasileira no Tech Crunch Disrupt, momento que ainda é mencionado por muitos como um grande marco do ecossistema. Nessa mesma época, por incentivo de alguns do grupo, houve uma rodada de aceleração da Startup Farm\footciteref{StartupFarm} em Brasília, uma das maiores aceleradoras de startups da América Latina. Sem dúvidas são pontos que demonstram o poder que uma liderança unida e coesa pode ter em um ecossistema de startups.

\citeonline{Feld2012} entende como líder a pessoa, ou grupo de pessoas, que domine a dinâmica do ecossistema e seja responsável por \"quebrar as barreiras invisíveis\" entre atores, \citeonline{Hwang2012} também entende como pontos chaves de um bom ecossistema os líderes capazes de estimular conexões entre o ecossistema. Partindo dessa definição é possível entender que o grupo \"Startup Brasília\" era formado por líderes, ou \"keystones\", do Distrito Federal e são exemplos de grande potencial que uma liderança unida e coesa pode ter sobre um ecossistema.

Infelizmente, segundo alguns empreendedores, esse foi o auge. Quando foi levantada a possibilidade de ser criada uma associação que melhor representasse o ecossistema, principalmente perante ao Estado, o grupo aparentava já estar \"quebrado\" e coincidentemente após esse momento alguns dos líderes do ecossistema deixaram o país para se dedicarem as suas startups no Vale do Silício, também foi quando a pessoa chave no Sebrae responsável pela área de startups também logo fora transferida para outro setor. 

Esse foi o momento em que o ecossistema do Distrito Federal começou a perder sua força, entre meados de 2012 e 2014. Também foi quando nasceu a Associação de Startups e Empreendedores Digitais (ASTEPS), criada por outro grupo de empreendedores que, por meio desta, tentou tomar a posição de liderança e referência do ecossistema mas sem o apoio, envolvimento e confiança dos líderes desse antigo grupo. A impressão repassada pelos empreendedores entrevistados é de que esse momento não foi orgânico/natural, de que não havia um consenso sobre o que deveria ser feito e de que não foram geradas novas lideranças no momento em que os antigos atores se afastaram, o que a bibliografia chama de \"spin-offs\". Claramente esse fora um ponto de ruptura.

Traçando um paralelo histórico do que alguns empreendedores apontaram como momentos cruciais para o ecossistema de startups do Distrito Federal esses pontos de quebra se tornam bem claros, para descrever os momentos é possível dizer que os pontos chaves foram: 1) Ida da Rota dos Concursos para aceleração na 500 Startups, a primeira startup brasileira aceita no programa, em 2011; 2) Fortalecimento do grupo \"Startup Brasília\" entre 2011 e 2012; 3) Nascimento da Associação de Startups e Empreendedores Digitais (ASTEPS) em 2012; 4) Anúncio da venda da startup brasiliense ZeroPaper para a empresa Intuit, em 2015; e 5) Entrada de investidores e aceleradoras no mercado local em 2016. 

\subsection{Sociocultural}
\label{subsection:pergunta_de_pesquisa_2}

%Sociocultural

%falar sobre pilar de cultura empreendedora, universidade melhor fase para empreender e atuação das universidades

%==== Participar da criação de uma empresa é uma atividade de altíssimo risco, \citeonline{haswell1989estimating} estima que cerca de 70\% das novas empresas não conseguem sobreviver aos dois primeiros anos de vida
%==== (relacionar com o livro startup ecosystem e os hubs, da pra relacionar com rainforest tb, na parte sobre afastamento entre ecossistema, empresas, etc)
%==== A galera queria algo orgânico, não uma associação. (relacionar com critica a asteps)

%Institucional
%==== falar da abertura de empresa em 5 dias

\subsection{Educacional}
\label{subsection:Educação}

\subsection{Institucional}
\label{subsection:Institucional}

\subsection{Tecnologia}
\label{subsection:Tecnologia}

\subsection{Investimento}
\label{subsection:Investimento}

%abreu2016panorama investimento fgv

%TRAZER PESQUISA DA FGV SOBRE ACELERADORAS

\section{Considerações Finais}
\label{subsection:consideracoes_finais}