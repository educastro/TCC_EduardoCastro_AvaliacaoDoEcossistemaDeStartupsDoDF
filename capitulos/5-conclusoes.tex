\chapter[Conclusões Pré-Eliminares]{Conclusões}
\label{cap-conclusoes}

A exploração do ecossistema de startups de Brasília tem sido deveras interessante, primeiro pela extensa rede de contatos que venho contruindo graças à essa pesquisa mas também por em tão pouco tempo já permitir uma visualização superficial e uma compreensão de diversos elementos desse Ecossistema e a razão de certas características peculiares. Esse também tem sido uma das primeiras, se não a primeira, pesquisa com foco nas startups de Brasília, o que pode ser de grande utilidade para outros atores e o início de diversos outros trabalhos. O próprio resultado esperado dessa pesquisa já despertou o interesse de muitos atores envolvidos.

A descoberta de outros pesquisadores envolvidos com a temática de ecossistemas e o contato com outros trabalhos similares também tem sido proveitosa, com os devidos ajustes em alguns dos fatores acredito que a metodologia utilizada por esse trabalho pode se mostrar uma das melhores opções disponíveis para avaliação de Ecossistemas e capaz de gerar uma boa base de dados e comparações entre as cidades, como a \citeonline{indiceglobaldoempreendedorismo} já é no Brasil mas com uma abordagem diferente e mais próxima das pessoas e das realidades locais. A abordagem qualitativa também tem se mostrado um grande diferencial de tal forma que a própria metodologia e o pesquisador se adaptam e evoluem conforme seu avanço.

\section{Planejamento para TCC2}
\label{section:cronograma_tcc2}

Por meio de um levantamento inicial e indicações de alguns dos Empreendedores selecionados foram mapeadas as seguintes pessoas que poderiam contribuir com o desenvolvimento deste trabalho, os nomes estão indicados nas Tabelas \ref{table:sugestao_de_empreendedores_para_entrevista}, \ref{table:sugestao_de_coworkings_para_entrevista}, \ref{table:sugestao_de_instituicoes_para_entrevista}, \ref{table:sugestao_de_investidores_para_entrevista} e \ref{table:sugestao_de_universidades_para_entrevista} disponíveis no Apêndice \ref{apendices:tabelas_de_entrevistados}, cada tabela para sua respectiva Categoria. 

De acordo com relatos de outros pesquisadores e pelo trabalho com a primeira entrevista formal estimo que cada hora de entrevista gere um trabalho de 3 a 5 horas de transcrição, além do tempo de codificação e interpretação de dados. Com o uso de softwares como o MaxQDA\footciteref{MaxQDA} esse trabalho se torna bem mais fácil, embora ainda bastante custoso em questão de tempo. 

Portanto, a meta é que seja realizada uma entrevista a cada duas semanas, e que sua respectifiva transcrição e codificação seja feita nesse período intermediário. Ao todo já foram listados cerca de 53 nomes e a previsão é que sejam feitas cerca de 20 entrevistas, essa lista de Empreendedores sugeridos pode aumentar conforme as entrevistas evoluírem e novos nomes sejam sugeridos. O gráfico de Gantt do planejamento está representado na Figura \ref{figure:gantt_chart_2016_2017}.

\begin{figure}[!htb]
	\centering
	\includegraphics[width=23cm, height=10cm, angle=270]{figuras/gantt_chart_2016_2017}
	\caption{Gráfico de Gantt representando o planejamento deste trabalho até 2017}
	\label{figure:gantt_chart_2016_2017}
\end{figure}

\subsection{Descrição das Atividades}
\label{subsection:descricao_das_atividades}

\begin{description}
  \item [Entrevistas:] Essa atividade consiste nas entrevistas com os Empreendedores como descrito em \ref{subsection:conducao_das_entrevistas} e no Apêndice \ref{apendices:perguntas_das_entrevistas}.

  \item [Transcrição:] Atividade realizada com auxílio do software MaxQDA, consiste em ouvir os áudios das entrevistas para que seja feita a transcrição para futura análise. 

  \item [Interpretação:] Estudo das transcrições realizadas para devida codificação e interpretação, também realizada com auxílio do software MaxQDA, com o objetivo de cruzar dados entre as entrevistas e obter um entendimento geral das informações obtidas.

  \item [Elaboração dos Resultados:] Elaboração do capítulo de resultados deste trabalho, expondo todo o aprendizado obtivo e resposta para as questões de pesquisa levantadas.

  \item [Criação do Mapa Conceitual:] Após um levantamento e interpretação de dados seremos capazes de elaborar um mapa conceitual sobre o ecossistema estudado.

  \item [Revisão:] Fase final do projeto para ajustes finais e correções antes da entrega final.
\end{description}

