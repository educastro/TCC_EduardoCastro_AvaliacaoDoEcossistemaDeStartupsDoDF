\chapter[Conclusões Pré-Eliminares]{Conclusões}
\label{cap-conclusoes}

A exploração do Ecossistema de Startups de Brasília tem sido deveras interessante, primeiro pela extensa rede de contatos que venho contruindo graças à essa pesquisa mas também por em tão pouco tempo já permitir uma visualização superficial e uma compreensão de diversos elementos desse Ecossistema e a razão de certas características locais.

Traçando uma espécie de linha temporal do Ecossistema de Startups do Distrito Federal, claramente a figura de uma forte liderança e de um Ecossistema unido e integrado era mais clara entre meados de 2010 e 2012, diversos Empreendedores citaram um grupo chamado ``Startup Brasília'' composto por Empreendedores de empresas como a Intacto, Qual Canal, SEA Tecnologia, Rota dos Concursos, IPê Tecnologia, Trip2gether, etc. 

Nessa época os encontros - também conhecidos como meetups - também eram mais frequentes e, segundo alguns Empreendedores, de maior qualidade. Foi relatado que os representantes do Startup Brasília muitas vezes atraiam palestrantes que eram referência em todo o Brasil e pagavam do próprio bolso para que víessem para atrair os Empreendedores, eram líderes natos do Ecossistema.

Mesmo sem um grande canal para investimentos na cidade um dos Empreendedores mencionou que foi uma época em que Brasília entrou no radar como um dos melhores Ecossistemas de Startups do Brasil. A presença do Governo também era forte, muitos Empreendedores citaram o constante apoio do SEBRAE DF como essencial para o crescimento do Ecossistema na época. A união entre o SEBRAE e o grupo Startup Brasília foi o principal ponto para que em 2012 Brasília tivesse a maior delegação Brasileira no Tech Crunch Disrupt, um grande marco do nosso Ecossistema. Nessa mesma época também houve uma edição do Startup Farm em Brasília, atraida por esses atores.

Infelizmente, segundo alguns Empreendedores, esse foi o auge da cidade. Chegou a ser discutida a possibilidade de ser criada uma Associação que melhor representasse o Ecossistema, principalmente perante ao Estado, mas o grupo optou por não seguir esse caminho e coincidentemente após esse momento alguns dos líderes do Ecossistema deixaram o país, alguns para serem acelerados no Vale do Silício, e a pessoa chave no Sebrae responsável pelas Startups também fora transferida para outra área. 

Esse é o ponto em que um dos Empreendedores classifica como o momento em que o Ecossistema do Distrito Federal começou a perder sua força, entre meados de 2012 e 2014. Também foi quando nasceu a Associação de Startups e Empreendedores Digitais, criada por outro grupo de Empreendedores, e naturalmente tomou a posição de liderança e referência do Ecossistema mas sem o apoio e a confiança desse antigo grupo.

Um dos Empreendedores menciona que entre 2015 e 2016, o Ecossistema de Startups de Brasília voltou a reagir. Ele não sabe dizer o que encadeou o movimento, mas diz que o cenário definitivamente não é mais o mesmo e voltou a crescer. Com a presença de dois grandes atores de investimento trazendo cerca de R\$ 100 milhões para Startups e três programas de aceleração se estabelecendo esse Empreendedor acredita que o Ecossistema de Brasília logo voltará a ser o que era se os atores se unirem e formarem um grupo forte novamente.

\section{Cronograma para TCC2}
\label{section:cronograma_tcc2}

\begin{table}[!htb]
	\centering
	\label{tabela:sugestao_de_empreendedores_para_entrevista}
	\begin{tabular}{ | p{3cm} | p{8cm} | p{4cm} | }
		\hline
		Categoria & Nome & Empresa \\ \hline
		Empreendedor & Alexandre Gomes & \\ \hline
		Empreendedor & André Eloy & São \\ \hline
		Empreendedor & André Macedo & ZeroPaper \\ \hline
		Empreendedor & Arthur Furlan & Configr \\ \hline
		Empreendedor & Bruno Kenj & Owl Docs \\ \hline
		Empreendedor & Bruno Rossi & Apetitar \\ \hline
		Empreendedor & Bruno Torquato & PDVend \\ \hline
		Empreendedor & Daniel Bordin & Mirante  \\ \hline
		Empreendedor & Daniel Sandoval & Loop \\ \hline
		Empreendedor & Fabrício Buzeto & Buzeto Tecnologia \\ \hline
		Empreendedor & Fernando Aquino & MovaMais \\ \hline
		Empreendedor & Flávio & Startaê \\ \hline	
		Empreendedor & Flávio Fonseca & Novatics \\ \hline
		Empreendedor & Gustavo Goreinstein & Poup \\ \hline	
		Empreendedor & Henrique Santana & Integrah \\ \hline
		Empreendedor & Iuri Costa & Axies \\ \hline
		Empreendedor & Jéssica Behrens & Tradr \\ \hline
		Empreendedor & Joaquim Venâncio & Ticies \\ \hline
		Empreendedor & Luis Sampaio & Preditiva  \\ \hline
		Empreendedor & Maximiliano ou Jens & WriteWork.com \\ \hline
		Empreendedor & Michele Protzek & Flama \\ \hline
		Empreendedor & Pedro Salum & Loop \\ \hline
		Empreendedor & Rafael & Startaê \\ \hline	
		Empreendedor & Renato & Startaê \\ \hline	
		Empreendedor & Ricardo & Funnifier \\ \hline
		Empreendedor & Roberto Mascarenhas & IPê \\ \hline
		Empreendedor & Saulo Camarotti & Behold Studios \\ \hline	
	\end{tabular}
	\caption{Mapeamento de Empreendedores para serem entrevistados}
\end{table}

\begin{table}[!htb]
	\centering
	\label{tabela:sugestao_de_coworkings_para_entrevista}
	\begin{tabular}{ | p{3cm} | p{8cm} | p{4cm} | }
		\hline
		Categoria & Nome & Empresa \\ \hline
		CW/Incu/Acel & Cris & Multiplicidade \\ \hline
		CW/Incu/Acel & Érika Lisboa & Casulo \\ \hline
		CW/Incu/Acel & Fernando Santiago & 4Legal \\ \hline
		CW/Incu/Acel & Hélio & Acceleratus \\ \hline
		CW/Incu/Acel & Heloísa & 4Legal \\ \hline
		CW/Incu/Acel & Juliana Guimarães & 4Legal \\ \hline
		CW/Incu/Acel & Mariana & Techmall \\ \hline
		CW/Incu/Acel & Wesley & Cotidiano \\ \hline
	\end{tabular}
	\caption{Mapeamento de Coworkings/Incubadoras/Aceleradoras para serem entrevistados}
\end{table}

\begin{table}[!htb]
	\centering
	\label{tabela:sugestao_de_instituicoes_para_entrevista}
	\begin{tabular}{ | p{3cm} | p{8cm} | p{4cm} | }
		\hline
		Categoria & Nome & Empresa \\ \hline
		
	\end{tabular}
	\caption{Mapeamento de Instituições de Apoio para serem entrevistados}
\end{table}

\begin{table}[!htb]
	\centering
	\label{tabela:sugestao_de_investidores_para_entrevista}
	\begin{tabular}{ | p{3cm} | p{8cm} | p{4cm} | }
		\hline
		Categoria & Nome & Empresa \\ \hline
		
	\end{tabular}
	\caption{Mapeamento de Investidores para serem entrevistados}
\end{table}

\begin{table}[!htb]
	\centering
	\label{tabela:sugestao_de_universidades_para_entrevista}
	\begin{tabular}{ | p{3cm} | p{8cm} | p{4cm} | }
		\hline
		Categoria & Nome & Empresa \\ \hline
		
	\end{tabular}
	\caption{Mapeamento de Representantes das Universidades para serem entrevistados}
\end{table}

\begin{table}[!htpb]
	\centering

	% definindo o tamanho da fonte para small
	% outros possíveis tamanhos: footnotesize, scriptsize
	\begin{small} 
	  
		% redefinindo o espaçamento das colunas
		\setlength{\tabcolsep}{3pt} 

		% \cline é semelhante ao \hline, porém é possível indicar as colunas que terão essa a linha horizontal
		% \multicolumn{10}{c|}{Meses} indica que dez colunas serão mescladas e a palavra Meses estará centralizada dentro delas.

		\begin{tabular}{|p{5cm}|p{2cm}|p{2cm}|p{2cm}|p{2cm}|p{2cm}|}\hline
		 & \multicolumn{5}{c|}{Cronograma 02/2016}\\ \cline{2-6}
		\raisebox{1.5ex}{Entrevistas} & Ago/2016 & Set/2016 & Out/2016 & Nov/2016 & Dez/2016 \\ \hline

		1 & X & X & X & X & X \\ \hline
		2 & X & X & X & X & X \\ \hline
		3 & X & X & X & X & X \\ \hline
		4 & X & X & X & X & X \\ \hline
		5 & X & X & X & X & X \\ \hline
		6 & X & X & X & X & X \\ \hline
		7 & X & X & X & X & X \\ \hline
		8 & X & X & X & X & X \\ \hline
		9 & X & X & X & X & X \\ \hline
		10 & X & X & X & X & X \\ \hline


		\end{tabular} 
	\end{small}
	\caption{Cronograma das atividades previstas}
	\label{t_cronograma}
\end{table} 

\begin{table}[!htpb]
	\centering

	% definindo o tamanho da fonte para small
	% outros possíveis tamanhos: footnotesize, scriptsize
	\begin{small} 
	  
		% redefinindo o espaçamento das colunas
		\setlength{\tabcolsep}{3pt} 

		% \cline é semelhante ao \hline, porém é possível indicar as colunas que terão essa a linha horizontal
		% \multicolumn{10}{c|}{Meses} indica que dez colunas serão mescladas e a palavra Meses estará centralizada dentro delas.

		\begin{tabular}{|p{5cm}|p{2cm}|p{2cm}|p{2cm}|p{2cm}|p{2cm}|}\hline
		 & \multicolumn{5}{c|}{Cronograma 01/2017}\\ \cline{2-6}
		\raisebox{1.5ex}{Entrevistas} & Jan/2017 & Fev/2017 & Mar/2017 & Abr/2017 & Mai/2017 \\ \hline

		11 & X & X & X & X & X \\ \hline
		12 & X & X & X & X & X \\ \hline
		13 & X & X & X & X & X \\ \hline
		14 & X & X & X & X & X \\ \hline
		15 & X & X & X & X & X \\ \hline
		16 & X & X & X & X & X \\ \hline
		17 & X & X & X & X & X \\ \hline
		18 & X & X & X & X & X \\ \hline
		19 & X & X & X & X & X \\ \hline
		20 & X & X & X & X & X \\ \hline

		\end{tabular} 
	\end{small}
	\caption{Cronograma das atividades previstas}
	\label{t_cronograma}
\end{table} 