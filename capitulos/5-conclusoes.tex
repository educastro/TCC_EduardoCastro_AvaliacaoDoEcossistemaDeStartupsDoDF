\chapter[Conclusões]{Conclusões}
\label{cap-conclusoes}

\section{O atual momento do ecossistema de startups do DF}
\label{contexto_atual_do_ecosssistema}

\citeonline{Feld2012} entende como líder a pessoa, ou grupo de pessoas, que domine a dinâmica do ecossistema e seja responsável por "quebrar as barreiras invisíveis" entre atores, \citeonline{Hwang2012} também entende como pontos chaves de um bom ecossistema os líderes capazes de estimular conexões entre o ecossistema. Partindo dessa definição é possível entender que o grupo "Startup Brasília" era formado por líderes, ou "keystones", do Distrito Federal e são exemplos de grande potencial que uma liderança unida e coesa pode ter sobre um ecossistema. Também foi possível identificar que a maior fraqueza do ecossistema estudado é justamente a falta de lideranças claras e atuantes conectando os mais diversos atores participantes e "orquestrando" essas interações em prol do desenvolvimento de novos negócios, parcerias, etc.

Nota-se pelos resultados obtidos com as entrevistas que o ecossistema de startups do Distrito Federal obteve grande destaque em meados de 2010 a 2012, fruto dessa forte união entre empreendedores e entidades de apoio, mas entre 2012 a 2014 passou por um período de enfraquecimento. Também foi quando nasceu a Associação de Startups e Empreendedores Digitais (ASTEPS), criada por outro grupo de empreendedores que, por meio desta, tenta se tornar uma liderança e referência do ecossistema mas sem o apoio, envolvimento e confiança dos líderes desse antigo grupo, demonstrando um forte conflito dentro da comunidade. A impressão repassada pelos empreendedores entrevistados é de que esse momento não foi orgânico/natural, de que não havia um consenso sobre o que deveria ser feito e de que não foram geradas novas lideranças no momento em que os antigos atores se afastaram, o que a bibliografia define como "spin-offs". Claramente esse fora um ponto de ruptura.

Traçando um paralelo histórico do que alguns empreendedores apontaram como momentos cruciais para o ecossistema de startups do Distrito Federal esses pontos de quebra se tornam bem claros, para exemplificar o contexto histórico do ecossistema de startups do Distrito Federal é possível definir que seus pontos chaves foram: 1) Ida da Rota dos Concursos para aceleração na 500 Startups, a primeira startup brasileira aceita no programa, em 2011; 2) Fortalecimento do grupo "Startup Brasília" entre 2011 e 2012; 3) Nascimento da Associação de Startups e Empreendedores Digitais (ASTEPS) em 2012; 4) Anúncio da venda da startup brasiliense ZeroPaper para a empresa Intuit, em 2015; e 5) Entrada de investidores e aceleradoras no mercado local em 2016. 

O ponto 1) se mostrou importante por ter sido o primeiro passo na internacionalização das startups do Distrito Federal, a aproximação de seus fundadores ao Vale do Silício abriu portas para outros empreendedores brasileiros e, em especial, brasilienses. Em conjunto com o ponto 2) esses dois elementos foram essenciais para a espressiva delegação do Distrito Federal no Tech Crunch Disrupt 2012, a maior do Brasil, e diversas outras ações de fomento ao ecossistema local, como edições do evento "Startup Weekend". Por diversos dos empreendedores entrevistados o ponto 3) foi tratado como um ponto de virada negativo do ecossistema e os pontos 4 e 5 como pontos de virada positivos e a representação de que o Distrito Federal tem potencial para voltar a ganhar destaque. 

Essa falta de lideranças bem definidas e atuantes no Distrito Federal também mostrou sinais de impactar a atuação das aceleradoras e grandes empresas por não haver sinais fortes de comunicação e interação entre esses atores, o SEBRAE DF e a ASTEPS por algum tempo fomentaram reuniões mensais entre alguns dos atores do ecossistema do Distrito Federal mas conforme os encontros foram acontecendo e os resultados não foram se desenvolvendo os encontros deixaram de acontecer. A percepção obtida é de que esse grupo poderia demonstrar um grande potencial de movimentação no ecossistema local de diversas maneiras como, por exemplo, abrir um momento para que cada ator demonstre suas ações no ecossistema (e impeça situações com bons eventos acontecendo de forma simultânea, algo recorrente no Distrito Federal) e abertura para que esses atores colaborem entre si para um maior sucesso dessas ações. Uma forte resistência para a criação desses momentos entre os atores do Distrito Federal também se tornou clara, talvez fruto dos baixos resultados, falta de confiança e/ou de conflitos pessoais.

Segundo relatos o brasiliense também parece optar pela estabilidade do emprego fixo ao invés do risco, sendo este um ponto criticado por vários dos entrevistados, e reflexo para uma baixa cultura empreendedora, este talvez seja o indicador do Distrito Federal que requer uma maior concentração de esforços para que melhore, em especial entre jovens e estudantes universitários. 

Ações para o fortalecimento da cultura empreendedora trabalhados de forma alinhada e conectada aos diversos atores que compõem o ecossistema empreendedor do Distrito Federal podem formar conexões entre empreendedores em busca de novos funcionários ou sócios, fomentar a criação de novos negócios e parcerias e uma cultura de colaboração entre as empresas e os empreendedores locais, disseminar as histórias de sucesso e inspirar e motivar novas gerações de empreendedores, criar situações em que novos empreendedores recebem apoio ou mentoria de empreendedores experientes, investidores tem a oportunidade de visualizar a movimentação do ecossistema em busca de investimentos promissores, professores universitários podem estar alinhados com a expectativa dessas empresas para profissionais e usufruir do potencial do ecossistema durante seus trabalhos de exploração científica. Com as entrevistas notou-se que "meetups", "startup weekends", espaços de "coworking", eventos e feiras de tecnologia e empreendedorismo, palestras, cursos e oficinas e momentos em que a comunidade possa participar do cotidiano das universidades, das aceleradoras e/ou das startups são capazes de contribuir para a formação e o fortalecimento dessas conexões. Como \citeonline{Hwang2012} descreve em seu livro, uma das características de ecossistemas frutíferos envolve o alto número de conexões que acontece de forma frequente, muitas vezes com contribuição de seus líderes, e quanto mais espaços e momentos para que essas pessoas possam se juntar melhor para o ecossistema como um todo.

Um dos acontecimentos que trazem otimismo ao cenário universitário do Distrito Federal são as instituições que demonstram esforços, ainda que incipientes, para tentar criar um ambiente favorável ao empreendedorismo. Quatro das grandes instituições de ensino superior do Distrito Federal contam com incubadoras de empresas em sua estrutura, atraem empreendedores para atuarem como mentores e realizar palestras, possuem programas educacionais e disciplinas com foco em formação de capacidades empreendedoras, etc. O IESB, por exemplo, promove "meetups" semanais abertos ao público e oferece uma espécie de "coworking" para empreendedores em um espaço chamado de "IESB Lab", lá também já promoveram ciclos de pré-aceleração de startups em parceria com empreendedores e aceleradoras do DF. Outros destaques mencionados como grande destaques foram o UniCeub, com o programa de aceleração de startups Impulso, e a Universidade Católica de Brasília, com o programa de formação de desenvolvedores de aplicativos para dispositivos móveis BEPID.

As entrevistas também demonstraram o potencial do Distrito Federal por se tratar do centro governamental do Brasil, podendo se tornar um cenário favorável tanto no sentido burocrático e de ambiente regulatório como também no sentido mercadológico, devido a presença de grandes compradores (entidades governamentais, organismos internacionais, bancos, grandes empresas prestadoras de serviço, etc) e possíveis investidores. Foi observado também que se trata de um bom mercado consumidor, tendo o maior PIB per capita, ótimos indicadores educacionais e os maiores índices de conectividade de banda larga e teledensidade do país.

Um ponto que foi observado como dicotômico nas entrevistas refere-se a atuação dos investidores no ecossistema local, enquanto observado que empreendedores mais experientes, e os próprios investidores, relatam que há capital em abundância disponível no Distrito Federal para investimento mas poucos empreendedores preparados e maduros o suficiente para receber aportes financeiros os empreendedores com pouca experiência relatam a grande dificuldade para obter um investimento anjo e conseguir acesso a potenciais investidores. Mais de um entrevistado relatou que o empreendedor do DF não está preparado para mostrar o que o investir quer e precisa ver para sentir segurança e confiança no investimento de alto risco, muito ainda se fala sobre ideias e propostas abstratas e não validadas, produtos e futuros idealizados e pouco sobre dados reais, indicadores de crescimento, retenção de usuários, faturamento, maturidade e sinergia da equipe, etc. Esse também é um ponto que merece forte atenção para que seja trabalhado entre os empreendedores de forma que estejam mais preparados para lidar com potenciais investidores e, juntamente com o desenvolvimento de uma maior cultura empreendedora do Distrito Federal e a formação de lideranças capazes de formar induzir conexões de qualidade, diminuir as barreiras que distanciam investidores e empreendedores.

O modelo de trabalho das tradicionais prestadoras de serviço e fábricas de software para o governo também parecem trazer impactos negativos na formação dos profissionais de tecnologia no ecossistema estudado, foi observado pelas entrevistas dificuldades em contratar profissionais preparados para trabalhar em ambientes dinâmicos, flexíveis e ágeis como startups. Um destaque muito elogiado por alguns dos empreendedores foi a Universidade de Brasília (UnB) pela formação de alunos com uma forte experiência em gestão de projetos ágeis e preparo em linguagens modernas para desenvolvimento web e móvel (Ruby on Rails, Ionic, React, Swift, Android, etc.). O Movimento Empresa Junior, muito forte na UnB, também foi mencionado como uma boa fonte de profissionais e possíveis sócios no ambiente universitário.

\subsection{Limitações}
\label{subsection:limitações}

A principal limitação encontrada durante a execução desse trabalho fora o baixo número de dados disponíveis sobre o ecossistema de startups do Distrito Federal, uma das características de ecossistemas com um baixo grau de maturidade apontada por \citeonline{Kon2014}, a metodologia proposta pelo pesquisador também toma como referência para a cidade de São Paulo alguns indicadores nacionais, o que pode refletir bem a realidade de um ecossistema forte para padrões nacionais como o de São Paulo mas não se mostraram condizentes com ecossistemas menos desenvolvidos, como o do Distrito Federal.

Essa limitação refletiu na supressão de 13 indicadores sugeridos pela metodologia completa e impediu a mensuração de um nível de maturidade geral para o ecossistema de acordo com as regras definidas por \citeonline{Kon2014} mas não impactou na exploração qualitativa do estudo.

Como sugestão para amadurecimento da metodologia proposta por Kon e uma melhor aplicação em ecossistemas brasileiros, sugere-se a utilização de indicadores que tenham como referência dados locais e não nacionais, como é caso do indicador "Valores culturais para o empreendedorismo" que neste trabalho foi substituído pelo indicador "Cultura Empreendedora" que tem como base o Índice de Cidades Empreendedoras, da Endeavor, e um foco maior na versão enxuta por melhor se adequar a realidades com poucos dados e estudos disponíveis. 

Com foco na versão enxuta, o indicador "Empreendedorismo nas Universidades", por exemplo, tem como base a porcentagem de ex-alunos universitários que fundaram uma empresa em até 5 anos após a graduação, este dado por si só justifica uma pesquisa e um estudo e publicações dedicadas, infelizmente não cabendo ao âmbito dessa pesquisa e provavelmente das que irão se propor a usar essa versão por estarem em busca de uma análise breve e rápida do ecossistema em estudo. Em contrapartida, os indicadores "Cultura Empreendedora", como mencionado anteriormente, "Incubadoras e Parques Tecnológicos" e "Ambiente Regulatório", provenientes da versão completa, podem ser facilmente adicionados a versão enxuta e explorados por ecossistemas brasileiros menos maduros por tratarem de dados de fácil obtenção e/ou que são monitorados anualmente pela equipe responsável pelo Índice de Cidades Empreendedoras, da Endeavor.

\subsection{Trabalhos futuros}
\label{subsection:trabalhos_futuros}







%Sociocultural

%falar sobre pilar de cultura empreendedora, universidade melhor fase para empreender e atuação das universidades

%==== Participar da criação de uma empresa é uma atividade de altíssimo risco, \citeonline{haswell1989estimating} estima que cerca de 70\% das novas empresas não conseguem sobreviver aos dois primeiros anos de vida
%==== (relacionar com o livro startup ecosystem e os hubs, da pra relacionar com rainforest tb, na parte sobre afastamento entre ecossistema, empresas, etc)
%==== A galera queria algo orgânico, não uma associação. (relacionar com critica a asteps)

%Institucional
%==== falar da abertura de empresa em 5 dias

%abreu2016panorama investimento fgv

%TRAZER PESQUISA DA FGV SOBRE ACELERADORAS