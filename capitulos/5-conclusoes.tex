\chapter[Conclusões Pré-Eliminares]{Conclusões}
\label{cap-conclusoes}

A exploração do Ecossistema de Startups de Brasília tem sido deveras interessante, primeiro pela extensa rede de contatos que venho contruindo graças à essa pesquisa mas também por em tão pouco tempo já permitir uma visualização superficial e uma compreensão de diversos elementos desse Ecossistema e a razão de certas características locais.

Traçando uma espécie de linha temporal do Ecossistema de Startups do Distrito Federal, claramente a figura de uma forte liderança e de um Ecossistema unido e integrado era mais clara entre meados de 2010 e 2012, diversos Empreendedores citaram um grupo chamado ``Startup Brasília'' composto por Empreendedores de empresas como a Intacto, Qual Canal, SEA Tecnologia, Rota dos Concursos, IPê Tecnologia, Trip2gether, etc. 

Nessa época os encontros - também conhecidos como meetups - também eram mais frequentes e, segundo alguns Empreendedores, de maior qualidade. Foi relatado que os representantes do Startup Brasília muitas vezes atraiam palestrantes que eram referência em todo o Brasil e pagavam do próprio bolso para que víessem para atrair os Empreendedores, eram líderes natos do Ecossistema.

Mesmo sem um grande canal para investimentos na cidade um dos Empreendedores mencionou que foi uma época em que Brasília entrou no radar como um dos melhores Ecossistemas de Startups do Brasil. A presença do Governo também era forte, muitos Empreendedores citaram o constante apoio do SEBRAE DF como essencial para o crescimento do Ecossistema na época. A união entre o SEBRAE e o grupo Startup Brasília foi o principal ponto para que em 2012 Brasília tivesse a maior delegação Brasileira no Tech Crunch Disrupt, um grande marco do nosso Ecossistema. Nessa mesma época também houve uma edição do Startup Farm em Brasília, atraida por esses atores.

Muitos dos participantes desse grupo e do Ecossistema em 2012 estão representados na Figura \ref{figure:startups_board_2012}, criada pelo Empreendedor Marcos Oliveira.

\begin{figure}[!htb]
	\centering
	\includegraphics[width=15cm,angle=0]{figuras/startups_board_2012}
	\caption{Representantes do Ecossistema de Startups de Brasília em 2012}
	\label{figure:startups_board_2012}
\end{figure}

Infelizmente, segundo alguns Empreendedores, esse foi o auge da cidade. Chegou a ser discutida a possibilidade de ser criada uma Associação que melhor representasse o Ecossistema, principalmente perante ao Estado, mas o grupo optou por não seguir esse caminho e coincidentemente após esse momento alguns dos líderes do Ecossistema deixaram o país, alguns para serem acelerados no Vale do Silício, e a pessoa chave no Sebrae responsável pelas Startups também fora transferida para outra área. 

Esse é o ponto em que um dos Empreendedores classifica como o momento em que o Ecossistema do Distrito Federal começou a perder sua força, entre meados de 2012 e 2014. Também foi quando nasceu a Associação de Startups e Empreendedores Digitais, criada por outro grupo de Empreendedores, e naturalmente tomou a posição de liderança e referência do Ecossistema mas sem o apoio e a confiança desse antigo grupo.

Um dos Empreendedores menciona que entre 2015 e 2016, o Ecossistema de Startups de Brasília voltou a reagir. Ele não sabe dizer o que encadeou o movimento, mas diz que o cenário definitivamente não é mais o mesmo e voltou a crescer. Com a presença de dois grandes atores de investimento trazendo cerca de R\$ 100 milhões para Startups e três programas de aceleração se estabelecendo esse Empreendedor acredita que o Ecossistema de Brasília logo voltará a ser o que era se os atores se unirem e formarem um grupo forte novamente.

Um dos Empreendedores, em contradição a outros empreendedores, afirma que hoje os Meetups são de maior qualidade. Que em meados de 2012 não costumavam ter assunto e eram apenas ``happy hours'' para que os Empreendedores se encontrassem sem uma grande troca de valor, enquanto hoje ele sente que aprende bastante e que voltou a frequentar esses encontros.

\section{Planejamento para TCC2}
\label{section:cronograma_tcc2}

Por meio de uma chuva de ideias - brainstorming - inicial e indicações de alguns dos Empreendedores selecionados foram mapeadas as seguintes pessoas que poderiam contribuir com o desenvolvimento deste trabalho, os nomes estão indicados nas Tabelas \ref{table:sugestao_de_empreendedores_para_entrevista}, \ref{table:sugestao_de_coworkings_para_entrevista}, \ref{table:sugestao_de_instituicoes_para_entrevista}, \ref{table:sugestao_de_investidores_para_entrevista} e \ref{table:sugestao_de_universidades_para_entrevista} disponíveis no Apêndice \ref{apendices:tabelas_de_entrevistados}, cada tabela para sua respectiva Categoria. 

De acordo com relatos de outros pesquisadores e pelo trabalho com a primeira entrevista formal estimo que cada hora de entrevista gere um trabalho de 3 a 5 horas de transcrição, além do tempo de codificação e interpretação de dados. Felizmente, com softwares como o MaxQDA esse trabalho se torna bem mais fácil, embora ainda bastante trabalhoso. 

Portanto, a meta é que seja realizada uma entrevista a cada duas semanas, e que sua respectifiva transcrição e codificação seja feita nesse período intermediário. Ao todo já foram listados cerca de 53 nomes e a previsão é que sejam feitas cerca de 20 entrevistas, essa lista de Empreendedores sugeridos pode aumentar conforme as entrevistas evoluírem e novos nomes sejam sugeridos. O gráfico de Gantt do planejamento está representado na Figura \ref{figure:gantt_chart_2016_2017}.

\begin{figure}[!htb]
	\centering
	\includegraphics[width=23cm, angle=270]{figuras/gantt_chart_2016_2017}
	\caption{Gráfico de Gantt representando o planejamento deste trabalho até 2017}
	\label{figure:gantt_chart_2016_2017}
\end{figure}