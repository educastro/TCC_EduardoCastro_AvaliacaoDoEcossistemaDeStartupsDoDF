\chapter[Conclusões]{Conclusões}
\label{cap-conclusoes}

Com o resultado das entrevistas é possível entender que o grupo ``\textit{Startup} Brasília'' era formado por figuras que podem ser vistas como lideranças do ecossistema de \textit{startups} do Distrito Federal e são exemplos do grande potencial que um grupo unido, coeso e bem conectado pode ter ao focar esforços em movimentar o ecossistema. O momento onde o DF obteve maior destaque nacional, em meados de 2010 a 2012, foi fruto dessa forte união e da parceria com outras entidades de apoio. Também foi possível identificar que a dispersão e o enfraquecimento desse grupo, entre meados de 2012 e 2014, refletiu em uma ausência de liderança no ecossistema, fator apontado por muitos empreendedores como uma das fraquezas do Distrito Federal. Também foi quando nasceu a Associação de \textit{Startups} e Empreendedores Digitais (ASTEPS), criada por outro grupo de empreendedores que, por meio desta, tenta se tornar uma liderança e referência do ecossistema mas sem o apoio, envolvimento e confiança dos líderes desse antigo grupo, demonstrando um forte conflito dentro da comunidade. A impressão repassada pelos empreendedores entrevistados é de que esse momento não foi orgânico/natural e de que não havia um consenso sobre o que deveria ser feito. Também não foram geradas novas lideranças no momento em que os antigos atores se afastaram, o que a bibliografia define como ``\textit{spin-offs}''. Claramente esse foi um ponto de ruptura.

Traçando um paralelo histórico do que alguns empreendedores apontaram como momentos cruciais para o ecossistema de \textit{startups} do Distrito Federal esses pontos de quebra se tornam bem claros, alguns indicaram a ida da \textit{startup} Rota dos Concursos para o programa de aceleração da ``500 \textit{Startups}'' como o primeiro momento de destaque do DF, ela foi a primeira \textit{startup} brasileira aceita no programa, em 2011, e responsável por abrir portas para diversos outros empreendedores na aceleradora e no vale do silício. Em seguida tivemos o fortalecimento do grupo ``\textit{Startup} Brasília'', entre 2011 e 2012, marcado pelo surgimento de diversos meetups, palestras, ``\textit{startup weekends}'' e um programa próprio chamado ``\textit{startup dojo}''. Essas movimentações resultaram em uma grande viagem de diversos empreendedores para o Vale do Silício (o momento em que Brasília foi destaque brasileiro no evento ``\textit{Tech Crunch Disrupt}''), um estreitamento de laços entre empreendedores do DF com outras lideranças de outros estados, etc. Também foi quando surgiu a Associação de \textit{Startups} e Empreendedores Digitais (ASTEPS), em 2012, e o antigo grupo de líderes se afastou. Sem a participação dos antigos atores o ecossistema de \textit{startups} do Distrito Federal claramente perdeu sua força e passou alguns anos sem maiores destaques.

Até que, em meados de 2015, surgiu o anúncio da venda da \textit{startup} brasiliense \textit{ZeroPaper} para a empresa americana \textit{Intuit}, relatado por alguns como a maior venda dentre \textit{startups} do Distrito Federal. Pouco depois, em 2016, começou a entrada de novos investidores e aceleradoras no mercado local, trazendo uma nova dinâmica para as \textit{startups} locais.

Essa falta de lideranças bem definidas e atuantes no Distrito Federal também mostra sinais de que impacta a atuação das aceleradoras e grandes empresas do DF por não haver sinais claros de comunicação e colaboração entre esses atores, o SEBRAE DF e a ASTEPS por algum tempo fomentaram reuniões mensais entre alguns dos atores do ecossistema do Distrito Federal mas conforme os encontros foram acontecendo e os resultados não foram se desenvolvendo e deixaram de acontecer. A percepção obtida é de que esse grupo poderia demonstrar um grande potencial de movimentação no ecossistema local de diversas maneiras como, por exemplo, abrir momentos para que cada ator demonstre suas ações no ecossistema (e impeça situações com bons eventos acontecendo de forma simultânea, algo recorrente no Distrito Federal) e abertura para que esses atores colaborem entre si para um maior impacto dessas ações. Uma forte resistência para a criação desses momentos entre os atores do Distrito Federal também se tornou clara, talvez fruto dos baixos resultados, falta de confiança e/ou de conflitos pessoais.

Segundo relatos, o brasiliense também parece optar pela estabilidade do emprego fixo ao invés do risco, sendo este um ponto criticado por vários dos entrevistados, e reflexo para uma baixa cultura empreendedora, este talvez seja o indicador do Distrito Federal que requer uma maior concentração de esforços para que melhore, em especial entre jovens e estudantes universitários. 

Ações para o fortalecimento da cultura empreendedora trabalhados de forma alinhada e conectada aos diversos atores que compõem o ecossistema empreendedor do Distrito Federal podem formar conexões entre empreendedores em busca de novos funcionários ou sócios, fomentar a criação de novos negócios e parcerias e uma cultura de colaboração entre as empresas e os empreendedores locais, disseminar as histórias de sucesso e inspirar e motivar novas gerações de empreendedores, criar situações em que novos empreendedores recebem apoio ou mentoria de empreendedores experientes, investidores tem a oportunidade de visualizar a movimentação do ecossistema em busca de investimentos promissores, professores universitários podem estar alinhados com a expectativa dessas empresas para profissionais e usufruir do potencial do ecossistema durante seus trabalhos de exploração científica. Com as entrevistas notou-se que ``\textit{meetups}'', ``\textit{startup weekends}'', espaços de \textit{coworking}, eventos e feiras de tecnologia e empreendedorismo, palestras, cursos e oficinas e momentos em que a comunidade possa participar do cotidiano das universidades, das aceleradoras e/ou das \textit{startups} são capazes de contribuir para a formação e o fortalecimento dessas conexões. Foi observado que uma das características de bons ecossistemas empreendedores envolve o alto número de conexões que são geradas com frequência e quanto mais espaços e momentos para que essas pessoas possam se juntar melhor para o ecossistema como um todo.

Um dos acontecimentos que trazem otimismo ao cenário universitário do Distrito Federal são as instituições que demonstraram esforços, ainda que incipientes, para tentar criar um ambiente favorável ao empreendedorismo. Quatro das grandes instituições de ensino superior do Distrito Federal contam com incubadoras de empresas em sua estrutura, atraem empreendedores para atuarem como mentores e realizar palestras, possuem programas educacionais e disciplinas com foco em formação de capacidades empreendedoras, etc. O IESB, por exemplo, promove ``\textit{meetups}'' semanais abertos ao público e oferece uma espécie de ``\textit{coworking}'' para empreendedores em um espaço chamado de ``IESB \textit{Lab}'', lá também já promoveram ciclos de pré-aceleração de \textit{startups} em parceria com empreendedores e aceleradoras do DF. Outros destaques mencionados como grandes destaques foram o UniCeub, com o programa de aceleração de \textit{startups} Impulso, e a Universidade Católica de Brasília, com o programa de formação de desenvolvedores de aplicativos para dispositivos móveis \textit{BEPID}.

As entrevistas também demonstraram o potencial do Distrito Federal por se tratar do centro governamental do Brasil, podendo se tornar um cenário favorável tanto no sentido burocrático e de ambiente regulatório como também no sentido mercadológico, devido a presença de organizações com um grande poder de compra (entidades governamentais, organismos internacionais, bancos, grandes empresas prestadoras de serviço, etc.) e possíveis investidores. Foi observado também que se trata de um bom mercado consumidor, tendo o maior PIB per capita, ótimos indicadores educacionais e os maiores índices de conectividade de banda larga e tele densidade do país.

Um ponto que foi observado como dicotômico nas entrevistas refere-se a atuação dos investidores no ecossistema local, enquanto empreendedores mais experientes e investidores relataram que há capital em abundância disponível no Distrito Federal mas poucos empreendedores preparados e maduros o suficiente. Por outro lado, empreendedores com menor experiência relataram grande dificuldade para encontrar um investimento anjo ou acesso a potenciais investidores institucionais. Mais de um entrevistado relatou que o empreendedor do DF não está preparado para mostrar o que o investir quer e precisa ver para sentir segurança e confiança no investimento de alto risco, muito ainda se fala sobre ideias e propostas abstratas e não validadas, produtos e futuros idealizados e pouco sobre dados reais, indicadores de crescimento, retenção de usuários, faturamento, maturidade e sinergia da equipe, etc. Esse também é um ponto que merece forte atenção para que seja trabalhado entre os empreendedores de forma que estejam mais preparados para lidar com potenciais investidores e, juntamente com o desenvolvimento de uma maior cultura empreendedora do Distrito Federal e a formação de lideranças capazes de formar induzir conexões de qualidade, diminuir as barreiras que distanciam investidores e empreendedores.

O modelo de trabalho das tradicionais prestadoras de serviço e fábricas de software para o governo também parecem trazer impactos negativos na formação dos profissionais de tecnologia no ecossistema estudado, foi observado pelas entrevistas dificuldades em contratar profissionais preparados para trabalhar em ambientes dinâmicos, flexíveis e ágeis como \textit{startups}. Um destaque muito elogiado por alguns dos empreendedores foi a Universidade de Brasília (UnB) pela formação de alunos com uma forte experiência em gestão de projetos ágeis e preparo em linguagens modernas para desenvolvimento web e móvel (\textit{Ruby on Rails}, \textit{Ionic}, \textit{React}, \textit{Swift}, \textit{Android}, etc.). O Movimento Empresa Júnior, muito forte na UnB, também foi mencionado como uma boa fonte de profissionais e possíveis sócios no ambiente universitário.

\section{Limitações}
\label{subsection:limitacoes}

A principal limitação encontrada durante a execução desse trabalho foi o baixo número de dados disponíveis sobre o ecossistema de startups do Distrito Federal, uma das características de ecossistemas com um baixo grau de maturidade apontada por \citeonline{Kon2014}, a metodologia proposta pelo pesquisador também toma como referência para a cidade de São Paulo alguns indicadores nacionais, o que pode refletir bem a realidade de um ecossistema forte para padrões nacionais como o de São Paulo mas não se mostraram condizentes com ecossistemas menos desenvolvidos, como o do Distrito Federal.

Essa limitação refletiu na supressão de 13 indicadores sugeridos pela metodologia completa e impediu a mensuração de um nível de maturidade geral para o ecossistema de acordo com as regras definidas por \citeonline{Kon2014} mas não impactou na exploração qualitativa do estudo.

Como sugestão para amadurecimento da metodologia proposta por Kon e uma melhor aplicação em ecossistemas brasileiros, sugere-se a utilização de indicadores que tenham como referência dados locais e não nacionais, como é caso do indicador ``Valores culturais para o empreendedorismo'' que neste trabalho foi substituído pelo indicador ``Cultura Empreendedora'' que tem como base o Índice de Cidades Empreendedoras, da Endeavor, e um foco maior na versão enxuta por melhor se adequar a realidades com poucos dados e estudos disponíveis. 

Com foco na versão enxuta, o indicador ``Empreendedorismo nas Universidades'', por exemplo, tem como base a porcentagem de ex-alunos universitários que fundaram uma empresa em até 5 anos após a graduação, este dado por si só justifica uma pesquisa e um estudo e publicações dedicadas, infelizmente não cabendo ao âmbito dessa pesquisa e provavelmente das que irão se propor a usar essa versão por estarem em busca de uma análise breve e rápida do ecossistema em estudo. Em contrapartida, os indicadores ``Cultura Empreendedora'', como mencionado anteriormente, ``Incubadoras e Parques Tecnológicos'' e ``Ambiente Regulatório'', provenientes da versão completa, podem ser facilmente adicionados à versão enxuta e explorados por ecossistemas brasileiros menos maduros por tratarem de dados de fácil obtenção e/ou que são monitorados anualmente pela equipe responsável pelo Índice de Cidades Empreendedoras, da Endeavor.

Também é notável que o nível de maturidade obtido pelo Distrito Federal não condiz com a maturidade real do ecossistema, dessa forma estaríamos no mesmo nível que São Paulo, o que não é uma realidade por se tratar de um ecossistema mais desenvolvido, com um maior número de \textit{startups}, maior acesso a investimentos, etc. Durante as entrevistas vários empreendedores do Distrito Federal mencionam o ecossistema de São Paulo como uma referência no Brasil e citam casos de \textit{startups} que optaram por se mudar para SP em busca de maiores recursos, apoio, investimento, profissionais mais preparados, etc. O resultado obtido também foi discutido com alguns empreendedores e os mesmos também não acreditam que o Distrito Federal possa ser inserido no mesmo nível que a capital paulista.

Acreditamos que os níveis de maturidade propostos podem funcionar bem em comparações internacionais e/ou ecossistemas maduros, como foi o caso de Tel-Aviv, Nova Iorque e São Paulo, mas, ao inserir ecossistemas menores podemos encontrar inconsistências que contestam a credibilidade da metodologia utilizada. De qualquer forma, o trabalho proposto por \citeonline{Kon2014} mostrou uma boa técnica para a exploração qualitativa e permitiu que desenvolvessemos uma boa percepção das dinâmicas internas do ecossistema do Distrito Federal e, principalmente, de pontos que devem ser melhorados.

\section{Trabalhos futuros}
\label{subsection:trabalhos_futuros}

As limitações retratadas na subseção anterior são reflexo da falta de estudos sistemáticos sobre o ecossistema de \textit{startups} do Distrito Federal, portanto mostra-se uma grande necessidade o acompanhamento do desenvolvimento deste ecossistema e o levantamento de dados que correspondam aos indicadores sugeridos por \citeonline{Kon2014}. Também pode ser de grande valor realizar essa mesma pesquisa em um futuro próximo com o objetivo de avaliar o desenvolvimento do ecossistema. Uma espécie de ``Observatório'' que realize essa pesquisa de forma frequente pode trazer grandes contribuições para o Distrito Federal. Como também mencionado na Subseção anterior (\ref{subsection:limitacoes}) os indicadores propostos por \citeonline{Cukier2016} e \citeonline{Kon2014} não se mostraram adequados para uso em ecossistemas menores e comparações de âmbito nacional ou local, abrindo espaço para um trabalho futuro com o objetivo de adequar a metodologia para esses fins.

Do ponto de vista de estudos com foco em políticas públicas o Distrito Federal se mostra um dos melhores ambientes do país para a execução de pesquisas que tenham como objeto \textit{startups}, visto a grande influência e importância dos órgãos do poder executivo do governo federal e a presença da Câmara dos Deputados e do Senado Federal, todos responsáveis pela construção e execução de políticas públicas de impacto nacional. Além desse fator o Distrito Federal também se mostra um potencial ``laboratório'' para validação de novas políticas, visto que há uma grande proximidade física e política com o governo federal e um ator a menos na rota da burocracia, visto que se trata de um Distrito. Poderia ser de grande valor a elaboração de um levantamento de boas práticas de gestão pública e/ou casos de sucesso com foco no mercado de \textit{startups} com o objetivo de influenciar e contribuir com esses atores. 

De forma análoga, estudos que tenham como objetivo identificar oportunidades e boas práticas para que empreendedores colaborem e vendam para o poder público também podem ser de grande valia, visto que há um mercado muito grande disponível e uma grande demanda por inovação e digitalização de serviços públicos ao mesmo tempo em que pequenos empreendedores relatam uma grande barreira de entrada nesse mercado. A Associação Brasileira de \textit{Startups} (AB \textit{Startups}), em parceria com a Dínamo e o Governo do Estado de São Paulo, e a aceleradora \textit{BrazilLab} vem desenvolvendo esforços com o objetivo de facilitar a entrada de \textit{startups} no setor público e podem ser um bom objeto de estudo.