\chapter[Fundamentação Teórica]{Fundamentação Teórica}
\label{cap-fundamentacao_teorica}

\section{O Empreendedorismo}
\label{section:o_empreendedorismo}

Segundo \cite{McCall2000}, a palavra Empreendedor tem sua origem na França, ainda no século XVII, e tem como um dos seus significados aquele encarregado por executar um determinado trabalho para os outros. Ele também diz que o termo também tem relação com a palavra francesa ``Entreprenant'', usada para descrever uma pessoa forte e audacioso e como alguém que constrói uma visão.

\cite{Brown2013} diz que a primeira citação documentada ao termo foi criada por Richard Cantillon (1697-1734) em seu livro ``Sobre a Natureza do Comércio em Geral''\footciteref{James1953}, o qual define Empreendedorismo como o processo de assumir risco ao adquirir bens a um determinado preço para vendê-los por um valor incerto no futuro, concretizando o lucro ou o prejuízo. Pelo seu contexto histórico enquanto em vida, um autor do século XVIII, muitas das suas visões eram baseadas nos negócios da época e para ele o empreendedor era prioritariamente como um fornecedor de produtos e serviços, e por isso menciona como empreendedores profissionais das mais diversas profissões, como proprietários de minas e teatros, fazendeiros, atacadistas de lã e grãos, mercadores, padeiros, artesãos, carpinteiros, pintores, médicos, advogados e até mesmo cervejeiros. 

Alguns séculos depois, \cite{Schumpeter1934} escreveu que o Empreendedorismo é o principal mecânismo no processo de desenvolvimento econômico e com ele é impossível não abalar o status quo do sistema econômico, ele defende que é graças as ações inovadoras dos empreendedores que nossos sistemas evoluem e são renovados. Outros autores como \cite{Holcombe1998, Acs2006} também defendem o Empreendedorismo como um impulsionador da economia. \cite{McClelland1961} sugere que o Empreendedorismo é responsável pelo avanço da civilicação por instigar o espiríto empreendedor na sociedade, de forma a permitir que explorem e inovem com a combinação dos recursos disponíveis.

\cite{Stevenson1985} diz que Administradores costumam definir ``Empreendedorismo'' com termos como inovador, flexível, dinâmico, tomador de risco e orientado ao crescimento. A mídia, ao promover o sucesso de grandes corporações como a Apple, o relacionam com a criação e expansão de novas empresas. No artigo ``The Heart of Entrepreneurship'' ele defende que o Empreendedorismo está muito mais relacionado à oportunidades do que aos recursos. \cite{Acs2006} diz que históricamente o termo é usado como referência à posse e gestão de uma empresa.

Para \cite{Dornelas2005} Empreendedorismo é o envolvimento de pessoas e processos que, em conjunto, levam à transformação de ideias em oportunidades. E a perfeita implementação destas oportunidades leva à criação de negócios de sucesso.

\cite{Drucker2006}, uma das maiores referências da área de Administração, define Empreendedorismo como uma disciplina que pode ser aprendida e praticada e como um processo para criação e gestão da inovação. 

Para ele inovar é criar uma nova forma de entregar um valor novo para os seus consumidores com os recursos que o empreendedor tem disponível. Com base nessas premissas e na visão de Drucker, se uma pessoa decide criar uma nova, porém tradicional, padaria como diversas outras em uma área residencial qualquer, ela provavelmente não será inovadora, e consequentemente seus criadores não estarão empreendendo, mas replicando modelos de negócios com processos e recursos já conhecidos e maturados por outras pessoas, talvez esses, sim, empreendedores. Ou seja, criar uma nova empresa não necessariamente significa estar empreendendo. Para isso é necessário inovar. 

Essa visão de Drucker pode ser facilmente relacionada a visão de \cite{Thurik2004}, que faz distinção entre empresas empreendedoras e empresas pequenas, com foco em pequenos negócios. \cite{Carland1984} define como ponto chave para essa distinção o crescimento expoencial das empresas empreendedoras, enquanto as pequenas empresas tendem a se manter pequenas por toda a sua vida, mesmo que demonstrem algum pequeno crescimento. 

No mesmo artigo \cite{Carland1984} define que pequenos negócios são quaisquer negócios independentes e não dominantes em seus mercados que não se envolvam com nenhuma prática de marketing ou inovação enquanto empresas empreendedoras são quaisquer negócios que se enquadrem em pelo menos uma das categorias de comportamento de \cite{Schumpeter1934} (novos produtos ou melhoria de produtos existentes, novos métodos de produção, abertura de novos mercados, domínio de novas fontes de fornecimento e matéria-prima ou reorganização industrial). Para ele, uma empresa empreendedora deve ser inovadora, lucrativa e estar em constante expansão.

Quando os irmãos McDonald\footciteref{RichardMcDonald2016} decidiram aplicar conceitos e técnicas de gestão de negócios e de produção com a padronização de seus sanduíches e desenvolvimento de processos de produção padronizados eles inovaram. Conseguiram maximizar os retornos com seus recursos, criaram um novo mercado, definiram um novo padrão para a indústria de alimentos e atualmente alimentam mais de 68 milhões de consumidores em cerca de 36 mil restaurantes em 119 países\footciteref{McDonald2016}, para uma empresa que iniciou operação em 1940 com um único restaurante, esse foi um crescimento muito acelerado.

\cite{Birley1986} define empresas empreendedoras como orgânicas e com um grande enfoque nos relacionamentos ao invés de mecânicas e burocrácias, essa características são muito presentes nas Startups.

\cite{McCall2000} relata que o ``The Entrepreneurship Center at Miami University of Ohio'' define Empreendedorismo como o processo de identificar, desenvolver e dar vida a uma visão. Essa visão pode ser uma idéia inovadora, uma oportunidade ou uma forma melhor de realizar alguma atividade ou serviço.

\cite{Byers2014} defende que o Empreendedorismo envolve a criação de um novo empreendimento que sirva a sociedade e crie mudanças positivas, algo muito além da criação de uma empresa e da geração de riquezas. 

Para \cite{Wallevik2016}, não há um senso comum do significado e do histórico do termo ``Empreendedorismo'', os conceitos variam de acordo com o contexto estudado podendo englobar diversos cenários distintos como a criação e expansão de grandes empresas mas também envolvendo atividades de impacto social, no campo, em pequenos negócios sociais e até mesmo como empregados de grandes corporações, sejam elas privadas ou públicas. 

% Fernando Dolabela Jose Dornelas, Planos de Negócios para Micro e Pequenas Empresas % d
\section{A Startup}
\label{section:as_startups}

Não se sabe ao certo quem criou o termo ``Startup'', \cite{Miranda2015, Brigidi2009} relatam que o termo tem sido usado de maneira ampla em diversos contexto e sem uma definição clara. \cite{Gitahy2010} descreve o termo como um sinônimo para criação de novas empresas e que, embora muito comum nos Estados Unidos da América há muitos anos, começou a ser usado no Brasil após a bolha da internet\footciteref{BolhaDaInternet}.  

\cite{Blank2014} enfatiza que uma Startup não é uma versão menor de uma grande companhia e a define como uma organização temporária em busca de um modelo de negócio escalável, recorrente e lucrativo. O motivo dessa definição está esclarecido na seção sobre Ciclo de Vida.

Para \cite{Ries2011}, uma Startup é uma instituição humana projetada para criar novos produtos e serviços sob condições de extrema incerteza. Ries não fala sobre tamanho, setor ou indústria e afirma que Startups podem co-existir até mesmo dentro de grandes corporações. Para ele, o maior objetivo de uma Startup é descobrir qual o produto certo que os consumidores queiram e estejam dispostos a comprar, e/ou usar, o mais rápido possível.

\cite{Graham2012} diz que o único fator essencial é o seu crescimento, que qualquer outro nada mais é do que um reflexo deste e que não é necessário que o trabalho seja relacionado à tecnologia ou receba investimento para que seja considerado uma Startup.

Em uma entrevista documentada por \cite{Robehmed2013} o Empreendedor Neil Blumenthal definiu como uma companhia trabalhando para resolver um problema o qual a solução não é óbvia e o sucesso incerto. Na rede social de perguntas e respostas Quora\footciteref{Quora} o Empreendedor Dave McClure definiu Startups como empresas confusas em relação ao que é o seu produto, quem é o seu cliente e como monetizar com sua solução e que logo após obter resposta para essas três perguntas elas deixam de ser uma Startup e se tornam negócios reais.

Alguns especialistas tentam definir métricas para traçar o momento em que esses novos empreendimentos deixem de ser classificados como Startups. \cite{Wilhelm2014} sugere a regra dos ``50, 100 ou 500'': US\$50 milhões em vendas nos últimos 12 meses, 100 ou mais empregados e valor de mercado avaliado em mais de US\$500 milhões.   

No primeiro Edital do Programa Startups Brasilia\footciteref{StartupBrasilia2015} do Governo de Brasília para subvenção de projetos de inovação tecnológica a Fundação de Apoio à Pesquisa do Distrito Federal define Startups como empresas cujo faturamento anual seja inferior R\$ 3,6 milhões e possuam menos de quatro anos de existência. 

A Fundação Carlos Chaga de Amparo à Pesquisa do Estado do Rio de Janeiro(FAPERJ) com o programa Startup Rio\footciteref{StartupRio2015} classifica como Startups as Micro Empresas com potencial de crescer rapidamente e iniciar operação em outros estados e países em poucos meses de atividade. Por esse motivo defendem que o termo é comumente utilizado para Micro Empresas de base tecnológica, por não possuirem tantas barreiras logísticas que impeçam uma expansão tão grande e tão rápido.

A Financiadora de Projetos e Pesquisa(FINEP)\footciteref{Finep2016}, outra entidade pública do Brasil para fomento à inovação, define Startup como uma Empresa Nascente de Base Tecnológica sujeita a frequentes mudanças e que tem como sua maior sustenção a inovação. Para a FINEP Startups possuem uma estrutura empresarial(uma ``quase empresa'', como eles definem em seu Glossário), não possuem posição definida no mercado e busca por oportunidades com produtos de alto valor agregado.

\section{O Ciclo de Vida}
\label{section:o_ciclo_de_vida}





\section{O Empreendedor}
\label{section:o_empreendedor}

Assim como em relação ao \cite{section:o_empreendedorismo}, ainda não há um consenso em relação ao significado do termo ``Empreendedor'' na Academia, como citado por \cite{Fernald2005} os autores sugerem uma série de critérios que envolvem criatividade, inovação, características pessoais e até mesmo aparência e estilo. O mesmo autor classifica Empreendedores como pessoas que tiram vantagens e conseguem obter valor das oportunidades que surgem.

\cite{Stevenson1985} diz que Empreendedores não são apenas seres oportunitas, mas também criativos e inovadoras. \cite{Drucker2006} relata que estão sempre em busca por mudanças, respodem 

Para \cite{Byers2014} Empreendedores são pessoas que identificam soluções entre problemas, possibilidades entre necessidades e oportunidades entre desafios, de forma a criar ótimas empresas que demonstram competência, liderança e longetividade.

https://www.entrepreneur.com/article/233919

The Entrepreneur always searches for change, responds to it, and exploits it as an opportunity.”  — Peter Drucker

an entrepreneur is someone who perceives an opportunity to define and creates an organization to pursue it http://www.slideshare.net/namekuwanday/theorizing-about-entrepreneurship-45779199

 mas o autor defende um sentido mais amplo ao fazer referência ao comportamento empreendedor de estar sempre em busca de uma nova oportunidade para Empreender, mas não necessariamente com a criação de um novo negócio. acs

while there are fewer
entrepreneurs in the innovation-driven economies, those
that do exist are more likely to affect their societies
through growth, innovation and internationalization. gem 2011

 From this
perspective entrepreneurial activities can furnish an excellent mechanism for upward
social mobility, that is the achievement of a higher social status for the individual and his
or her family (p 156)

Entrepreneurs’ main distinctive features are linked with their courage for ‘carrying
out a new plan’ even if they do not have complete knowledge11 of the market situation;
“the success of everything depends upon intuition, the capacity of seeing things in a way
which afterwards proves to be true… The more accurately, however, we learn to know
the natural and social world, the more perfect our control of facts becomes, and the
greater the extent, with time and progressive rationalization, within which things can be
simply calculated, and indeed quickly and reliably calculated, the more the significance of
this function decreases”(p 85-86)

Schackle wrote,
“The entrepreneur is a maker of history, but his guide in making it is his judgment of possibilities
and not a calculation of certainties.	

Jean Baptise Say (1767-1832) improved Cantillion’s definition by adding that the entrepreneur brings people together to build a productive item.

\section{O dinâmico mercado das Startups}

\cite{James1953} fez uma análise que demonstra que o valor intríseco de um determinado produto, o seu valor de venda, está associado ao custo de oportunidade e não ao seu custo de produção. Dessa forma ele mostra como a economia poderia ser auto-regulada, visto que os empreendedores precisam respeitar os sinais do mercado para determinar o valor de um produto e essa é, para ele, a base do empreendedorismo e se mostra muito presente no mundo mesmo após 300 anos, principalmente entre algumas Startups e seu altíssimo valor de mercado mesmo quando ainda não há faturamento, como relatado por \cite{Luckerson2013}.

shumpeter argued that anyone seeking profits must innovate


\section{O Ecossistema}
\label{section:ecossistemas_e_suas_pecas}

\cite{James1953} também diz que o crescimento econômico, a formação e o crescimento de cidades está diretamente ligado ao Empreendedorismo e as decisões que são tomadas por Empreendedores

While government agencies and educational institutions can create conditions favorable for entrepreneurship to take hold, it is up to individual organizations to foster the conditions that allow it to flourish the heart of entrepreneurship

schumpeter emphatizes the role of the entrepreneur as a prime cause of economic development,  thurik

\section{O Brasil}

“I am a huge believer in the idea that starting during a downturn is the best time to start,” says HBS Senior Lecturer Janet J. Kraus. “Opportunity costs are low, and if you’re able to turn a profit in a down market, then you will be very profitable when markets recover.”



\section{Trabalhos Relacionados}
\label{section:trabalhos_relacionados}

