\chapter[Fundamentação Teórica]{Fundamentação Teórica}
\label{cap-fundamentacao_teorica}

\section{O Empreendedor}
\label{section:o empreendedor}

\cite{Brown2013} acredita que a primeira citação documentada ao termo foi criada por Richard Cantillon (1697-1734) em seu livro ``Sobre a Natureza do Comércio em Geral''\footciteref{James1953}, o qual define Empreendedorismo como o processo de assumir o risco ao adquirir bens a um determinado preço para vendê-los por um valor incerto, concretizando o lucro ou o prejuízo. 

Pelo seu contexto histórico enquanto em vida, um autor do século XVIII, muitas das suas visões eram baseadas nos negócios da época, e por isso menciona como empreendedores pessoas com as mais diversas profissões, como proprietários de minas e teatros, fazendeiros, atacadistas de lã e grãos, mercadores, padeiros, artesãos, carpinteiros, pintores, médicos, advogados e até mesmo cervejeiros. 

\cite{Murphy1987} diz que as contribuições de Richard Cantillon ainda são muito relevantes para a Administração e para o desenvolvimento de diversas teorias que envolvem economia monetária e , \cite{Brown2013} o descreve como o pai da teoria economica. \cite{James1953} fez uma análise que demonstra que o valor intríseco de um determinado produto, o seu valor de venda, está associado ao custo de oportunidade e não ao seu custo de produção. Dessa forma ele mostra como a economia poderia ser auto-regulada, visto que os empreendedores precisam respeitar os sinais do mercado para determinar o valor de um produto. \cite{Brown2013} em sua análise sobre o livro em questão relata que Cantillon, além de questões envolvendo valores de mercado, também aborda a forma como a quantidade de dinheiro disponível pode afetar as taxas de juros, comércio exterior, taxas de câmbio, bancos, inflação e ciclos de negócio.

Como dito por \cite{Wallevik2016}, não há um senso comum do significado do termo ``Empreendedorismo'', os conceitos variam de acordo com o contexto estudado podendo englobar diversos cenários distintos como a criação e expansão de grandes empresas mas também envolvendo atividades de impacto social, no campo, em pequenos negócios sociais e até mesmo como empregados de grandes corporações, sejam elas privadas ou públicas.

\cite{Drucker2006}, uma das maiores referências da área de Administração, define Empreendedorismo como uma disciplina que pode ser aprendida e praticada e como o processo para criação e gestão da inovação. Para ele inovar é criar uma nova forma de gerar recursos com os recursos disponíveis. 

Com base nessas premissas e na visão de Drucker, se uma pessoa decide criar uma nova e tradicional padaria como diversas outras em uma área residencial qualquer, ela provavelmente não estará inovando, e consequentemente não estará empreendendo empreendendo, mas replicando modelos de negócios e processos conhecidos e já maturados por outras pessoas, talvez esses, sim, empreendedores, utilizando recursos similares. Ou seja, ao criar uma empresa não necessariamente significa estar empreendendo, para isso, é necessário inovar.


\section{Startups e Modelos de Negócio}
\label{section:startups_e_modelos_de_negocio}

\section{Ecossistemas e seus elementos}
\label{section:ecossistemas_e_suas_pecas}

\section{Trabalhos Relacionados}
\label{section:trabalhos_relacionados}

