\chapter[Fundamentação Teórica]{Fundamentação Teórica}
\label{cap-fundamentacao_teorica}

\section{O Empreendedorismo}
\label{section:o_empreendedorismo}

Segundo \cite{McCall2000}, a palavra Empreendedor tem sua origem na França, ainda no século XVII, e tem como um dos seus significados aquele encarregado por executar um determinado trabalho para os outros. Ele também diz que o termo também tem relação com a palavra francesa ``Entreprenant'', usada para descrever uma pessoa forte e audacioso e como alguém que constrói uma visão.

\cite{Brown2013} diz que a primeira citação documentada ao termo foi criada por Richard Cantillon (1697-1734) em seu livro ``Sobre a Natureza do Comércio em Geral''\footciteref{James1953}, o qual define Empreendedorismo como o processo de assumir risco ao adquirir bens a um determinado preço para vendê-los por um valor incerto no futuro, concretizando o lucro ou o prejuízo. Pelo seu contexto histórico enquanto em vida, um autor do século XVIII, muitas das suas visões eram baseadas nos negócios da época e para ele o empreendedor era prioritariamente como um fornecedor de produtos e serviços, e por isso menciona como empreendedores profissionais das mais diversas profissões, como proprietários de minas e teatros, fazendeiros, atacadistas de lã e grãos, mercadores, padeiros, artesãos, carpinteiros, pintores, médicos, advogados e até mesmo cervejeiros. 

Alguns anos depois, \cite{Schumpeter1934} defendeu que o Empreendedorismo é o principal mecânismo no processo de desenvolvimento econômico e que é impossível que o status quo do sistema econômico não seja abalado, é graças à ações inovadoras que nossos sistemas evoluem e são renovados, outros autores como \cite{Holcombe1998} e 

\cite{Stevenson1985} diz que Administradores costumam definir o Empreendedorismo com termos como inovador, flexível, dinâmico, tomador de risco e orientado ao crescimento. A mídia, ao promover o sucesso de grandes corporações como a Apple, o relacionam com a criação e expansão de novas empresas. No artigo ``The Heart of Entrepreneurship'' ele defende que o Empreendedorismo está muito mais relacionado à oportunidades do que aos recursos. \cite{Acs2006} diz que históricamente se refere à posse e gestão de uma empresa mas em um sentido mais amplo faz referência ao comportamento empreendedor de estar sempre em busca de uma nova oportunidade para Empreender, mas não necessariamente com a criação de um novo negócio.

\cite{Drucker2006}, uma das maiores referências da área de Administração, define Empreendedorismo como uma disciplina que pode ser aprendida e praticada e como o processo para criação e gestão da inovação. Para ele inovar é criar uma nova forma de gerar recursos com os recursos disponíveis. Com base nessas premissas e na visão de Drucker, se uma pessoa decide criar uma nova e tradicional padaria como diversas outras em uma área residencial qualquer, ela provavelmente não estará inovando, e consequentemente não estará empreendendo empreendendo, mas replicando modelos de negócios e processos conhecidos e já maturados por outras pessoas, talvez esses, sim, empreendedores, utilizando recursos similares. Ou seja, ao criar uma empresa não necessariamente significa estar empreendendo, para isso, é necessário inovar.

O ``The Entrepreneurship Center at Miami University of Ohio'' define Empreendedorismo como o processo de identificar, desenvolver e dar vida a uma visão. Essa visão pode ser uma idéia inovadora, uma oportunidade ou uma forma melhor de realizar alguma atividade ou serviço.

Como dito por \cite{Wallevik2016}, não há um senso comum do significado e do histórico do termo ``Empreendedorismo'', os conceitos variam de acordo com o contexto estudado podendo englobar diversos cenários distintos como a criação e expansão de grandes empresas mas também envolvendo atividades de impacto social, no campo, em pequenos negócios sociais e até mesmo como empregados de grandes corporações, sejam elas privadas ou públicas. 



% Fernando Dolabela Jose Dornelas, Planos de Negócios para Micro e Pequenas Empresas % d

O Empreendedor 

Entrepreneurs are not just opportunistic; they are also creative and innovative. stevenson1985

The Entrepreneur always searches for change, responds to it, and exploits it as an opportunity.”  — Peter Drucker

an entrepreneur is someone who perceives an opportunity to define and creates an organization to pursue it http://www.slideshare.net/namekuwanday/theorizing-about-entrepreneurship-45779199

while there are fewer
entrepreneurs in the innovation-driven economies, those
that do exist are more likely to affect their societies
through growth, innovation and internationalization. gem 2011

 From this
perspective entrepreneurial activities can furnish an excellent mechanism for upward
social mobility, that is the achievement of a higher social status for the individual and his
or her family (p 156)

Entrepreneurs’ main distinctive features are linked with their courage for ‘carrying
out a new plan’ even if they do not have complete knowledge11 of the market situation;
“the success of everything depends upon intuition, the capacity of seeing things in a way
which afterwards proves to be true… The more accurately, however, we learn to know
the natural and social world, the more perfect our control of facts becomes, and the
greater the extent, with time and progressive rationalization, within which things can be
simply calculated, and indeed quickly and reliably calculated, the more the significance of
this function decreases”(p 85-86)

\section{Startups e Modelos de Negócio}
\label{section:startups_e_modelos_de_negocio}

\cite{James1953} fez uma análise que demonstra que o valor intríseco de um determinado produto, o seu valor de venda, está associado ao custo de oportunidade e não ao seu custo de produção. Dessa forma ele mostra como a economia poderia ser auto-regulada, visto que os empreendedores precisam respeitar os sinais do mercado para determinar o valor de um produto e essa é, para ele, a base do empreendedorismo e se mostra muito presente no mundo mesmo após 300 anos, principalmente entre algumas Startups e seu altíssimo valor de mercado mesmo quando ainda não há faturamento, como relatado por \cite{Luckerson2013}.

shumpeter argued that anyone seeking profits must innovate


\section{Ecossistemas e seus elementos}
\label{section:ecossistemas_e_suas_pecas}

\cite{James1953} também diz que o crescimento econômico, a formação e o crescimento de cidades está diretamente ligado ao Empreendedorismo e as decisões que são tomadas por Empreendedores

While government agencies and educational institutions can create conditions favorable for entrepreneurship to take hold, it is up to individual organizations to foster the conditions that allow it to flourish the heart of entrepreneurship

schumpeter emphatizes the role of the entrepreneur as a prime cause of economic development,  thurik


\section{Trabalhos Relacionados}
\label{section:trabalhos_relacionados}

